\section{Analyse technique et macroéconomique du blé et du nickel}
\subsection{Présentation des deux matières premières}
\subsubsection{Le blé meunier}
Le travail porte tout d'abord le cours du contrat à terme de blé de meunerie, qui est coté en euros sur le marché des futures à la bourse de Paris (Euronext). 
Le contrat est basé sur du blé d'origine de l'Union Européenne et la taille d'un lot est de 50 tonnes. \\[11pt]
Le blé meunier est une variété de blé tendre (\textit{Triticum aestivum} ) est la variété de blé la plus couramment cultivée dans les régions tempérées 
du monde. Le blé meunier est particulièrement apprécié pour sa concentration élevée en gluten, ce dernier donne à la pâte de blé une texture élastique et une capacité à 
lever. La farine faite à partir du blé meunier est majoritairement utilisée pour la fabrication du pain, pâtisseries et d'autres denrées a base de farine. Célèbre comptine 
du patrimoine français : meunier tu dors tirée d'une chanson de Léon Raiter et de Fernand Pothier composée au XXème siècle.\\[11pt]
Au niveau de l'agriculture mondiale, la culture du blé est l'une des céréales les plus largement cultivées avec le riz et le mais. Parmi les principaux
producteurs de blé : la Chine, l'Inde, la Russie, les États-Unis d'Amérique.\footcite{fao_2021}.
\begin{figure}[H]
    \centering
    \label{fig:ble_prod}
    \resizebox{0.8\textwidth}{!}{%% Creator: Matplotlib, PGF backend
%%
%% To include the figure in your LaTeX document, write
%%   \input{<filename>.pgf}
%%
%% Make sure the required packages are loaded in your preamble
%%   \usepackage{pgf}
%%
%% Also ensure that all the required font packages are loaded; for instance,
%% the lmodern package is sometimes necessary when using math font.
%%   \usepackage{lmodern}
%%
%% Figures using additional raster images can only be included by \input if
%% they are in the same directory as the main LaTeX file. For loading figures
%% from other directories you can use the `import` package
%%   \usepackage{import}
%%
%% and then include the figures with
%%   \import{<path to file>}{<filename>.pgf}
%%
%% Matplotlib used the following preamble
%%   \usepackage{fontspec}
%%   \setmainfont{DejaVuSerif.ttf}[Path=\detokenize{C:/Users/Joseph/miniconda3/Lib/site-packages/matplotlib/mpl-data/fonts/ttf/}]
%%   \setsansfont{arial.ttf}[Path=\detokenize{C:/Windows/Fonts/}]
%%   \setmonofont{DejaVuSansMono.ttf}[Path=\detokenize{C:/Users/Joseph/miniconda3/Lib/site-packages/matplotlib/mpl-data/fonts/ttf/}]
%%
\begingroup%
\makeatletter%
\begin{pgfpicture}%
\pgfpathrectangle{\pgfpointorigin}{\pgfqpoint{5.558596in}{3.845283in}}%
\pgfusepath{use as bounding box, clip}%
\begin{pgfscope}%
\pgfsetbuttcap%
\pgfsetmiterjoin%
\definecolor{currentfill}{rgb}{1.000000,1.000000,1.000000}%
\pgfsetfillcolor{currentfill}%
\pgfsetlinewidth{0.000000pt}%
\definecolor{currentstroke}{rgb}{1.000000,1.000000,1.000000}%
\pgfsetstrokecolor{currentstroke}%
\pgfsetdash{}{0pt}%
\pgfpathmoveto{\pgfqpoint{-0.000000in}{0.000000in}}%
\pgfpathlineto{\pgfqpoint{5.558596in}{0.000000in}}%
\pgfpathlineto{\pgfqpoint{5.558596in}{3.845283in}}%
\pgfpathlineto{\pgfqpoint{-0.000000in}{3.845283in}}%
\pgfpathlineto{\pgfqpoint{-0.000000in}{0.000000in}}%
\pgfpathclose%
\pgfusepath{fill}%
\end{pgfscope}%
\begin{pgfscope}%
\pgfsetbuttcap%
\pgfsetmiterjoin%
\definecolor{currentfill}{rgb}{1.000000,1.000000,1.000000}%
\pgfsetfillcolor{currentfill}%
\pgfsetlinewidth{0.000000pt}%
\definecolor{currentstroke}{rgb}{0.000000,0.000000,0.000000}%
\pgfsetstrokecolor{currentstroke}%
\pgfsetstrokeopacity{0.000000}%
\pgfsetdash{}{0pt}%
\pgfpathmoveto{\pgfqpoint{0.684066in}{0.665283in}}%
\pgfpathlineto{\pgfqpoint{5.334066in}{0.665283in}}%
\pgfpathlineto{\pgfqpoint{5.334066in}{3.745283in}}%
\pgfpathlineto{\pgfqpoint{0.684066in}{3.745283in}}%
\pgfpathlineto{\pgfqpoint{0.684066in}{0.665283in}}%
\pgfpathclose%
\pgfusepath{fill}%
\end{pgfscope}%
\begin{pgfscope}%
\definecolor{textcolor}{rgb}{0.150000,0.150000,0.150000}%
\pgfsetstrokecolor{textcolor}%
\pgfsetfillcolor{textcolor}%
\pgftext[x=0.752371in, y=0.265524in, left, base,rotate=25.000000]{\color{textcolor}\sffamily\fontsize{11.000000}{13.200000}\selectfont Chine}%
\end{pgfscope}%
\begin{pgfscope}%
\definecolor{textcolor}{rgb}{0.150000,0.150000,0.150000}%
\pgfsetstrokecolor{textcolor}%
\pgfsetfillcolor{textcolor}%
\pgftext[x=1.263514in, y=0.308558in, left, base,rotate=25.000000]{\color{textcolor}\sffamily\fontsize{11.000000}{13.200000}\selectfont Inde}%
\end{pgfscope}%
\begin{pgfscope}%
\definecolor{textcolor}{rgb}{0.150000,0.150000,0.150000}%
\pgfsetstrokecolor{textcolor}%
\pgfsetfillcolor{textcolor}%
\pgftext[x=1.651642in, y=0.236866in, left, base,rotate=25.000000]{\color{textcolor}\sffamily\fontsize{11.000000}{13.200000}\selectfont Russie}%
\end{pgfscope}%
\begin{pgfscope}%
\definecolor{textcolor}{rgb}{0.150000,0.150000,0.150000}%
\pgfsetstrokecolor{textcolor}%
\pgfsetfillcolor{textcolor}%
\pgftext[x=2.185908in, y=0.301464in, left, base,rotate=25.000000]{\color{textcolor}\sffamily\fontsize{11.000000}{13.200000}\selectfont USA}%
\end{pgfscope}%
\begin{pgfscope}%
\definecolor{textcolor}{rgb}{0.150000,0.150000,0.150000}%
\pgfsetstrokecolor{textcolor}%
\pgfsetfillcolor{textcolor}%
\pgftext[x=2.577788in, y=0.233272in, left, base,rotate=25.000000]{\color{textcolor}\sffamily\fontsize{11.000000}{13.200000}\selectfont France}%
\end{pgfscope}%
\begin{pgfscope}%
\definecolor{textcolor}{rgb}{0.150000,0.150000,0.150000}%
\pgfsetstrokecolor{textcolor}%
\pgfsetfillcolor{textcolor}%
\pgftext[x=3.019700in, y=0.211739in, left, base,rotate=25.000000]{\color{textcolor}\sffamily\fontsize{11.000000}{13.200000}\selectfont Ukraine}%
\end{pgfscope}%
\begin{pgfscope}%
\definecolor{textcolor}{rgb}{0.150000,0.150000,0.150000}%
\pgfsetstrokecolor{textcolor}%
\pgfsetfillcolor{textcolor}%
\pgftext[x=3.453904in, y=0.183019in, left, base,rotate=25.000000]{\color{textcolor}\sffamily\fontsize{11.000000}{13.200000}\selectfont Australie}%
\end{pgfscope}%
\begin{pgfscope}%
\definecolor{textcolor}{rgb}{0.150000,0.150000,0.150000}%
\pgfsetstrokecolor{textcolor}%
\pgfsetfillcolor{textcolor}%
\pgftext[x=3.922724in, y=0.186581in, left, base,rotate=25.000000]{\color{textcolor}\sffamily\fontsize{11.000000}{13.200000}\selectfont Pakistan}%
\end{pgfscope}%
\begin{pgfscope}%
\definecolor{textcolor}{rgb}{0.150000,0.150000,0.150000}%
\pgfsetstrokecolor{textcolor}%
\pgfsetfillcolor{textcolor}%
\pgftext[x=4.410745in, y=0.208051in, left, base,rotate=25.000000]{\color{textcolor}\sffamily\fontsize{11.000000}{13.200000}\selectfont Canada}%
\end{pgfscope}%
\begin{pgfscope}%
\definecolor{textcolor}{rgb}{0.150000,0.150000,0.150000}%
\pgfsetstrokecolor{textcolor}%
\pgfsetfillcolor{textcolor}%
\pgftext[x=4.790753in, y=0.129139in, left, base,rotate=25.000000]{\color{textcolor}\sffamily\fontsize{11.000000}{13.200000}\selectfont Allemagne}%
\end{pgfscope}%
\begin{pgfscope}%
\pgfpathrectangle{\pgfqpoint{0.684066in}{0.665283in}}{\pgfqpoint{4.650000in}{3.080000in}}%
\pgfusepath{clip}%
\pgfsetroundcap%
\pgfsetroundjoin%
\pgfsetlinewidth{1.003750pt}%
\definecolor{currentstroke}{rgb}{0.800000,0.800000,0.800000}%
\pgfsetstrokecolor{currentstroke}%
\pgfsetstrokeopacity{0.400000}%
\pgfsetdash{}{0pt}%
\pgfpathmoveto{\pgfqpoint{0.684066in}{0.665283in}}%
\pgfpathlineto{\pgfqpoint{5.334066in}{0.665283in}}%
\pgfusepath{stroke}%
\end{pgfscope}%
\begin{pgfscope}%
\definecolor{textcolor}{rgb}{0.150000,0.150000,0.150000}%
\pgfsetstrokecolor{textcolor}%
\pgfsetfillcolor{textcolor}%
\pgftext[x=0.467153in, y=0.610602in, left, base]{\color{textcolor}\sffamily\fontsize{11.000000}{13.200000}\selectfont 0}%
\end{pgfscope}%
\begin{pgfscope}%
\pgfpathrectangle{\pgfqpoint{0.684066in}{0.665283in}}{\pgfqpoint{4.650000in}{3.080000in}}%
\pgfusepath{clip}%
\pgfsetroundcap%
\pgfsetroundjoin%
\pgfsetlinewidth{1.003750pt}%
\definecolor{currentstroke}{rgb}{0.800000,0.800000,0.800000}%
\pgfsetstrokecolor{currentstroke}%
\pgfsetstrokeopacity{0.400000}%
\pgfsetdash{}{0pt}%
\pgfpathmoveto{\pgfqpoint{0.684066in}{1.093507in}}%
\pgfpathlineto{\pgfqpoint{5.334066in}{1.093507in}}%
\pgfusepath{stroke}%
\end{pgfscope}%
\begin{pgfscope}%
\definecolor{textcolor}{rgb}{0.150000,0.150000,0.150000}%
\pgfsetstrokecolor{textcolor}%
\pgfsetfillcolor{textcolor}%
\pgftext[x=0.382186in, y=1.038826in, left, base]{\color{textcolor}\sffamily\fontsize{11.000000}{13.200000}\selectfont 20}%
\end{pgfscope}%
\begin{pgfscope}%
\pgfpathrectangle{\pgfqpoint{0.684066in}{0.665283in}}{\pgfqpoint{4.650000in}{3.080000in}}%
\pgfusepath{clip}%
\pgfsetroundcap%
\pgfsetroundjoin%
\pgfsetlinewidth{1.003750pt}%
\definecolor{currentstroke}{rgb}{0.800000,0.800000,0.800000}%
\pgfsetstrokecolor{currentstroke}%
\pgfsetstrokeopacity{0.400000}%
\pgfsetdash{}{0pt}%
\pgfpathmoveto{\pgfqpoint{0.684066in}{1.521730in}}%
\pgfpathlineto{\pgfqpoint{5.334066in}{1.521730in}}%
\pgfusepath{stroke}%
\end{pgfscope}%
\begin{pgfscope}%
\definecolor{textcolor}{rgb}{0.150000,0.150000,0.150000}%
\pgfsetstrokecolor{textcolor}%
\pgfsetfillcolor{textcolor}%
\pgftext[x=0.382186in, y=1.467050in, left, base]{\color{textcolor}\sffamily\fontsize{11.000000}{13.200000}\selectfont 40}%
\end{pgfscope}%
\begin{pgfscope}%
\pgfpathrectangle{\pgfqpoint{0.684066in}{0.665283in}}{\pgfqpoint{4.650000in}{3.080000in}}%
\pgfusepath{clip}%
\pgfsetroundcap%
\pgfsetroundjoin%
\pgfsetlinewidth{1.003750pt}%
\definecolor{currentstroke}{rgb}{0.800000,0.800000,0.800000}%
\pgfsetstrokecolor{currentstroke}%
\pgfsetstrokeopacity{0.400000}%
\pgfsetdash{}{0pt}%
\pgfpathmoveto{\pgfqpoint{0.684066in}{1.949954in}}%
\pgfpathlineto{\pgfqpoint{5.334066in}{1.949954in}}%
\pgfusepath{stroke}%
\end{pgfscope}%
\begin{pgfscope}%
\definecolor{textcolor}{rgb}{0.150000,0.150000,0.150000}%
\pgfsetstrokecolor{textcolor}%
\pgfsetfillcolor{textcolor}%
\pgftext[x=0.382186in, y=1.895274in, left, base]{\color{textcolor}\sffamily\fontsize{11.000000}{13.200000}\selectfont 60}%
\end{pgfscope}%
\begin{pgfscope}%
\pgfpathrectangle{\pgfqpoint{0.684066in}{0.665283in}}{\pgfqpoint{4.650000in}{3.080000in}}%
\pgfusepath{clip}%
\pgfsetroundcap%
\pgfsetroundjoin%
\pgfsetlinewidth{1.003750pt}%
\definecolor{currentstroke}{rgb}{0.800000,0.800000,0.800000}%
\pgfsetstrokecolor{currentstroke}%
\pgfsetstrokeopacity{0.400000}%
\pgfsetdash{}{0pt}%
\pgfpathmoveto{\pgfqpoint{0.684066in}{2.378178in}}%
\pgfpathlineto{\pgfqpoint{5.334066in}{2.378178in}}%
\pgfusepath{stroke}%
\end{pgfscope}%
\begin{pgfscope}%
\definecolor{textcolor}{rgb}{0.150000,0.150000,0.150000}%
\pgfsetstrokecolor{textcolor}%
\pgfsetfillcolor{textcolor}%
\pgftext[x=0.382186in, y=2.323497in, left, base]{\color{textcolor}\sffamily\fontsize{11.000000}{13.200000}\selectfont 80}%
\end{pgfscope}%
\begin{pgfscope}%
\pgfpathrectangle{\pgfqpoint{0.684066in}{0.665283in}}{\pgfqpoint{4.650000in}{3.080000in}}%
\pgfusepath{clip}%
\pgfsetroundcap%
\pgfsetroundjoin%
\pgfsetlinewidth{1.003750pt}%
\definecolor{currentstroke}{rgb}{0.800000,0.800000,0.800000}%
\pgfsetstrokecolor{currentstroke}%
\pgfsetstrokeopacity{0.400000}%
\pgfsetdash{}{0pt}%
\pgfpathmoveto{\pgfqpoint{0.684066in}{2.806402in}}%
\pgfpathlineto{\pgfqpoint{5.334066in}{2.806402in}}%
\pgfusepath{stroke}%
\end{pgfscope}%
\begin{pgfscope}%
\definecolor{textcolor}{rgb}{0.150000,0.150000,0.150000}%
\pgfsetstrokecolor{textcolor}%
\pgfsetfillcolor{textcolor}%
\pgftext[x=0.297218in, y=2.751721in, left, base]{\color{textcolor}\sffamily\fontsize{11.000000}{13.200000}\selectfont 100}%
\end{pgfscope}%
\begin{pgfscope}%
\pgfpathrectangle{\pgfqpoint{0.684066in}{0.665283in}}{\pgfqpoint{4.650000in}{3.080000in}}%
\pgfusepath{clip}%
\pgfsetroundcap%
\pgfsetroundjoin%
\pgfsetlinewidth{1.003750pt}%
\definecolor{currentstroke}{rgb}{0.800000,0.800000,0.800000}%
\pgfsetstrokecolor{currentstroke}%
\pgfsetstrokeopacity{0.400000}%
\pgfsetdash{}{0pt}%
\pgfpathmoveto{\pgfqpoint{0.684066in}{3.234626in}}%
\pgfpathlineto{\pgfqpoint{5.334066in}{3.234626in}}%
\pgfusepath{stroke}%
\end{pgfscope}%
\begin{pgfscope}%
\definecolor{textcolor}{rgb}{0.150000,0.150000,0.150000}%
\pgfsetstrokecolor{textcolor}%
\pgfsetfillcolor{textcolor}%
\pgftext[x=0.297218in, y=3.179945in, left, base]{\color{textcolor}\sffamily\fontsize{11.000000}{13.200000}\selectfont 120}%
\end{pgfscope}%
\begin{pgfscope}%
\pgfpathrectangle{\pgfqpoint{0.684066in}{0.665283in}}{\pgfqpoint{4.650000in}{3.080000in}}%
\pgfusepath{clip}%
\pgfsetroundcap%
\pgfsetroundjoin%
\pgfsetlinewidth{1.003750pt}%
\definecolor{currentstroke}{rgb}{0.800000,0.800000,0.800000}%
\pgfsetstrokecolor{currentstroke}%
\pgfsetstrokeopacity{0.400000}%
\pgfsetdash{}{0pt}%
\pgfpathmoveto{\pgfqpoint{0.684066in}{3.662850in}}%
\pgfpathlineto{\pgfqpoint{5.334066in}{3.662850in}}%
\pgfusepath{stroke}%
\end{pgfscope}%
\begin{pgfscope}%
\definecolor{textcolor}{rgb}{0.150000,0.150000,0.150000}%
\pgfsetstrokecolor{textcolor}%
\pgfsetfillcolor{textcolor}%
\pgftext[x=0.297218in, y=3.608169in, left, base]{\color{textcolor}\sffamily\fontsize{11.000000}{13.200000}\selectfont 140}%
\end{pgfscope}%
\begin{pgfscope}%
\definecolor{textcolor}{rgb}{0.150000,0.150000,0.150000}%
\pgfsetstrokecolor{textcolor}%
\pgfsetfillcolor{textcolor}%
\pgftext[x=0.241662in,y=2.205283in,,bottom,rotate=90.000000]{\color{textcolor}\sffamily\fontsize{11.000000}{13.200000}\selectfont Quantité produite en millions de Tonnes}%
\end{pgfscope}%
\begin{pgfscope}%
\pgfpathrectangle{\pgfqpoint{0.684066in}{0.665283in}}{\pgfqpoint{4.650000in}{3.080000in}}%
\pgfusepath{clip}%
\pgfsetbuttcap%
\pgfsetmiterjoin%
\definecolor{currentfill}{rgb}{0.865196,0.379902,0.315196}%
\pgfsetfillcolor{currentfill}%
\pgfsetlinewidth{1.003750pt}%
\definecolor{currentstroke}{rgb}{1.000000,1.000000,1.000000}%
\pgfsetstrokecolor{currentstroke}%
\pgfsetdash{}{0pt}%
\pgfpathmoveto{\pgfqpoint{0.730566in}{0.665283in}}%
\pgfpathlineto{\pgfqpoint{1.102566in}{0.665283in}}%
\pgfpathlineto{\pgfqpoint{1.102566in}{3.598616in}}%
\pgfpathlineto{\pgfqpoint{0.730566in}{3.598616in}}%
\pgfpathlineto{\pgfqpoint{0.730566in}{0.665283in}}%
\pgfpathclose%
\pgfusepath{stroke,fill}%
\end{pgfscope}%
\begin{pgfscope}%
\pgfpathrectangle{\pgfqpoint{0.684066in}{0.665283in}}{\pgfqpoint{4.650000in}{3.080000in}}%
\pgfusepath{clip}%
\pgfsetbuttcap%
\pgfsetmiterjoin%
\definecolor{currentfill}{rgb}{0.865196,0.379902,0.315196}%
\pgfsetfillcolor{currentfill}%
\pgfsetlinewidth{1.003750pt}%
\definecolor{currentstroke}{rgb}{1.000000,1.000000,1.000000}%
\pgfsetstrokecolor{currentstroke}%
\pgfsetdash{}{0pt}%
\pgfpathmoveto{\pgfqpoint{1.195566in}{0.665283in}}%
\pgfpathlineto{\pgfqpoint{1.567566in}{0.665283in}}%
\pgfpathlineto{\pgfqpoint{1.567566in}{3.020514in}}%
\pgfpathlineto{\pgfqpoint{1.195566in}{3.020514in}}%
\pgfpathlineto{\pgfqpoint{1.195566in}{0.665283in}}%
\pgfpathclose%
\pgfusepath{stroke,fill}%
\end{pgfscope}%
\begin{pgfscope}%
\pgfpathrectangle{\pgfqpoint{0.684066in}{0.665283in}}{\pgfqpoint{4.650000in}{3.080000in}}%
\pgfusepath{clip}%
\pgfsetbuttcap%
\pgfsetmiterjoin%
\definecolor{currentfill}{rgb}{0.865196,0.379902,0.315196}%
\pgfsetfillcolor{currentfill}%
\pgfsetlinewidth{1.003750pt}%
\definecolor{currentstroke}{rgb}{1.000000,1.000000,1.000000}%
\pgfsetstrokecolor{currentstroke}%
\pgfsetdash{}{0pt}%
\pgfpathmoveto{\pgfqpoint{1.660566in}{0.665283in}}%
\pgfpathlineto{\pgfqpoint{2.032566in}{0.665283in}}%
\pgfpathlineto{\pgfqpoint{2.032566in}{2.294675in}}%
\pgfpathlineto{\pgfqpoint{1.660566in}{2.294675in}}%
\pgfpathlineto{\pgfqpoint{1.660566in}{0.665283in}}%
\pgfpathclose%
\pgfusepath{stroke,fill}%
\end{pgfscope}%
\begin{pgfscope}%
\pgfpathrectangle{\pgfqpoint{0.684066in}{0.665283in}}{\pgfqpoint{4.650000in}{3.080000in}}%
\pgfusepath{clip}%
\pgfsetbuttcap%
\pgfsetmiterjoin%
\definecolor{currentfill}{rgb}{0.865196,0.379902,0.315196}%
\pgfsetfillcolor{currentfill}%
\pgfsetlinewidth{1.003750pt}%
\definecolor{currentstroke}{rgb}{1.000000,1.000000,1.000000}%
\pgfsetstrokecolor{currentstroke}%
\pgfsetdash{}{0pt}%
\pgfpathmoveto{\pgfqpoint{2.125566in}{0.665283in}}%
\pgfpathlineto{\pgfqpoint{2.497566in}{0.665283in}}%
\pgfpathlineto{\pgfqpoint{2.497566in}{1.624504in}}%
\pgfpathlineto{\pgfqpoint{2.125566in}{1.624504in}}%
\pgfpathlineto{\pgfqpoint{2.125566in}{0.665283in}}%
\pgfpathclose%
\pgfusepath{stroke,fill}%
\end{pgfscope}%
\begin{pgfscope}%
\pgfpathrectangle{\pgfqpoint{0.684066in}{0.665283in}}{\pgfqpoint{4.650000in}{3.080000in}}%
\pgfusepath{clip}%
\pgfsetbuttcap%
\pgfsetmiterjoin%
\definecolor{currentfill}{rgb}{0.865196,0.379902,0.315196}%
\pgfsetfillcolor{currentfill}%
\pgfsetlinewidth{1.003750pt}%
\definecolor{currentstroke}{rgb}{1.000000,1.000000,1.000000}%
\pgfsetstrokecolor{currentstroke}%
\pgfsetdash{}{0pt}%
\pgfpathmoveto{\pgfqpoint{2.590566in}{0.665283in}}%
\pgfpathlineto{\pgfqpoint{2.962566in}{0.665283in}}%
\pgfpathlineto{\pgfqpoint{2.962566in}{1.448932in}}%
\pgfpathlineto{\pgfqpoint{2.590566in}{1.448932in}}%
\pgfpathlineto{\pgfqpoint{2.590566in}{0.665283in}}%
\pgfpathclose%
\pgfusepath{stroke,fill}%
\end{pgfscope}%
\begin{pgfscope}%
\pgfpathrectangle{\pgfqpoint{0.684066in}{0.665283in}}{\pgfqpoint{4.650000in}{3.080000in}}%
\pgfusepath{clip}%
\pgfsetbuttcap%
\pgfsetmiterjoin%
\definecolor{currentfill}{rgb}{0.865196,0.379902,0.315196}%
\pgfsetfillcolor{currentfill}%
\pgfsetlinewidth{1.003750pt}%
\definecolor{currentstroke}{rgb}{1.000000,1.000000,1.000000}%
\pgfsetstrokecolor{currentstroke}%
\pgfsetdash{}{0pt}%
\pgfpathmoveto{\pgfqpoint{3.055566in}{0.665283in}}%
\pgfpathlineto{\pgfqpoint{3.427566in}{0.665283in}}%
\pgfpathlineto{\pgfqpoint{3.427566in}{1.354723in}}%
\pgfpathlineto{\pgfqpoint{3.055566in}{1.354723in}}%
\pgfpathlineto{\pgfqpoint{3.055566in}{0.665283in}}%
\pgfpathclose%
\pgfusepath{stroke,fill}%
\end{pgfscope}%
\begin{pgfscope}%
\pgfpathrectangle{\pgfqpoint{0.684066in}{0.665283in}}{\pgfqpoint{4.650000in}{3.080000in}}%
\pgfusepath{clip}%
\pgfsetbuttcap%
\pgfsetmiterjoin%
\definecolor{currentfill}{rgb}{0.865196,0.379902,0.315196}%
\pgfsetfillcolor{currentfill}%
\pgfsetlinewidth{1.003750pt}%
\definecolor{currentstroke}{rgb}{1.000000,1.000000,1.000000}%
\pgfsetstrokecolor{currentstroke}%
\pgfsetdash{}{0pt}%
\pgfpathmoveto{\pgfqpoint{3.520566in}{0.665283in}}%
\pgfpathlineto{\pgfqpoint{3.892566in}{0.665283in}}%
\pgfpathlineto{\pgfqpoint{3.892566in}{1.348300in}}%
\pgfpathlineto{\pgfqpoint{3.520566in}{1.348300in}}%
\pgfpathlineto{\pgfqpoint{3.520566in}{0.665283in}}%
\pgfpathclose%
\pgfusepath{stroke,fill}%
\end{pgfscope}%
\begin{pgfscope}%
\pgfpathrectangle{\pgfqpoint{0.684066in}{0.665283in}}{\pgfqpoint{4.650000in}{3.080000in}}%
\pgfusepath{clip}%
\pgfsetbuttcap%
\pgfsetmiterjoin%
\definecolor{currentfill}{rgb}{0.865196,0.379902,0.315196}%
\pgfsetfillcolor{currentfill}%
\pgfsetlinewidth{1.003750pt}%
\definecolor{currentstroke}{rgb}{1.000000,1.000000,1.000000}%
\pgfsetstrokecolor{currentstroke}%
\pgfsetdash{}{0pt}%
\pgfpathmoveto{\pgfqpoint{3.985566in}{0.665283in}}%
\pgfpathlineto{\pgfqpoint{4.357566in}{0.665283in}}%
\pgfpathlineto{\pgfqpoint{4.357566in}{1.254091in}}%
\pgfpathlineto{\pgfqpoint{3.985566in}{1.254091in}}%
\pgfpathlineto{\pgfqpoint{3.985566in}{0.665283in}}%
\pgfpathclose%
\pgfusepath{stroke,fill}%
\end{pgfscope}%
\begin{pgfscope}%
\pgfpathrectangle{\pgfqpoint{0.684066in}{0.665283in}}{\pgfqpoint{4.650000in}{3.080000in}}%
\pgfusepath{clip}%
\pgfsetbuttcap%
\pgfsetmiterjoin%
\definecolor{currentfill}{rgb}{0.865196,0.379902,0.315196}%
\pgfsetfillcolor{currentfill}%
\pgfsetlinewidth{1.003750pt}%
\definecolor{currentstroke}{rgb}{1.000000,1.000000,1.000000}%
\pgfsetstrokecolor{currentstroke}%
\pgfsetdash{}{0pt}%
\pgfpathmoveto{\pgfqpoint{4.450566in}{0.665283in}}%
\pgfpathlineto{\pgfqpoint{4.822566in}{0.665283in}}%
\pgfpathlineto{\pgfqpoint{4.822566in}{1.142752in}}%
\pgfpathlineto{\pgfqpoint{4.450566in}{1.142752in}}%
\pgfpathlineto{\pgfqpoint{4.450566in}{0.665283in}}%
\pgfpathclose%
\pgfusepath{stroke,fill}%
\end{pgfscope}%
\begin{pgfscope}%
\pgfpathrectangle{\pgfqpoint{0.684066in}{0.665283in}}{\pgfqpoint{4.650000in}{3.080000in}}%
\pgfusepath{clip}%
\pgfsetbuttcap%
\pgfsetmiterjoin%
\definecolor{currentfill}{rgb}{0.865196,0.379902,0.315196}%
\pgfsetfillcolor{currentfill}%
\pgfsetlinewidth{1.003750pt}%
\definecolor{currentstroke}{rgb}{1.000000,1.000000,1.000000}%
\pgfsetstrokecolor{currentstroke}%
\pgfsetdash{}{0pt}%
\pgfpathmoveto{\pgfqpoint{4.915566in}{0.665283in}}%
\pgfpathlineto{\pgfqpoint{5.287566in}{0.665283in}}%
\pgfpathlineto{\pgfqpoint{5.287566in}{1.125623in}}%
\pgfpathlineto{\pgfqpoint{4.915566in}{1.125623in}}%
\pgfpathlineto{\pgfqpoint{4.915566in}{0.665283in}}%
\pgfpathclose%
\pgfusepath{stroke,fill}%
\end{pgfscope}%
\begin{pgfscope}%
\pgfpathrectangle{\pgfqpoint{0.684066in}{0.665283in}}{\pgfqpoint{4.650000in}{3.080000in}}%
\pgfusepath{clip}%
\pgfsetroundcap%
\pgfsetroundjoin%
\pgfsetlinewidth{2.710125pt}%
\definecolor{currentstroke}{rgb}{0.260000,0.260000,0.260000}%
\pgfsetstrokecolor{currentstroke}%
\pgfsetdash{}{0pt}%
\pgfusepath{stroke}%
\end{pgfscope}%
\begin{pgfscope}%
\pgfpathrectangle{\pgfqpoint{0.684066in}{0.665283in}}{\pgfqpoint{4.650000in}{3.080000in}}%
\pgfusepath{clip}%
\pgfsetroundcap%
\pgfsetroundjoin%
\pgfsetlinewidth{2.710125pt}%
\definecolor{currentstroke}{rgb}{0.260000,0.260000,0.260000}%
\pgfsetstrokecolor{currentstroke}%
\pgfsetdash{}{0pt}%
\pgfusepath{stroke}%
\end{pgfscope}%
\begin{pgfscope}%
\pgfpathrectangle{\pgfqpoint{0.684066in}{0.665283in}}{\pgfqpoint{4.650000in}{3.080000in}}%
\pgfusepath{clip}%
\pgfsetroundcap%
\pgfsetroundjoin%
\pgfsetlinewidth{2.710125pt}%
\definecolor{currentstroke}{rgb}{0.260000,0.260000,0.260000}%
\pgfsetstrokecolor{currentstroke}%
\pgfsetdash{}{0pt}%
\pgfusepath{stroke}%
\end{pgfscope}%
\begin{pgfscope}%
\pgfpathrectangle{\pgfqpoint{0.684066in}{0.665283in}}{\pgfqpoint{4.650000in}{3.080000in}}%
\pgfusepath{clip}%
\pgfsetroundcap%
\pgfsetroundjoin%
\pgfsetlinewidth{2.710125pt}%
\definecolor{currentstroke}{rgb}{0.260000,0.260000,0.260000}%
\pgfsetstrokecolor{currentstroke}%
\pgfsetdash{}{0pt}%
\pgfusepath{stroke}%
\end{pgfscope}%
\begin{pgfscope}%
\pgfpathrectangle{\pgfqpoint{0.684066in}{0.665283in}}{\pgfqpoint{4.650000in}{3.080000in}}%
\pgfusepath{clip}%
\pgfsetroundcap%
\pgfsetroundjoin%
\pgfsetlinewidth{2.710125pt}%
\definecolor{currentstroke}{rgb}{0.260000,0.260000,0.260000}%
\pgfsetstrokecolor{currentstroke}%
\pgfsetdash{}{0pt}%
\pgfusepath{stroke}%
\end{pgfscope}%
\begin{pgfscope}%
\pgfpathrectangle{\pgfqpoint{0.684066in}{0.665283in}}{\pgfqpoint{4.650000in}{3.080000in}}%
\pgfusepath{clip}%
\pgfsetroundcap%
\pgfsetroundjoin%
\pgfsetlinewidth{2.710125pt}%
\definecolor{currentstroke}{rgb}{0.260000,0.260000,0.260000}%
\pgfsetstrokecolor{currentstroke}%
\pgfsetdash{}{0pt}%
\pgfusepath{stroke}%
\end{pgfscope}%
\begin{pgfscope}%
\pgfpathrectangle{\pgfqpoint{0.684066in}{0.665283in}}{\pgfqpoint{4.650000in}{3.080000in}}%
\pgfusepath{clip}%
\pgfsetroundcap%
\pgfsetroundjoin%
\pgfsetlinewidth{2.710125pt}%
\definecolor{currentstroke}{rgb}{0.260000,0.260000,0.260000}%
\pgfsetstrokecolor{currentstroke}%
\pgfsetdash{}{0pt}%
\pgfusepath{stroke}%
\end{pgfscope}%
\begin{pgfscope}%
\pgfpathrectangle{\pgfqpoint{0.684066in}{0.665283in}}{\pgfqpoint{4.650000in}{3.080000in}}%
\pgfusepath{clip}%
\pgfsetroundcap%
\pgfsetroundjoin%
\pgfsetlinewidth{2.710125pt}%
\definecolor{currentstroke}{rgb}{0.260000,0.260000,0.260000}%
\pgfsetstrokecolor{currentstroke}%
\pgfsetdash{}{0pt}%
\pgfusepath{stroke}%
\end{pgfscope}%
\begin{pgfscope}%
\pgfpathrectangle{\pgfqpoint{0.684066in}{0.665283in}}{\pgfqpoint{4.650000in}{3.080000in}}%
\pgfusepath{clip}%
\pgfsetroundcap%
\pgfsetroundjoin%
\pgfsetlinewidth{2.710125pt}%
\definecolor{currentstroke}{rgb}{0.260000,0.260000,0.260000}%
\pgfsetstrokecolor{currentstroke}%
\pgfsetdash{}{0pt}%
\pgfusepath{stroke}%
\end{pgfscope}%
\begin{pgfscope}%
\pgfpathrectangle{\pgfqpoint{0.684066in}{0.665283in}}{\pgfqpoint{4.650000in}{3.080000in}}%
\pgfusepath{clip}%
\pgfsetroundcap%
\pgfsetroundjoin%
\pgfsetlinewidth{2.710125pt}%
\definecolor{currentstroke}{rgb}{0.260000,0.260000,0.260000}%
\pgfsetstrokecolor{currentstroke}%
\pgfsetdash{}{0pt}%
\pgfusepath{stroke}%
\end{pgfscope}%
\begin{pgfscope}%
\pgfsetrectcap%
\pgfsetmiterjoin%
\pgfsetlinewidth{1.254687pt}%
\definecolor{currentstroke}{rgb}{0.150000,0.150000,0.150000}%
\pgfsetstrokecolor{currentstroke}%
\pgfsetdash{}{0pt}%
\pgfpathmoveto{\pgfqpoint{0.684066in}{0.665283in}}%
\pgfpathlineto{\pgfqpoint{0.684066in}{3.745283in}}%
\pgfusepath{stroke}%
\end{pgfscope}%
\begin{pgfscope}%
\pgfsetrectcap%
\pgfsetmiterjoin%
\pgfsetlinewidth{1.254687pt}%
\definecolor{currentstroke}{rgb}{0.150000,0.150000,0.150000}%
\pgfsetstrokecolor{currentstroke}%
\pgfsetdash{}{0pt}%
\pgfpathmoveto{\pgfqpoint{5.334066in}{0.665283in}}%
\pgfpathlineto{\pgfqpoint{5.334066in}{3.745283in}}%
\pgfusepath{stroke}%
\end{pgfscope}%
\begin{pgfscope}%
\pgfsetrectcap%
\pgfsetmiterjoin%
\pgfsetlinewidth{1.254687pt}%
\definecolor{currentstroke}{rgb}{0.150000,0.150000,0.150000}%
\pgfsetstrokecolor{currentstroke}%
\pgfsetdash{}{0pt}%
\pgfpathmoveto{\pgfqpoint{0.684066in}{0.665283in}}%
\pgfpathlineto{\pgfqpoint{5.334066in}{0.665283in}}%
\pgfusepath{stroke}%
\end{pgfscope}%
\begin{pgfscope}%
\pgfsetrectcap%
\pgfsetmiterjoin%
\pgfsetlinewidth{1.254687pt}%
\definecolor{currentstroke}{rgb}{0.150000,0.150000,0.150000}%
\pgfsetstrokecolor{currentstroke}%
\pgfsetdash{}{0pt}%
\pgfpathmoveto{\pgfqpoint{0.684066in}{3.745283in}}%
\pgfpathlineto{\pgfqpoint{5.334066in}{3.745283in}}%
\pgfusepath{stroke}%
\end{pgfscope}%
\end{pgfpicture}%
\makeatother%
\endgroup%
}
    \caption{Principaux pays producteurs de blé (2021)}
\end{figure}
Ici une période s'étalant de 2003 à 2022 est choisie afin de dresser des statistiques descriptives sur le cours du contrat a terme sur le blé meunier. 
\begin{table}[H]
    \centering
    \caption{Statistiques descriptives sur le cours du blé de 2003 à 2022}
    \sffamily
    \begin{tabular}{ccccccc}
    \toprule
    Moyenne & Écart-Type & Minimum & Maximum & Médiane & Q1 & Q3 \\
    \midrule
    185,26 \euro & 57,65 \euro & 101,50 \euro & 400,75 \euro & 180,37 \euro & 148,88 \euro & 209,88 \euro \\
    \bottomrule
\end{tabular}
\end{table}
Le prix moyen d'un contrat a terme sur le blé sur les 20 ans est de 185,26 \euro\ pour une écart-type de 57 \euro. Sur la période, le prix minimum 101,50 \euro\ 
date du mois d'avril 2005. En contrepartie, le prix maximum de 400,75 \euro\ a été atteint en avril 2022, l'étendue entre 2003 et 2022 est donc de 299,25 \euro. 
\begin{table}[H]
    \centering
    \caption{Statistiques sur les rendement mensuels du cours du blé de 2003 à 2022}
    \sffamily
    \begin{tabular}{ccccccc}
    \toprule
    Moyenne & Écart-Type & Minimum & Maximum & Skewness & Kurtosis \\
    \midrule
    0,43 \% & 7,41\% & -24,35\%  & 30,94\% & 0,09 & 2,10 \\
    \bottomrule
\end{tabular}
\end{table}
Il est aussi intéressant d'analyser les rendements (logarithmiques) du cours. En effet, la moyenne des rendements est quasiment égale à 0 sur les 20 ans, les rendements
positifs et négatifs se compensent donc entre eux. Avec 7\% de volatilité mensuel, le cours est assez peu volatile. De plus le Skewness étant proche de 0 et le kurtosis 
proche de 3, la distribution des rendements semble normale, cela semble être confirmé graphiquement par l'histogramme de répartition des rendements.
\begin{figure}[H]
    \centering
    \label{fig:ble_rendement}
    \resizebox{\textwidth}{!}{%% Creator: Matplotlib, PGF backend
%%
%% To include the figure in your LaTeX document, write
%%   \input{<filename>.pgf}
%%
%% Make sure the required packages are loaded in your preamble
%%   \usepackage{pgf}
%%
%% Also ensure that all the required font packages are loaded; for instance,
%% the lmodern package is sometimes necessary when using math font.
%%   \usepackage{lmodern}
%%
%% Figures using additional raster images can only be included by \input if
%% they are in the same directory as the main LaTeX file. For loading figures
%% from other directories you can use the `import` package
%%   \usepackage{import}
%%
%% and then include the figures with
%%   \import{<path to file>}{<filename>.pgf}
%%
%% Matplotlib used the following preamble
%%   \usepackage{fontspec}
%%   \setmainfont{DejaVuSerif.ttf}[Path=\detokenize{C:/Users/Joseph/miniconda3/Lib/site-packages/matplotlib/mpl-data/fonts/ttf/}]
%%   \setsansfont{arial.ttf}[Path=\detokenize{C:/Windows/Fonts/}]
%%   \setmonofont{DejaVuSansMono.ttf}[Path=\detokenize{C:/Users/Joseph/miniconda3/Lib/site-packages/matplotlib/mpl-data/fonts/ttf/}]
%%
\begingroup%
\makeatletter%
\begin{pgfpicture}%
\pgfpathrectangle{\pgfpointorigin}{\pgfqpoint{12.820050in}{5.502881in}}%
\pgfusepath{use as bounding box, clip}%
\begin{pgfscope}%
\pgfsetbuttcap%
\pgfsetmiterjoin%
\definecolor{currentfill}{rgb}{1.000000,1.000000,1.000000}%
\pgfsetfillcolor{currentfill}%
\pgfsetlinewidth{0.000000pt}%
\definecolor{currentstroke}{rgb}{1.000000,1.000000,1.000000}%
\pgfsetstrokecolor{currentstroke}%
\pgfsetdash{}{0pt}%
\pgfpathmoveto{\pgfqpoint{0.000000in}{0.000000in}}%
\pgfpathlineto{\pgfqpoint{12.820050in}{0.000000in}}%
\pgfpathlineto{\pgfqpoint{12.820050in}{5.502881in}}%
\pgfpathlineto{\pgfqpoint{0.000000in}{5.502881in}}%
\pgfpathlineto{\pgfqpoint{0.000000in}{0.000000in}}%
\pgfpathclose%
\pgfusepath{fill}%
\end{pgfscope}%
\begin{pgfscope}%
\pgfsetbuttcap%
\pgfsetmiterjoin%
\definecolor{currentfill}{rgb}{1.000000,1.000000,1.000000}%
\pgfsetfillcolor{currentfill}%
\pgfsetlinewidth{0.000000pt}%
\definecolor{currentstroke}{rgb}{0.000000,0.000000,0.000000}%
\pgfsetstrokecolor{currentstroke}%
\pgfsetstrokeopacity{0.000000}%
\pgfsetdash{}{0pt}%
\pgfpathmoveto{\pgfqpoint{1.095050in}{0.782881in}}%
\pgfpathlineto{\pgfqpoint{9.601148in}{0.782881in}}%
\pgfpathlineto{\pgfqpoint{9.601148in}{5.402881in}}%
\pgfpathlineto{\pgfqpoint{1.095050in}{5.402881in}}%
\pgfpathlineto{\pgfqpoint{1.095050in}{0.782881in}}%
\pgfpathclose%
\pgfusepath{fill}%
\end{pgfscope}%
\begin{pgfscope}%
\pgfpathrectangle{\pgfqpoint{1.095050in}{0.782881in}}{\pgfqpoint{8.506098in}{4.620000in}}%
\pgfusepath{clip}%
\pgfsetroundcap%
\pgfsetroundjoin%
\pgfsetlinewidth{1.003750pt}%
\definecolor{currentstroke}{rgb}{0.800000,0.800000,0.800000}%
\pgfsetstrokecolor{currentstroke}%
\pgfsetstrokeopacity{0.000000}%
\pgfsetdash{}{0pt}%
\pgfpathmoveto{\pgfqpoint{1.521875in}{0.782881in}}%
\pgfpathlineto{\pgfqpoint{1.521875in}{5.402881in}}%
\pgfusepath{stroke}%
\end{pgfscope}%
\begin{pgfscope}%
\definecolor{textcolor}{rgb}{0.150000,0.150000,0.150000}%
\pgfsetstrokecolor{textcolor}%
\pgfsetfillcolor{textcolor}%
\pgftext[x=1.521875in,y=0.650937in,,top]{\color{textcolor}\sffamily\fontsize{19.000000}{22.800000}\selectfont 2004}%
\end{pgfscope}%
\begin{pgfscope}%
\pgfpathrectangle{\pgfqpoint{1.095050in}{0.782881in}}{\pgfqpoint{8.506098in}{4.620000in}}%
\pgfusepath{clip}%
\pgfsetroundcap%
\pgfsetroundjoin%
\pgfsetlinewidth{1.003750pt}%
\definecolor{currentstroke}{rgb}{0.800000,0.800000,0.800000}%
\pgfsetstrokecolor{currentstroke}%
\pgfsetstrokeopacity{0.000000}%
\pgfsetdash{}{0pt}%
\pgfpathmoveto{\pgfqpoint{2.376695in}{0.782881in}}%
\pgfpathlineto{\pgfqpoint{2.376695in}{5.402881in}}%
\pgfusepath{stroke}%
\end{pgfscope}%
\begin{pgfscope}%
\definecolor{textcolor}{rgb}{0.150000,0.150000,0.150000}%
\pgfsetstrokecolor{textcolor}%
\pgfsetfillcolor{textcolor}%
\pgftext[x=2.376695in,y=0.650937in,,top]{\color{textcolor}\sffamily\fontsize{19.000000}{22.800000}\selectfont 2006}%
\end{pgfscope}%
\begin{pgfscope}%
\pgfpathrectangle{\pgfqpoint{1.095050in}{0.782881in}}{\pgfqpoint{8.506098in}{4.620000in}}%
\pgfusepath{clip}%
\pgfsetroundcap%
\pgfsetroundjoin%
\pgfsetlinewidth{1.003750pt}%
\definecolor{currentstroke}{rgb}{0.800000,0.800000,0.800000}%
\pgfsetstrokecolor{currentstroke}%
\pgfsetstrokeopacity{0.000000}%
\pgfsetdash{}{0pt}%
\pgfpathmoveto{\pgfqpoint{3.230345in}{0.782881in}}%
\pgfpathlineto{\pgfqpoint{3.230345in}{5.402881in}}%
\pgfusepath{stroke}%
\end{pgfscope}%
\begin{pgfscope}%
\definecolor{textcolor}{rgb}{0.150000,0.150000,0.150000}%
\pgfsetstrokecolor{textcolor}%
\pgfsetfillcolor{textcolor}%
\pgftext[x=3.230345in,y=0.650937in,,top]{\color{textcolor}\sffamily\fontsize{19.000000}{22.800000}\selectfont 2008}%
\end{pgfscope}%
\begin{pgfscope}%
\pgfpathrectangle{\pgfqpoint{1.095050in}{0.782881in}}{\pgfqpoint{8.506098in}{4.620000in}}%
\pgfusepath{clip}%
\pgfsetroundcap%
\pgfsetroundjoin%
\pgfsetlinewidth{1.003750pt}%
\definecolor{currentstroke}{rgb}{0.800000,0.800000,0.800000}%
\pgfsetstrokecolor{currentstroke}%
\pgfsetstrokeopacity{0.000000}%
\pgfsetdash{}{0pt}%
\pgfpathmoveto{\pgfqpoint{4.085165in}{0.782881in}}%
\pgfpathlineto{\pgfqpoint{4.085165in}{5.402881in}}%
\pgfusepath{stroke}%
\end{pgfscope}%
\begin{pgfscope}%
\definecolor{textcolor}{rgb}{0.150000,0.150000,0.150000}%
\pgfsetstrokecolor{textcolor}%
\pgfsetfillcolor{textcolor}%
\pgftext[x=4.085165in,y=0.650937in,,top]{\color{textcolor}\sffamily\fontsize{19.000000}{22.800000}\selectfont 2010}%
\end{pgfscope}%
\begin{pgfscope}%
\pgfpathrectangle{\pgfqpoint{1.095050in}{0.782881in}}{\pgfqpoint{8.506098in}{4.620000in}}%
\pgfusepath{clip}%
\pgfsetroundcap%
\pgfsetroundjoin%
\pgfsetlinewidth{1.003750pt}%
\definecolor{currentstroke}{rgb}{0.800000,0.800000,0.800000}%
\pgfsetstrokecolor{currentstroke}%
\pgfsetstrokeopacity{0.000000}%
\pgfsetdash{}{0pt}%
\pgfpathmoveto{\pgfqpoint{4.938815in}{0.782881in}}%
\pgfpathlineto{\pgfqpoint{4.938815in}{5.402881in}}%
\pgfusepath{stroke}%
\end{pgfscope}%
\begin{pgfscope}%
\definecolor{textcolor}{rgb}{0.150000,0.150000,0.150000}%
\pgfsetstrokecolor{textcolor}%
\pgfsetfillcolor{textcolor}%
\pgftext[x=4.938815in,y=0.650937in,,top]{\color{textcolor}\sffamily\fontsize{19.000000}{22.800000}\selectfont 2012}%
\end{pgfscope}%
\begin{pgfscope}%
\pgfpathrectangle{\pgfqpoint{1.095050in}{0.782881in}}{\pgfqpoint{8.506098in}{4.620000in}}%
\pgfusepath{clip}%
\pgfsetroundcap%
\pgfsetroundjoin%
\pgfsetlinewidth{1.003750pt}%
\definecolor{currentstroke}{rgb}{0.800000,0.800000,0.800000}%
\pgfsetstrokecolor{currentstroke}%
\pgfsetstrokeopacity{0.000000}%
\pgfsetdash{}{0pt}%
\pgfpathmoveto{\pgfqpoint{5.793634in}{0.782881in}}%
\pgfpathlineto{\pgfqpoint{5.793634in}{5.402881in}}%
\pgfusepath{stroke}%
\end{pgfscope}%
\begin{pgfscope}%
\definecolor{textcolor}{rgb}{0.150000,0.150000,0.150000}%
\pgfsetstrokecolor{textcolor}%
\pgfsetfillcolor{textcolor}%
\pgftext[x=5.793634in,y=0.650937in,,top]{\color{textcolor}\sffamily\fontsize{19.000000}{22.800000}\selectfont 2014}%
\end{pgfscope}%
\begin{pgfscope}%
\pgfpathrectangle{\pgfqpoint{1.095050in}{0.782881in}}{\pgfqpoint{8.506098in}{4.620000in}}%
\pgfusepath{clip}%
\pgfsetroundcap%
\pgfsetroundjoin%
\pgfsetlinewidth{1.003750pt}%
\definecolor{currentstroke}{rgb}{0.800000,0.800000,0.800000}%
\pgfsetstrokecolor{currentstroke}%
\pgfsetstrokeopacity{0.000000}%
\pgfsetdash{}{0pt}%
\pgfpathmoveto{\pgfqpoint{6.647285in}{0.782881in}}%
\pgfpathlineto{\pgfqpoint{6.647285in}{5.402881in}}%
\pgfusepath{stroke}%
\end{pgfscope}%
\begin{pgfscope}%
\definecolor{textcolor}{rgb}{0.150000,0.150000,0.150000}%
\pgfsetstrokecolor{textcolor}%
\pgfsetfillcolor{textcolor}%
\pgftext[x=6.647285in,y=0.650937in,,top]{\color{textcolor}\sffamily\fontsize{19.000000}{22.800000}\selectfont 2016}%
\end{pgfscope}%
\begin{pgfscope}%
\pgfpathrectangle{\pgfqpoint{1.095050in}{0.782881in}}{\pgfqpoint{8.506098in}{4.620000in}}%
\pgfusepath{clip}%
\pgfsetroundcap%
\pgfsetroundjoin%
\pgfsetlinewidth{1.003750pt}%
\definecolor{currentstroke}{rgb}{0.800000,0.800000,0.800000}%
\pgfsetstrokecolor{currentstroke}%
\pgfsetstrokeopacity{0.000000}%
\pgfsetdash{}{0pt}%
\pgfpathmoveto{\pgfqpoint{7.502104in}{0.782881in}}%
\pgfpathlineto{\pgfqpoint{7.502104in}{5.402881in}}%
\pgfusepath{stroke}%
\end{pgfscope}%
\begin{pgfscope}%
\definecolor{textcolor}{rgb}{0.150000,0.150000,0.150000}%
\pgfsetstrokecolor{textcolor}%
\pgfsetfillcolor{textcolor}%
\pgftext[x=7.502104in,y=0.650937in,,top]{\color{textcolor}\sffamily\fontsize{19.000000}{22.800000}\selectfont 2018}%
\end{pgfscope}%
\begin{pgfscope}%
\pgfpathrectangle{\pgfqpoint{1.095050in}{0.782881in}}{\pgfqpoint{8.506098in}{4.620000in}}%
\pgfusepath{clip}%
\pgfsetroundcap%
\pgfsetroundjoin%
\pgfsetlinewidth{1.003750pt}%
\definecolor{currentstroke}{rgb}{0.800000,0.800000,0.800000}%
\pgfsetstrokecolor{currentstroke}%
\pgfsetstrokeopacity{0.000000}%
\pgfsetdash{}{0pt}%
\pgfpathmoveto{\pgfqpoint{8.355754in}{0.782881in}}%
\pgfpathlineto{\pgfqpoint{8.355754in}{5.402881in}}%
\pgfusepath{stroke}%
\end{pgfscope}%
\begin{pgfscope}%
\definecolor{textcolor}{rgb}{0.150000,0.150000,0.150000}%
\pgfsetstrokecolor{textcolor}%
\pgfsetfillcolor{textcolor}%
\pgftext[x=8.355754in,y=0.650937in,,top]{\color{textcolor}\sffamily\fontsize{19.000000}{22.800000}\selectfont 2020}%
\end{pgfscope}%
\begin{pgfscope}%
\pgfpathrectangle{\pgfqpoint{1.095050in}{0.782881in}}{\pgfqpoint{8.506098in}{4.620000in}}%
\pgfusepath{clip}%
\pgfsetroundcap%
\pgfsetroundjoin%
\pgfsetlinewidth{1.003750pt}%
\definecolor{currentstroke}{rgb}{0.800000,0.800000,0.800000}%
\pgfsetstrokecolor{currentstroke}%
\pgfsetstrokeopacity{0.000000}%
\pgfsetdash{}{0pt}%
\pgfpathmoveto{\pgfqpoint{9.210574in}{0.782881in}}%
\pgfpathlineto{\pgfqpoint{9.210574in}{5.402881in}}%
\pgfusepath{stroke}%
\end{pgfscope}%
\begin{pgfscope}%
\definecolor{textcolor}{rgb}{0.150000,0.150000,0.150000}%
\pgfsetstrokecolor{textcolor}%
\pgfsetfillcolor{textcolor}%
\pgftext[x=9.210574in,y=0.650937in,,top]{\color{textcolor}\sffamily\fontsize{19.000000}{22.800000}\selectfont 2022}%
\end{pgfscope}%
\begin{pgfscope}%
\definecolor{textcolor}{rgb}{0.150000,0.150000,0.150000}%
\pgfsetstrokecolor{textcolor}%
\pgfsetfillcolor{textcolor}%
\pgftext[x=5.348099in,y=0.354042in,,top]{\color{textcolor}\sffamily\fontsize{20.000000}{24.000000}\selectfont Date}%
\end{pgfscope}%
\begin{pgfscope}%
\pgfpathrectangle{\pgfqpoint{1.095050in}{0.782881in}}{\pgfqpoint{8.506098in}{4.620000in}}%
\pgfusepath{clip}%
\pgfsetroundcap%
\pgfsetroundjoin%
\pgfsetlinewidth{1.003750pt}%
\definecolor{currentstroke}{rgb}{0.800000,0.800000,0.800000}%
\pgfsetstrokecolor{currentstroke}%
\pgfsetstrokeopacity{0.400000}%
\pgfsetdash{}{0pt}%
\pgfpathmoveto{\pgfqpoint{1.095050in}{1.323139in}}%
\pgfpathlineto{\pgfqpoint{9.601148in}{1.323139in}}%
\pgfusepath{stroke}%
\end{pgfscope}%
\begin{pgfscope}%
\definecolor{textcolor}{rgb}{0.150000,0.150000,0.150000}%
\pgfsetstrokecolor{textcolor}%
\pgfsetfillcolor{textcolor}%
\pgftext[x=0.409597in, y=1.228691in, left, base]{\color{textcolor}\sffamily\fontsize{19.000000}{22.800000}\selectfont \ensuremath{-}0.2}%
\end{pgfscope}%
\begin{pgfscope}%
\pgfpathrectangle{\pgfqpoint{1.095050in}{0.782881in}}{\pgfqpoint{8.506098in}{4.620000in}}%
\pgfusepath{clip}%
\pgfsetroundcap%
\pgfsetroundjoin%
\pgfsetlinewidth{1.003750pt}%
\definecolor{currentstroke}{rgb}{0.800000,0.800000,0.800000}%
\pgfsetstrokecolor{currentstroke}%
\pgfsetstrokeopacity{0.400000}%
\pgfsetdash{}{0pt}%
\pgfpathmoveto{\pgfqpoint{1.095050in}{2.082773in}}%
\pgfpathlineto{\pgfqpoint{9.601148in}{2.082773in}}%
\pgfusepath{stroke}%
\end{pgfscope}%
\begin{pgfscope}%
\definecolor{textcolor}{rgb}{0.150000,0.150000,0.150000}%
\pgfsetstrokecolor{textcolor}%
\pgfsetfillcolor{textcolor}%
\pgftext[x=0.409597in, y=1.988325in, left, base]{\color{textcolor}\sffamily\fontsize{19.000000}{22.800000}\selectfont \ensuremath{-}0.1}%
\end{pgfscope}%
\begin{pgfscope}%
\pgfpathrectangle{\pgfqpoint{1.095050in}{0.782881in}}{\pgfqpoint{8.506098in}{4.620000in}}%
\pgfusepath{clip}%
\pgfsetroundcap%
\pgfsetroundjoin%
\pgfsetlinewidth{1.003750pt}%
\definecolor{currentstroke}{rgb}{0.800000,0.800000,0.800000}%
\pgfsetstrokecolor{currentstroke}%
\pgfsetstrokeopacity{0.400000}%
\pgfsetdash{}{0pt}%
\pgfpathmoveto{\pgfqpoint{1.095050in}{2.842407in}}%
\pgfpathlineto{\pgfqpoint{9.601148in}{2.842407in}}%
\pgfusepath{stroke}%
\end{pgfscope}%
\begin{pgfscope}%
\definecolor{textcolor}{rgb}{0.150000,0.150000,0.150000}%
\pgfsetstrokecolor{textcolor}%
\pgfsetfillcolor{textcolor}%
\pgftext[x=0.596264in, y=2.747958in, left, base]{\color{textcolor}\sffamily\fontsize{19.000000}{22.800000}\selectfont 0.0}%
\end{pgfscope}%
\begin{pgfscope}%
\pgfpathrectangle{\pgfqpoint{1.095050in}{0.782881in}}{\pgfqpoint{8.506098in}{4.620000in}}%
\pgfusepath{clip}%
\pgfsetroundcap%
\pgfsetroundjoin%
\pgfsetlinewidth{1.003750pt}%
\definecolor{currentstroke}{rgb}{0.800000,0.800000,0.800000}%
\pgfsetstrokecolor{currentstroke}%
\pgfsetstrokeopacity{0.400000}%
\pgfsetdash{}{0pt}%
\pgfpathmoveto{\pgfqpoint{1.095050in}{3.602041in}}%
\pgfpathlineto{\pgfqpoint{9.601148in}{3.602041in}}%
\pgfusepath{stroke}%
\end{pgfscope}%
\begin{pgfscope}%
\definecolor{textcolor}{rgb}{0.150000,0.150000,0.150000}%
\pgfsetstrokecolor{textcolor}%
\pgfsetfillcolor{textcolor}%
\pgftext[x=0.596264in, y=3.507592in, left, base]{\color{textcolor}\sffamily\fontsize{19.000000}{22.800000}\selectfont 0.1}%
\end{pgfscope}%
\begin{pgfscope}%
\pgfpathrectangle{\pgfqpoint{1.095050in}{0.782881in}}{\pgfqpoint{8.506098in}{4.620000in}}%
\pgfusepath{clip}%
\pgfsetroundcap%
\pgfsetroundjoin%
\pgfsetlinewidth{1.003750pt}%
\definecolor{currentstroke}{rgb}{0.800000,0.800000,0.800000}%
\pgfsetstrokecolor{currentstroke}%
\pgfsetstrokeopacity{0.400000}%
\pgfsetdash{}{0pt}%
\pgfpathmoveto{\pgfqpoint{1.095050in}{4.361674in}}%
\pgfpathlineto{\pgfqpoint{9.601148in}{4.361674in}}%
\pgfusepath{stroke}%
\end{pgfscope}%
\begin{pgfscope}%
\definecolor{textcolor}{rgb}{0.150000,0.150000,0.150000}%
\pgfsetstrokecolor{textcolor}%
\pgfsetfillcolor{textcolor}%
\pgftext[x=0.596264in, y=4.267226in, left, base]{\color{textcolor}\sffamily\fontsize{19.000000}{22.800000}\selectfont 0.2}%
\end{pgfscope}%
\begin{pgfscope}%
\pgfpathrectangle{\pgfqpoint{1.095050in}{0.782881in}}{\pgfqpoint{8.506098in}{4.620000in}}%
\pgfusepath{clip}%
\pgfsetroundcap%
\pgfsetroundjoin%
\pgfsetlinewidth{1.003750pt}%
\definecolor{currentstroke}{rgb}{0.800000,0.800000,0.800000}%
\pgfsetstrokecolor{currentstroke}%
\pgfsetstrokeopacity{0.400000}%
\pgfsetdash{}{0pt}%
\pgfpathmoveto{\pgfqpoint{1.095050in}{5.121308in}}%
\pgfpathlineto{\pgfqpoint{9.601148in}{5.121308in}}%
\pgfusepath{stroke}%
\end{pgfscope}%
\begin{pgfscope}%
\definecolor{textcolor}{rgb}{0.150000,0.150000,0.150000}%
\pgfsetstrokecolor{textcolor}%
\pgfsetfillcolor{textcolor}%
\pgftext[x=0.596264in, y=5.026860in, left, base]{\color{textcolor}\sffamily\fontsize{19.000000}{22.800000}\selectfont 0.3}%
\end{pgfscope}%
\begin{pgfscope}%
\definecolor{textcolor}{rgb}{0.150000,0.150000,0.150000}%
\pgfsetstrokecolor{textcolor}%
\pgfsetfillcolor{textcolor}%
\pgftext[x=0.354042in,y=3.092881in,,bottom,rotate=90.000000]{\color{textcolor}\sffamily\fontsize{20.000000}{24.000000}\selectfont Rendements mensuels}%
\end{pgfscope}%
\begin{pgfscope}%
\pgfpathrectangle{\pgfqpoint{1.095050in}{0.782881in}}{\pgfqpoint{8.506098in}{4.620000in}}%
\pgfusepath{clip}%
\pgfsetroundcap%
\pgfsetroundjoin%
\pgfsetlinewidth{1.505625pt}%
\definecolor{currentstroke}{rgb}{0.983576,0.412795,0.288351}%
\pgfsetstrokecolor{currentstroke}%
\pgfsetdash{}{0pt}%
\pgfpathmoveto{\pgfqpoint{1.131301in}{2.651403in}}%
\pgfpathlineto{\pgfqpoint{1.164044in}{3.033411in}}%
\pgfpathlineto{\pgfqpoint{1.200295in}{2.995188in}}%
\pgfpathlineto{\pgfqpoint{1.235376in}{2.638009in}}%
\pgfpathlineto{\pgfqpoint{1.271627in}{3.113734in}}%
\pgfpathlineto{\pgfqpoint{1.306709in}{3.295089in}}%
\pgfpathlineto{\pgfqpoint{1.342960in}{3.560706in}}%
\pgfpathlineto{\pgfqpoint{1.379211in}{3.067500in}}%
\pgfpathlineto{\pgfqpoint{1.414292in}{3.480236in}}%
\pgfpathlineto{\pgfqpoint{1.450543in}{3.477841in}}%
\pgfpathlineto{\pgfqpoint{1.485625in}{2.432996in}}%
\pgfpathlineto{\pgfqpoint{1.521875in}{3.013668in}}%
\pgfpathlineto{\pgfqpoint{1.558126in}{2.646362in}}%
\pgfpathlineto{\pgfqpoint{1.592039in}{2.965527in}}%
\pgfpathlineto{\pgfqpoint{1.628289in}{2.817942in}}%
\pgfpathlineto{\pgfqpoint{1.663371in}{1.208348in}}%
\pgfpathlineto{\pgfqpoint{1.699622in}{2.595350in}}%
\pgfpathlineto{\pgfqpoint{1.734703in}{2.135643in}}%
\pgfpathlineto{\pgfqpoint{1.770954in}{2.945062in}}%
\pgfpathlineto{\pgfqpoint{1.807205in}{2.618208in}}%
\pgfpathlineto{\pgfqpoint{1.842287in}{2.842407in}}%
\pgfpathlineto{\pgfqpoint{1.878538in}{2.772070in}}%
\pgfpathlineto{\pgfqpoint{1.913619in}{2.735665in}}%
\pgfpathlineto{\pgfqpoint{1.949870in}{2.931462in}}%
\pgfpathlineto{\pgfqpoint{1.986121in}{2.965356in}}%
\pgfpathlineto{\pgfqpoint{2.018864in}{2.503944in}}%
\pgfpathlineto{\pgfqpoint{2.055114in}{2.639334in}}%
\pgfpathlineto{\pgfqpoint{2.090196in}{3.171938in}}%
\pgfpathlineto{\pgfqpoint{2.126447in}{2.679428in}}%
\pgfpathlineto{\pgfqpoint{2.161528in}{2.951447in}}%
\pgfpathlineto{\pgfqpoint{2.197779in}{2.878409in}}%
\pgfpathlineto{\pgfqpoint{2.234030in}{3.019900in}}%
\pgfpathlineto{\pgfqpoint{2.269112in}{2.912259in}}%
\pgfpathlineto{\pgfqpoint{2.305363in}{2.894378in}}%
\pgfpathlineto{\pgfqpoint{2.340444in}{2.773033in}}%
\pgfpathlineto{\pgfqpoint{2.376695in}{3.031680in}}%
\pgfpathlineto{\pgfqpoint{2.412946in}{2.825394in}}%
\pgfpathlineto{\pgfqpoint{2.445689in}{2.739520in}}%
\pgfpathlineto{\pgfqpoint{2.481940in}{3.113734in}}%
\pgfpathlineto{\pgfqpoint{2.517021in}{2.875651in}}%
\pgfpathlineto{\pgfqpoint{2.553272in}{2.674711in}}%
\pgfpathlineto{\pgfqpoint{2.588353in}{3.398086in}}%
\pgfpathlineto{\pgfqpoint{2.624604in}{3.954623in}}%
\pgfpathlineto{\pgfqpoint{2.660855in}{3.266006in}}%
\pgfpathlineto{\pgfqpoint{2.695937in}{3.268013in}}%
\pgfpathlineto{\pgfqpoint{2.732188in}{2.582360in}}%
\pgfpathlineto{\pgfqpoint{2.767269in}{2.854994in}}%
\pgfpathlineto{\pgfqpoint{2.803520in}{2.689967in}}%
\pgfpathlineto{\pgfqpoint{2.839771in}{3.069935in}}%
\pgfpathlineto{\pgfqpoint{2.872514in}{2.640499in}}%
\pgfpathlineto{\pgfqpoint{2.908765in}{3.468125in}}%
\pgfpathlineto{\pgfqpoint{2.943846in}{2.639508in}}%
\pgfpathlineto{\pgfqpoint{2.980097in}{4.068484in}}%
\pgfpathlineto{\pgfqpoint{3.015179in}{3.474155in}}%
\pgfpathlineto{\pgfqpoint{3.051429in}{4.706087in}}%
\pgfpathlineto{\pgfqpoint{3.122762in}{1.739555in}}%
\pgfpathlineto{\pgfqpoint{3.159013in}{3.410032in}}%
\pgfpathlineto{\pgfqpoint{3.194094in}{2.872732in}}%
\pgfpathlineto{\pgfqpoint{3.230345in}{2.643097in}}%
\pgfpathlineto{\pgfqpoint{3.266596in}{3.980026in}}%
\pgfpathlineto{\pgfqpoint{3.300508in}{1.355104in}}%
\pgfpathlineto{\pgfqpoint{3.336759in}{2.090696in}}%
\pgfpathlineto{\pgfqpoint{3.371841in}{1.701298in}}%
\pgfpathlineto{\pgfqpoint{3.408092in}{3.463270in}}%
\pgfpathlineto{\pgfqpoint{3.443173in}{2.637747in}}%
\pgfpathlineto{\pgfqpoint{3.479424in}{2.591348in}}%
\pgfpathlineto{\pgfqpoint{3.515675in}{1.757720in}}%
\pgfpathlineto{\pgfqpoint{3.550756in}{1.982937in}}%
\pgfpathlineto{\pgfqpoint{3.587007in}{2.309831in}}%
\pgfpathlineto{\pgfqpoint{3.622089in}{3.010289in}}%
\pgfpathlineto{\pgfqpoint{3.658340in}{3.567673in}}%
\pgfpathlineto{\pgfqpoint{3.694591in}{2.158538in}}%
\pgfpathlineto{\pgfqpoint{3.727333in}{2.504736in}}%
\pgfpathlineto{\pgfqpoint{3.763584in}{3.397130in}}%
\pgfpathlineto{\pgfqpoint{3.798666in}{3.334335in}}%
\pgfpathlineto{\pgfqpoint{3.834917in}{1.909988in}}%
\pgfpathlineto{\pgfqpoint{3.869998in}{2.464691in}}%
\pgfpathlineto{\pgfqpoint{3.906249in}{2.812559in}}%
\pgfpathlineto{\pgfqpoint{3.942500in}{2.568361in}}%
\pgfpathlineto{\pgfqpoint{3.977581in}{3.101485in}}%
\pgfpathlineto{\pgfqpoint{4.013832in}{3.121878in}}%
\pgfpathlineto{\pgfqpoint{4.048914in}{2.827951in}}%
\pgfpathlineto{\pgfqpoint{4.085165in}{2.502105in}}%
\pgfpathlineto{\pgfqpoint{4.121416in}{2.643097in}}%
\pgfpathlineto{\pgfqpoint{4.154158in}{3.026570in}}%
\pgfpathlineto{\pgfqpoint{4.190409in}{3.241140in}}%
\pgfpathlineto{\pgfqpoint{4.225491in}{2.828006in}}%
\pgfpathlineto{\pgfqpoint{4.261742in}{3.111457in}}%
\pgfpathlineto{\pgfqpoint{4.296823in}{5.192881in}}%
\pgfpathlineto{\pgfqpoint{4.333074in}{4.380679in}}%
\pgfpathlineto{\pgfqpoint{4.369325in}{2.153340in}}%
\pgfpathlineto{\pgfqpoint{4.404406in}{3.439193in}}%
\pgfpathlineto{\pgfqpoint{4.440657in}{2.808570in}}%
\pgfpathlineto{\pgfqpoint{4.475739in}{3.752184in}}%
\pgfpathlineto{\pgfqpoint{4.511990in}{3.323256in}}%
\pgfpathlineto{\pgfqpoint{4.548241in}{2.510510in}}%
\pgfpathlineto{\pgfqpoint{4.580983in}{2.307771in}}%
\pgfpathlineto{\pgfqpoint{4.617234in}{2.983519in}}%
\pgfpathlineto{\pgfqpoint{4.652316in}{2.573622in}}%
\pgfpathlineto{\pgfqpoint{4.688567in}{0.992881in}}%
\pgfpathlineto{\pgfqpoint{4.723648in}{3.387003in}}%
\pgfpathlineto{\pgfqpoint{4.759899in}{3.305742in}}%
\pgfpathlineto{\pgfqpoint{4.796150in}{1.772633in}}%
\pgfpathlineto{\pgfqpoint{4.831232in}{2.985932in}}%
\pgfpathlineto{\pgfqpoint{4.867482in}{2.489025in}}%
\pgfpathlineto{\pgfqpoint{4.902564in}{3.800695in}}%
\pgfpathlineto{\pgfqpoint{4.938815in}{3.315059in}}%
\pgfpathlineto{\pgfqpoint{4.975066in}{2.753767in}}%
\pgfpathlineto{\pgfqpoint{5.008978in}{2.833486in}}%
\pgfpathlineto{\pgfqpoint{5.045229in}{2.975136in}}%
\pgfpathlineto{\pgfqpoint{5.080310in}{2.538155in}}%
\pgfpathlineto{\pgfqpoint{5.116561in}{3.573052in}}%
\pgfpathlineto{\pgfqpoint{5.151643in}{3.806835in}}%
\pgfpathlineto{\pgfqpoint{5.187894in}{2.951187in}}%
\pgfpathlineto{\pgfqpoint{5.224145in}{2.899792in}}%
\pgfpathlineto{\pgfqpoint{5.259226in}{2.828101in}}%
\pgfpathlineto{\pgfqpoint{5.295477in}{2.963155in}}%
\pgfpathlineto{\pgfqpoint{5.330558in}{2.279458in}}%
\pgfpathlineto{\pgfqpoint{5.366809in}{2.766138in}}%
\pgfpathlineto{\pgfqpoint{5.403060in}{2.857722in}}%
\pgfpathlineto{\pgfqpoint{5.435803in}{2.546003in}}%
\pgfpathlineto{\pgfqpoint{5.472054in}{3.260233in}}%
\pgfpathlineto{\pgfqpoint{5.507135in}{1.303811in}}%
\pgfpathlineto{\pgfqpoint{5.543386in}{2.376694in}}%
\pgfpathlineto{\pgfqpoint{5.578468in}{2.683938in}}%
\pgfpathlineto{\pgfqpoint{5.614719in}{2.741658in}}%
\pgfpathlineto{\pgfqpoint{5.650970in}{3.081996in}}%
\pgfpathlineto{\pgfqpoint{5.686051in}{3.253637in}}%
\pgfpathlineto{\pgfqpoint{5.722302in}{3.053557in}}%
\pgfpathlineto{\pgfqpoint{5.757384in}{2.815196in}}%
\pgfpathlineto{\pgfqpoint{5.793634in}{2.217699in}}%
\pgfpathlineto{\pgfqpoint{5.829885in}{3.180078in}}%
\pgfpathlineto{\pgfqpoint{5.862628in}{3.083876in}}%
\pgfpathlineto{\pgfqpoint{5.898879in}{3.111809in}}%
\pgfpathlineto{\pgfqpoint{5.933960in}{1.954301in}}%
\pgfpathlineto{\pgfqpoint{5.970211in}{2.610824in}}%
\pgfpathlineto{\pgfqpoint{6.005293in}{2.191656in}}%
\pgfpathlineto{\pgfqpoint{6.041544in}{3.007671in}}%
\pgfpathlineto{\pgfqpoint{6.077795in}{1.842057in}}%
\pgfpathlineto{\pgfqpoint{6.112876in}{3.755064in}}%
\pgfpathlineto{\pgfqpoint{6.149127in}{3.353995in}}%
\pgfpathlineto{\pgfqpoint{6.184209in}{3.465488in}}%
\pgfpathlineto{\pgfqpoint{6.220459in}{2.270688in}}%
\pgfpathlineto{\pgfqpoint{6.256710in}{2.923870in}}%
\pgfpathlineto{\pgfqpoint{6.289453in}{2.842407in}}%
\pgfpathlineto{\pgfqpoint{6.325704in}{2.447433in}}%
\pgfpathlineto{\pgfqpoint{6.360786in}{2.756572in}}%
\pgfpathlineto{\pgfqpoint{6.397036in}{3.860802in}}%
\pgfpathlineto{\pgfqpoint{6.432118in}{2.015795in}}%
\pgfpathlineto{\pgfqpoint{6.468369in}{1.938476in}}%
\pgfpathlineto{\pgfqpoint{6.504620in}{3.489536in}}%
\pgfpathlineto{\pgfqpoint{6.539701in}{3.109723in}}%
\pgfpathlineto{\pgfqpoint{6.575952in}{2.672412in}}%
\pgfpathlineto{\pgfqpoint{6.611034in}{2.701429in}}%
\pgfpathlineto{\pgfqpoint{6.647285in}{2.403060in}}%
\pgfpathlineto{\pgfqpoint{6.683535in}{2.035607in}}%
\pgfpathlineto{\pgfqpoint{6.717448in}{3.120970in}}%
\pgfpathlineto{\pgfqpoint{6.753698in}{2.779988in}}%
\pgfpathlineto{\pgfqpoint{6.788780in}{3.479311in}}%
\pgfpathlineto{\pgfqpoint{6.825031in}{2.415667in}}%
\pgfpathlineto{\pgfqpoint{6.860112in}{3.326564in}}%
\pgfpathlineto{\pgfqpoint{6.896363in}{2.309320in}}%
\pgfpathlineto{\pgfqpoint{6.932614in}{3.119544in}}%
\pgfpathlineto{\pgfqpoint{6.967696in}{3.086908in}}%
\pgfpathlineto{\pgfqpoint{7.003947in}{2.668570in}}%
\pgfpathlineto{\pgfqpoint{7.039028in}{3.118668in}}%
\pgfpathlineto{\pgfqpoint{7.075279in}{2.728517in}}%
\pgfpathlineto{\pgfqpoint{7.111530in}{3.146076in}}%
\pgfpathlineto{\pgfqpoint{7.144273in}{2.492699in}}%
\pgfpathlineto{\pgfqpoint{7.180524in}{3.013631in}}%
\pgfpathlineto{\pgfqpoint{7.215605in}{2.774380in}}%
\pgfpathlineto{\pgfqpoint{7.251856in}{3.274070in}}%
\pgfpathlineto{\pgfqpoint{7.286937in}{2.478771in}}%
\pgfpathlineto{\pgfqpoint{7.323188in}{2.255979in}}%
\pgfpathlineto{\pgfqpoint{7.359439in}{3.326564in}}%
\pgfpathlineto{\pgfqpoint{7.394521in}{2.657121in}}%
\pgfpathlineto{\pgfqpoint{7.430772in}{2.724266in}}%
\pgfpathlineto{\pgfqpoint{7.465853in}{2.818557in}}%
\pgfpathlineto{\pgfqpoint{7.502104in}{2.806490in}}%
\pgfpathlineto{\pgfqpoint{7.538355in}{3.239844in}}%
\pgfpathlineto{\pgfqpoint{7.571098in}{2.704497in}}%
\pgfpathlineto{\pgfqpoint{7.607349in}{3.048344in}}%
\pgfpathlineto{\pgfqpoint{7.642430in}{3.470382in}}%
\pgfpathlineto{\pgfqpoint{7.678681in}{2.663661in}}%
\pgfpathlineto{\pgfqpoint{7.713763in}{3.791311in}}%
\pgfpathlineto{\pgfqpoint{7.750013in}{2.935729in}}%
\pgfpathlineto{\pgfqpoint{7.786264in}{2.720863in}}%
\pgfpathlineto{\pgfqpoint{7.821346in}{2.728460in}}%
\pgfpathlineto{\pgfqpoint{7.857597in}{2.946924in}}%
\pgfpathlineto{\pgfqpoint{7.892678in}{2.917526in}}%
\pgfpathlineto{\pgfqpoint{7.928929in}{2.879690in}}%
\pgfpathlineto{\pgfqpoint{7.965180in}{2.421873in}}%
\pgfpathlineto{\pgfqpoint{7.997923in}{2.541721in}}%
\pgfpathlineto{\pgfqpoint{8.069255in}{2.924865in}}%
\pgfpathlineto{\pgfqpoint{8.105506in}{2.634560in}}%
\pgfpathlineto{\pgfqpoint{8.140588in}{2.607009in}}%
\pgfpathlineto{\pgfqpoint{8.176839in}{2.371687in}}%
\pgfpathlineto{\pgfqpoint{8.213089in}{3.313127in}}%
\pgfpathlineto{\pgfqpoint{8.248171in}{2.982386in}}%
\pgfpathlineto{\pgfqpoint{8.284422in}{3.155918in}}%
\pgfpathlineto{\pgfqpoint{8.319503in}{2.974344in}}%
\pgfpathlineto{\pgfqpoint{8.355754in}{2.932424in}}%
\pgfpathlineto{\pgfqpoint{8.392005in}{2.701916in}}%
\pgfpathlineto{\pgfqpoint{8.425917in}{3.188880in}}%
\pgfpathlineto{\pgfqpoint{8.462168in}{2.823028in}}%
\pgfpathlineto{\pgfqpoint{8.497250in}{2.545637in}}%
\pgfpathlineto{\pgfqpoint{8.533501in}{2.523056in}}%
\pgfpathlineto{\pgfqpoint{8.568582in}{2.936513in}}%
\pgfpathlineto{\pgfqpoint{8.604833in}{3.047449in}}%
\pgfpathlineto{\pgfqpoint{8.641084in}{3.236598in}}%
\pgfpathlineto{\pgfqpoint{8.676165in}{3.125182in}}%
\pgfpathlineto{\pgfqpoint{8.712416in}{3.025240in}}%
\pgfpathlineto{\pgfqpoint{8.747498in}{2.950031in}}%
\pgfpathlineto{\pgfqpoint{8.783749in}{3.317063in}}%
\pgfpathlineto{\pgfqpoint{8.820000in}{3.422070in}}%
\pgfpathlineto{\pgfqpoint{8.852742in}{1.867817in}}%
\pgfpathlineto{\pgfqpoint{8.888993in}{4.202373in}}%
\pgfpathlineto{\pgfqpoint{8.924075in}{1.420502in}}%
\pgfpathlineto{\pgfqpoint{8.960326in}{2.680777in}}%
\pgfpathlineto{\pgfqpoint{8.995407in}{3.334365in}}%
\pgfpathlineto{\pgfqpoint{9.031658in}{3.663998in}}%
\pgfpathlineto{\pgfqpoint{9.067909in}{3.119758in}}%
\pgfpathlineto{\pgfqpoint{9.102990in}{3.551679in}}%
\pgfpathlineto{\pgfqpoint{9.139241in}{2.741166in}}%
\pgfpathlineto{\pgfqpoint{9.174323in}{2.815180in}}%
\pgfpathlineto{\pgfqpoint{9.210574in}{2.493570in}}%
\pgfpathlineto{\pgfqpoint{9.246825in}{4.305513in}}%
\pgfpathlineto{\pgfqpoint{9.279567in}{3.875870in}}%
\pgfpathlineto{\pgfqpoint{9.315818in}{3.459131in}}%
\pgfpathlineto{\pgfqpoint{9.350900in}{2.679554in}}%
\pgfpathlineto{\pgfqpoint{9.387151in}{1.982105in}}%
\pgfpathlineto{\pgfqpoint{9.422232in}{2.683516in}}%
\pgfpathlineto{\pgfqpoint{9.458483in}{2.600519in}}%
\pgfpathlineto{\pgfqpoint{9.494734in}{3.382868in}}%
\pgfpathlineto{\pgfqpoint{9.529816in}{2.745978in}}%
\pgfpathlineto{\pgfqpoint{9.566066in}{2.265760in}}%
\pgfpathlineto{\pgfqpoint{9.601148in}{2.430078in}}%
\pgfpathlineto{\pgfqpoint{9.601148in}{2.430078in}}%
\pgfusepath{stroke}%
\end{pgfscope}%
\begin{pgfscope}%
\pgfsetrectcap%
\pgfsetmiterjoin%
\pgfsetlinewidth{1.254687pt}%
\definecolor{currentstroke}{rgb}{0.150000,0.150000,0.150000}%
\pgfsetstrokecolor{currentstroke}%
\pgfsetdash{}{0pt}%
\pgfpathmoveto{\pgfqpoint{1.095050in}{0.782881in}}%
\pgfpathlineto{\pgfqpoint{1.095050in}{5.402881in}}%
\pgfusepath{stroke}%
\end{pgfscope}%
\begin{pgfscope}%
\pgfsetrectcap%
\pgfsetmiterjoin%
\pgfsetlinewidth{1.254687pt}%
\definecolor{currentstroke}{rgb}{0.150000,0.150000,0.150000}%
\pgfsetstrokecolor{currentstroke}%
\pgfsetdash{}{0pt}%
\pgfpathmoveto{\pgfqpoint{9.601148in}{0.782881in}}%
\pgfpathlineto{\pgfqpoint{9.601148in}{5.402881in}}%
\pgfusepath{stroke}%
\end{pgfscope}%
\begin{pgfscope}%
\pgfsetrectcap%
\pgfsetmiterjoin%
\pgfsetlinewidth{1.254687pt}%
\definecolor{currentstroke}{rgb}{0.150000,0.150000,0.150000}%
\pgfsetstrokecolor{currentstroke}%
\pgfsetdash{}{0pt}%
\pgfpathmoveto{\pgfqpoint{1.095050in}{0.782881in}}%
\pgfpathlineto{\pgfqpoint{9.601148in}{0.782881in}}%
\pgfusepath{stroke}%
\end{pgfscope}%
\begin{pgfscope}%
\pgfsetrectcap%
\pgfsetmiterjoin%
\pgfsetlinewidth{1.254687pt}%
\definecolor{currentstroke}{rgb}{0.150000,0.150000,0.150000}%
\pgfsetstrokecolor{currentstroke}%
\pgfsetdash{}{0pt}%
\pgfpathmoveto{\pgfqpoint{1.095050in}{5.402881in}}%
\pgfpathlineto{\pgfqpoint{9.601148in}{5.402881in}}%
\pgfusepath{stroke}%
\end{pgfscope}%
\begin{pgfscope}%
\pgfsetbuttcap%
\pgfsetmiterjoin%
\definecolor{currentfill}{rgb}{1.000000,1.000000,1.000000}%
\pgfsetfillcolor{currentfill}%
\pgfsetlinewidth{0.000000pt}%
\definecolor{currentstroke}{rgb}{0.000000,0.000000,0.000000}%
\pgfsetstrokecolor{currentstroke}%
\pgfsetstrokeopacity{0.000000}%
\pgfsetdash{}{0pt}%
\pgfpathmoveto{\pgfqpoint{9.884685in}{0.782881in}}%
\pgfpathlineto{\pgfqpoint{12.720050in}{0.782881in}}%
\pgfpathlineto{\pgfqpoint{12.720050in}{5.402881in}}%
\pgfpathlineto{\pgfqpoint{9.884685in}{5.402881in}}%
\pgfpathlineto{\pgfqpoint{9.884685in}{0.782881in}}%
\pgfpathclose%
\pgfusepath{fill}%
\end{pgfscope}%
\begin{pgfscope}%
\pgfpathrectangle{\pgfqpoint{9.884685in}{0.782881in}}{\pgfqpoint{2.835366in}{4.620000in}}%
\pgfusepath{clip}%
\pgfsetroundcap%
\pgfsetroundjoin%
\pgfsetlinewidth{1.003750pt}%
\definecolor{currentstroke}{rgb}{0.800000,0.800000,0.800000}%
\pgfsetstrokecolor{currentstroke}%
\pgfsetstrokeopacity{0.400000}%
\pgfsetdash{}{0pt}%
\pgfpathmoveto{\pgfqpoint{9.884685in}{0.782881in}}%
\pgfpathlineto{\pgfqpoint{9.884685in}{5.402881in}}%
\pgfusepath{stroke}%
\end{pgfscope}%
\begin{pgfscope}%
\definecolor{textcolor}{rgb}{0.150000,0.150000,0.150000}%
\pgfsetstrokecolor{textcolor}%
\pgfsetfillcolor{textcolor}%
\pgftext[x=9.884685in,y=0.650937in,,top]{\color{textcolor}\sffamily\fontsize{19.000000}{22.800000}\selectfont 0}%
\end{pgfscope}%
\begin{pgfscope}%
\pgfpathrectangle{\pgfqpoint{9.884685in}{0.782881in}}{\pgfqpoint{2.835366in}{4.620000in}}%
\pgfusepath{clip}%
\pgfsetroundcap%
\pgfsetroundjoin%
\pgfsetlinewidth{1.003750pt}%
\definecolor{currentstroke}{rgb}{0.800000,0.800000,0.800000}%
\pgfsetstrokecolor{currentstroke}%
\pgfsetstrokeopacity{0.400000}%
\pgfsetdash{}{0pt}%
\pgfpathmoveto{\pgfqpoint{11.170565in}{0.782881in}}%
\pgfpathlineto{\pgfqpoint{11.170565in}{5.402881in}}%
\pgfusepath{stroke}%
\end{pgfscope}%
\begin{pgfscope}%
\definecolor{textcolor}{rgb}{0.150000,0.150000,0.150000}%
\pgfsetstrokecolor{textcolor}%
\pgfsetfillcolor{textcolor}%
\pgftext[x=11.170565in,y=0.650937in,,top]{\color{textcolor}\sffamily\fontsize{19.000000}{22.800000}\selectfont 20}%
\end{pgfscope}%
\begin{pgfscope}%
\pgfpathrectangle{\pgfqpoint{9.884685in}{0.782881in}}{\pgfqpoint{2.835366in}{4.620000in}}%
\pgfusepath{clip}%
\pgfsetroundcap%
\pgfsetroundjoin%
\pgfsetlinewidth{1.003750pt}%
\definecolor{currentstroke}{rgb}{0.800000,0.800000,0.800000}%
\pgfsetstrokecolor{currentstroke}%
\pgfsetstrokeopacity{0.400000}%
\pgfsetdash{}{0pt}%
\pgfpathmoveto{\pgfqpoint{12.456445in}{0.782881in}}%
\pgfpathlineto{\pgfqpoint{12.456445in}{5.402881in}}%
\pgfusepath{stroke}%
\end{pgfscope}%
\begin{pgfscope}%
\definecolor{textcolor}{rgb}{0.150000,0.150000,0.150000}%
\pgfsetstrokecolor{textcolor}%
\pgfsetfillcolor{textcolor}%
\pgftext[x=12.456445in,y=0.650937in,,top]{\color{textcolor}\sffamily\fontsize{19.000000}{22.800000}\selectfont 40}%
\end{pgfscope}%
\begin{pgfscope}%
\definecolor{textcolor}{rgb}{0.150000,0.150000,0.150000}%
\pgfsetstrokecolor{textcolor}%
\pgfsetfillcolor{textcolor}%
\pgftext[x=11.302367in,y=0.354042in,,top]{\color{textcolor}\sffamily\fontsize{20.000000}{24.000000}\selectfont Nombre d'occurences}%
\end{pgfscope}%
\begin{pgfscope}%
\pgfpathrectangle{\pgfqpoint{9.884685in}{0.782881in}}{\pgfqpoint{2.835366in}{4.620000in}}%
\pgfusepath{clip}%
\pgfsetbuttcap%
\pgfsetmiterjoin%
\definecolor{currentfill}{rgb}{0.956863,0.309804,0.223529}%
\pgfsetfillcolor{currentfill}%
\pgfsetfillopacity{0.750000}%
\pgfsetlinewidth{1.003750pt}%
\definecolor{currentstroke}{rgb}{1.000000,1.000000,1.000000}%
\pgfsetstrokecolor{currentstroke}%
\pgfsetdash{}{0pt}%
\pgfpathmoveto{\pgfqpoint{9.884685in}{0.992881in}}%
\pgfpathlineto{\pgfqpoint{9.948979in}{0.992881in}}%
\pgfpathlineto{\pgfqpoint{9.948979in}{1.167881in}}%
\pgfpathlineto{\pgfqpoint{9.884685in}{1.167881in}}%
\pgfpathlineto{\pgfqpoint{9.884685in}{0.992881in}}%
\pgfpathclose%
\pgfusepath{stroke,fill}%
\end{pgfscope}%
\begin{pgfscope}%
\pgfpathrectangle{\pgfqpoint{9.884685in}{0.782881in}}{\pgfqpoint{2.835366in}{4.620000in}}%
\pgfusepath{clip}%
\pgfsetbuttcap%
\pgfsetmiterjoin%
\definecolor{currentfill}{rgb}{0.956863,0.309804,0.223529}%
\pgfsetfillcolor{currentfill}%
\pgfsetfillopacity{0.750000}%
\pgfsetlinewidth{1.003750pt}%
\definecolor{currentstroke}{rgb}{1.000000,1.000000,1.000000}%
\pgfsetstrokecolor{currentstroke}%
\pgfsetdash{}{0pt}%
\pgfpathmoveto{\pgfqpoint{9.884685in}{1.167881in}}%
\pgfpathlineto{\pgfqpoint{10.013273in}{1.167881in}}%
\pgfpathlineto{\pgfqpoint{10.013273in}{1.342881in}}%
\pgfpathlineto{\pgfqpoint{9.884685in}{1.342881in}}%
\pgfpathlineto{\pgfqpoint{9.884685in}{1.167881in}}%
\pgfpathclose%
\pgfusepath{stroke,fill}%
\end{pgfscope}%
\begin{pgfscope}%
\pgfpathrectangle{\pgfqpoint{9.884685in}{0.782881in}}{\pgfqpoint{2.835366in}{4.620000in}}%
\pgfusepath{clip}%
\pgfsetbuttcap%
\pgfsetmiterjoin%
\definecolor{currentfill}{rgb}{0.956863,0.309804,0.223529}%
\pgfsetfillcolor{currentfill}%
\pgfsetfillopacity{0.750000}%
\pgfsetlinewidth{1.003750pt}%
\definecolor{currentstroke}{rgb}{1.000000,1.000000,1.000000}%
\pgfsetstrokecolor{currentstroke}%
\pgfsetdash{}{0pt}%
\pgfpathmoveto{\pgfqpoint{9.884685in}{1.342881in}}%
\pgfpathlineto{\pgfqpoint{10.013273in}{1.342881in}}%
\pgfpathlineto{\pgfqpoint{10.013273in}{1.517881in}}%
\pgfpathlineto{\pgfqpoint{9.884685in}{1.517881in}}%
\pgfpathlineto{\pgfqpoint{9.884685in}{1.342881in}}%
\pgfpathclose%
\pgfusepath{stroke,fill}%
\end{pgfscope}%
\begin{pgfscope}%
\pgfpathrectangle{\pgfqpoint{9.884685in}{0.782881in}}{\pgfqpoint{2.835366in}{4.620000in}}%
\pgfusepath{clip}%
\pgfsetbuttcap%
\pgfsetmiterjoin%
\definecolor{currentfill}{rgb}{0.956863,0.309804,0.223529}%
\pgfsetfillcolor{currentfill}%
\pgfsetfillopacity{0.750000}%
\pgfsetlinewidth{1.003750pt}%
\definecolor{currentstroke}{rgb}{1.000000,1.000000,1.000000}%
\pgfsetstrokecolor{currentstroke}%
\pgfsetdash{}{0pt}%
\pgfpathmoveto{\pgfqpoint{9.884685in}{1.517881in}}%
\pgfpathlineto{\pgfqpoint{9.884685in}{1.517881in}}%
\pgfpathlineto{\pgfqpoint{9.884685in}{1.692881in}}%
\pgfpathlineto{\pgfqpoint{9.884685in}{1.692881in}}%
\pgfpathlineto{\pgfqpoint{9.884685in}{1.517881in}}%
\pgfpathclose%
\pgfusepath{stroke,fill}%
\end{pgfscope}%
\begin{pgfscope}%
\pgfpathrectangle{\pgfqpoint{9.884685in}{0.782881in}}{\pgfqpoint{2.835366in}{4.620000in}}%
\pgfusepath{clip}%
\pgfsetbuttcap%
\pgfsetmiterjoin%
\definecolor{currentfill}{rgb}{0.956863,0.309804,0.223529}%
\pgfsetfillcolor{currentfill}%
\pgfsetfillopacity{0.750000}%
\pgfsetlinewidth{1.003750pt}%
\definecolor{currentstroke}{rgb}{1.000000,1.000000,1.000000}%
\pgfsetstrokecolor{currentstroke}%
\pgfsetdash{}{0pt}%
\pgfpathmoveto{\pgfqpoint{9.884685in}{1.692881in}}%
\pgfpathlineto{\pgfqpoint{10.270449in}{1.692881in}}%
\pgfpathlineto{\pgfqpoint{10.270449in}{1.867881in}}%
\pgfpathlineto{\pgfqpoint{9.884685in}{1.867881in}}%
\pgfpathlineto{\pgfqpoint{9.884685in}{1.692881in}}%
\pgfpathclose%
\pgfusepath{stroke,fill}%
\end{pgfscope}%
\begin{pgfscope}%
\pgfpathrectangle{\pgfqpoint{9.884685in}{0.782881in}}{\pgfqpoint{2.835366in}{4.620000in}}%
\pgfusepath{clip}%
\pgfsetbuttcap%
\pgfsetmiterjoin%
\definecolor{currentfill}{rgb}{0.956863,0.309804,0.223529}%
\pgfsetfillcolor{currentfill}%
\pgfsetfillopacity{0.750000}%
\pgfsetlinewidth{1.003750pt}%
\definecolor{currentstroke}{rgb}{1.000000,1.000000,1.000000}%
\pgfsetstrokecolor{currentstroke}%
\pgfsetdash{}{0pt}%
\pgfpathmoveto{\pgfqpoint{9.884685in}{1.867881in}}%
\pgfpathlineto{\pgfqpoint{10.334743in}{1.867881in}}%
\pgfpathlineto{\pgfqpoint{10.334743in}{2.042881in}}%
\pgfpathlineto{\pgfqpoint{9.884685in}{2.042881in}}%
\pgfpathlineto{\pgfqpoint{9.884685in}{1.867881in}}%
\pgfpathclose%
\pgfusepath{stroke,fill}%
\end{pgfscope}%
\begin{pgfscope}%
\pgfpathrectangle{\pgfqpoint{9.884685in}{0.782881in}}{\pgfqpoint{2.835366in}{4.620000in}}%
\pgfusepath{clip}%
\pgfsetbuttcap%
\pgfsetmiterjoin%
\definecolor{currentfill}{rgb}{0.956863,0.309804,0.223529}%
\pgfsetfillcolor{currentfill}%
\pgfsetfillopacity{0.750000}%
\pgfsetlinewidth{1.003750pt}%
\definecolor{currentstroke}{rgb}{1.000000,1.000000,1.000000}%
\pgfsetstrokecolor{currentstroke}%
\pgfsetdash{}{0pt}%
\pgfpathmoveto{\pgfqpoint{9.884685in}{2.042881in}}%
\pgfpathlineto{\pgfqpoint{10.270449in}{2.042881in}}%
\pgfpathlineto{\pgfqpoint{10.270449in}{2.217881in}}%
\pgfpathlineto{\pgfqpoint{9.884685in}{2.217881in}}%
\pgfpathlineto{\pgfqpoint{9.884685in}{2.042881in}}%
\pgfpathclose%
\pgfusepath{stroke,fill}%
\end{pgfscope}%
\begin{pgfscope}%
\pgfpathrectangle{\pgfqpoint{9.884685in}{0.782881in}}{\pgfqpoint{2.835366in}{4.620000in}}%
\pgfusepath{clip}%
\pgfsetbuttcap%
\pgfsetmiterjoin%
\definecolor{currentfill}{rgb}{0.956863,0.309804,0.223529}%
\pgfsetfillcolor{currentfill}%
\pgfsetfillopacity{0.750000}%
\pgfsetlinewidth{1.003750pt}%
\definecolor{currentstroke}{rgb}{1.000000,1.000000,1.000000}%
\pgfsetstrokecolor{currentstroke}%
\pgfsetdash{}{0pt}%
\pgfpathmoveto{\pgfqpoint{9.884685in}{2.217881in}}%
\pgfpathlineto{\pgfqpoint{10.463331in}{2.217881in}}%
\pgfpathlineto{\pgfqpoint{10.463331in}{2.392881in}}%
\pgfpathlineto{\pgfqpoint{9.884685in}{2.392881in}}%
\pgfpathlineto{\pgfqpoint{9.884685in}{2.217881in}}%
\pgfpathclose%
\pgfusepath{stroke,fill}%
\end{pgfscope}%
\begin{pgfscope}%
\pgfpathrectangle{\pgfqpoint{9.884685in}{0.782881in}}{\pgfqpoint{2.835366in}{4.620000in}}%
\pgfusepath{clip}%
\pgfsetbuttcap%
\pgfsetmiterjoin%
\definecolor{currentfill}{rgb}{0.956863,0.309804,0.223529}%
\pgfsetfillcolor{currentfill}%
\pgfsetfillopacity{0.750000}%
\pgfsetlinewidth{1.003750pt}%
\definecolor{currentstroke}{rgb}{1.000000,1.000000,1.000000}%
\pgfsetstrokecolor{currentstroke}%
\pgfsetdash{}{0pt}%
\pgfpathmoveto{\pgfqpoint{9.884685in}{2.392881in}}%
\pgfpathlineto{\pgfqpoint{11.170565in}{2.392881in}}%
\pgfpathlineto{\pgfqpoint{11.170565in}{2.567881in}}%
\pgfpathlineto{\pgfqpoint{9.884685in}{2.567881in}}%
\pgfpathlineto{\pgfqpoint{9.884685in}{2.392881in}}%
\pgfpathclose%
\pgfusepath{stroke,fill}%
\end{pgfscope}%
\begin{pgfscope}%
\pgfpathrectangle{\pgfqpoint{9.884685in}{0.782881in}}{\pgfqpoint{2.835366in}{4.620000in}}%
\pgfusepath{clip}%
\pgfsetbuttcap%
\pgfsetmiterjoin%
\definecolor{currentfill}{rgb}{0.956863,0.309804,0.223529}%
\pgfsetfillcolor{currentfill}%
\pgfsetfillopacity{0.750000}%
\pgfsetlinewidth{1.003750pt}%
\definecolor{currentstroke}{rgb}{1.000000,1.000000,1.000000}%
\pgfsetstrokecolor{currentstroke}%
\pgfsetdash{}{0pt}%
\pgfpathmoveto{\pgfqpoint{9.884685in}{2.567881in}}%
\pgfpathlineto{\pgfqpoint{12.585033in}{2.567881in}}%
\pgfpathlineto{\pgfqpoint{12.585033in}{2.742881in}}%
\pgfpathlineto{\pgfqpoint{9.884685in}{2.742881in}}%
\pgfpathlineto{\pgfqpoint{9.884685in}{2.567881in}}%
\pgfpathclose%
\pgfusepath{stroke,fill}%
\end{pgfscope}%
\begin{pgfscope}%
\pgfpathrectangle{\pgfqpoint{9.884685in}{0.782881in}}{\pgfqpoint{2.835366in}{4.620000in}}%
\pgfusepath{clip}%
\pgfsetbuttcap%
\pgfsetmiterjoin%
\definecolor{currentfill}{rgb}{0.956863,0.309804,0.223529}%
\pgfsetfillcolor{currentfill}%
\pgfsetfillopacity{0.750000}%
\pgfsetlinewidth{1.003750pt}%
\definecolor{currentstroke}{rgb}{1.000000,1.000000,1.000000}%
\pgfsetstrokecolor{currentstroke}%
\pgfsetdash{}{0pt}%
\pgfpathmoveto{\pgfqpoint{9.884685in}{2.742881in}}%
\pgfpathlineto{\pgfqpoint{12.006387in}{2.742881in}}%
\pgfpathlineto{\pgfqpoint{12.006387in}{2.917881in}}%
\pgfpathlineto{\pgfqpoint{9.884685in}{2.917881in}}%
\pgfpathlineto{\pgfqpoint{9.884685in}{2.742881in}}%
\pgfpathclose%
\pgfusepath{stroke,fill}%
\end{pgfscope}%
\begin{pgfscope}%
\pgfpathrectangle{\pgfqpoint{9.884685in}{0.782881in}}{\pgfqpoint{2.835366in}{4.620000in}}%
\pgfusepath{clip}%
\pgfsetbuttcap%
\pgfsetmiterjoin%
\definecolor{currentfill}{rgb}{0.956863,0.309804,0.223529}%
\pgfsetfillcolor{currentfill}%
\pgfsetfillopacity{0.750000}%
\pgfsetlinewidth{1.003750pt}%
\definecolor{currentstroke}{rgb}{1.000000,1.000000,1.000000}%
\pgfsetstrokecolor{currentstroke}%
\pgfsetdash{}{0pt}%
\pgfpathmoveto{\pgfqpoint{9.884685in}{2.917881in}}%
\pgfpathlineto{\pgfqpoint{12.263563in}{2.917881in}}%
\pgfpathlineto{\pgfqpoint{12.263563in}{3.092881in}}%
\pgfpathlineto{\pgfqpoint{9.884685in}{3.092881in}}%
\pgfpathlineto{\pgfqpoint{9.884685in}{2.917881in}}%
\pgfpathclose%
\pgfusepath{stroke,fill}%
\end{pgfscope}%
\begin{pgfscope}%
\pgfpathrectangle{\pgfqpoint{9.884685in}{0.782881in}}{\pgfqpoint{2.835366in}{4.620000in}}%
\pgfusepath{clip}%
\pgfsetbuttcap%
\pgfsetmiterjoin%
\definecolor{currentfill}{rgb}{0.956863,0.309804,0.223529}%
\pgfsetfillcolor{currentfill}%
\pgfsetfillopacity{0.750000}%
\pgfsetlinewidth{1.003750pt}%
\definecolor{currentstroke}{rgb}{1.000000,1.000000,1.000000}%
\pgfsetstrokecolor{currentstroke}%
\pgfsetdash{}{0pt}%
\pgfpathmoveto{\pgfqpoint{9.884685in}{3.092881in}}%
\pgfpathlineto{\pgfqpoint{11.427741in}{3.092881in}}%
\pgfpathlineto{\pgfqpoint{11.427741in}{3.267881in}}%
\pgfpathlineto{\pgfqpoint{9.884685in}{3.267881in}}%
\pgfpathlineto{\pgfqpoint{9.884685in}{3.092881in}}%
\pgfpathclose%
\pgfusepath{stroke,fill}%
\end{pgfscope}%
\begin{pgfscope}%
\pgfpathrectangle{\pgfqpoint{9.884685in}{0.782881in}}{\pgfqpoint{2.835366in}{4.620000in}}%
\pgfusepath{clip}%
\pgfsetbuttcap%
\pgfsetmiterjoin%
\definecolor{currentfill}{rgb}{0.956863,0.309804,0.223529}%
\pgfsetfillcolor{currentfill}%
\pgfsetfillopacity{0.750000}%
\pgfsetlinewidth{1.003750pt}%
\definecolor{currentstroke}{rgb}{1.000000,1.000000,1.000000}%
\pgfsetstrokecolor{currentstroke}%
\pgfsetdash{}{0pt}%
\pgfpathmoveto{\pgfqpoint{9.884685in}{3.267881in}}%
\pgfpathlineto{\pgfqpoint{11.170565in}{3.267881in}}%
\pgfpathlineto{\pgfqpoint{11.170565in}{3.442881in}}%
\pgfpathlineto{\pgfqpoint{9.884685in}{3.442881in}}%
\pgfpathlineto{\pgfqpoint{9.884685in}{3.267881in}}%
\pgfpathclose%
\pgfusepath{stroke,fill}%
\end{pgfscope}%
\begin{pgfscope}%
\pgfpathrectangle{\pgfqpoint{9.884685in}{0.782881in}}{\pgfqpoint{2.835366in}{4.620000in}}%
\pgfusepath{clip}%
\pgfsetbuttcap%
\pgfsetmiterjoin%
\definecolor{currentfill}{rgb}{0.956863,0.309804,0.223529}%
\pgfsetfillcolor{currentfill}%
\pgfsetfillopacity{0.750000}%
\pgfsetlinewidth{1.003750pt}%
\definecolor{currentstroke}{rgb}{1.000000,1.000000,1.000000}%
\pgfsetstrokecolor{currentstroke}%
\pgfsetdash{}{0pt}%
\pgfpathmoveto{\pgfqpoint{9.884685in}{3.442881in}}%
\pgfpathlineto{\pgfqpoint{10.784801in}{3.442881in}}%
\pgfpathlineto{\pgfqpoint{10.784801in}{3.617881in}}%
\pgfpathlineto{\pgfqpoint{9.884685in}{3.617881in}}%
\pgfpathlineto{\pgfqpoint{9.884685in}{3.442881in}}%
\pgfpathclose%
\pgfusepath{stroke,fill}%
\end{pgfscope}%
\begin{pgfscope}%
\pgfpathrectangle{\pgfqpoint{9.884685in}{0.782881in}}{\pgfqpoint{2.835366in}{4.620000in}}%
\pgfusepath{clip}%
\pgfsetbuttcap%
\pgfsetmiterjoin%
\definecolor{currentfill}{rgb}{0.956863,0.309804,0.223529}%
\pgfsetfillcolor{currentfill}%
\pgfsetfillopacity{0.750000}%
\pgfsetlinewidth{1.003750pt}%
\definecolor{currentstroke}{rgb}{1.000000,1.000000,1.000000}%
\pgfsetstrokecolor{currentstroke}%
\pgfsetdash{}{0pt}%
\pgfpathmoveto{\pgfqpoint{9.884685in}{3.617881in}}%
\pgfpathlineto{\pgfqpoint{10.141861in}{3.617881in}}%
\pgfpathlineto{\pgfqpoint{10.141861in}{3.792881in}}%
\pgfpathlineto{\pgfqpoint{9.884685in}{3.792881in}}%
\pgfpathlineto{\pgfqpoint{9.884685in}{3.617881in}}%
\pgfpathclose%
\pgfusepath{stroke,fill}%
\end{pgfscope}%
\begin{pgfscope}%
\pgfpathrectangle{\pgfqpoint{9.884685in}{0.782881in}}{\pgfqpoint{2.835366in}{4.620000in}}%
\pgfusepath{clip}%
\pgfsetbuttcap%
\pgfsetmiterjoin%
\definecolor{currentfill}{rgb}{0.956863,0.309804,0.223529}%
\pgfsetfillcolor{currentfill}%
\pgfsetfillopacity{0.750000}%
\pgfsetlinewidth{1.003750pt}%
\definecolor{currentstroke}{rgb}{1.000000,1.000000,1.000000}%
\pgfsetstrokecolor{currentstroke}%
\pgfsetdash{}{0pt}%
\pgfpathmoveto{\pgfqpoint{9.884685in}{3.792881in}}%
\pgfpathlineto{\pgfqpoint{10.206155in}{3.792881in}}%
\pgfpathlineto{\pgfqpoint{10.206155in}{3.967881in}}%
\pgfpathlineto{\pgfqpoint{9.884685in}{3.967881in}}%
\pgfpathlineto{\pgfqpoint{9.884685in}{3.792881in}}%
\pgfpathclose%
\pgfusepath{stroke,fill}%
\end{pgfscope}%
\begin{pgfscope}%
\pgfpathrectangle{\pgfqpoint{9.884685in}{0.782881in}}{\pgfqpoint{2.835366in}{4.620000in}}%
\pgfusepath{clip}%
\pgfsetbuttcap%
\pgfsetmiterjoin%
\definecolor{currentfill}{rgb}{0.956863,0.309804,0.223529}%
\pgfsetfillcolor{currentfill}%
\pgfsetfillopacity{0.750000}%
\pgfsetlinewidth{1.003750pt}%
\definecolor{currentstroke}{rgb}{1.000000,1.000000,1.000000}%
\pgfsetstrokecolor{currentstroke}%
\pgfsetdash{}{0pt}%
\pgfpathmoveto{\pgfqpoint{9.884685in}{3.967881in}}%
\pgfpathlineto{\pgfqpoint{10.013273in}{3.967881in}}%
\pgfpathlineto{\pgfqpoint{10.013273in}{4.142881in}}%
\pgfpathlineto{\pgfqpoint{9.884685in}{4.142881in}}%
\pgfpathlineto{\pgfqpoint{9.884685in}{3.967881in}}%
\pgfpathclose%
\pgfusepath{stroke,fill}%
\end{pgfscope}%
\begin{pgfscope}%
\pgfpathrectangle{\pgfqpoint{9.884685in}{0.782881in}}{\pgfqpoint{2.835366in}{4.620000in}}%
\pgfusepath{clip}%
\pgfsetbuttcap%
\pgfsetmiterjoin%
\definecolor{currentfill}{rgb}{0.956863,0.309804,0.223529}%
\pgfsetfillcolor{currentfill}%
\pgfsetfillopacity{0.750000}%
\pgfsetlinewidth{1.003750pt}%
\definecolor{currentstroke}{rgb}{1.000000,1.000000,1.000000}%
\pgfsetstrokecolor{currentstroke}%
\pgfsetdash{}{0pt}%
\pgfpathmoveto{\pgfqpoint{9.884685in}{4.142881in}}%
\pgfpathlineto{\pgfqpoint{10.013273in}{4.142881in}}%
\pgfpathlineto{\pgfqpoint{10.013273in}{4.317881in}}%
\pgfpathlineto{\pgfqpoint{9.884685in}{4.317881in}}%
\pgfpathlineto{\pgfqpoint{9.884685in}{4.142881in}}%
\pgfpathclose%
\pgfusepath{stroke,fill}%
\end{pgfscope}%
\begin{pgfscope}%
\pgfpathrectangle{\pgfqpoint{9.884685in}{0.782881in}}{\pgfqpoint{2.835366in}{4.620000in}}%
\pgfusepath{clip}%
\pgfsetbuttcap%
\pgfsetmiterjoin%
\definecolor{currentfill}{rgb}{0.956863,0.309804,0.223529}%
\pgfsetfillcolor{currentfill}%
\pgfsetfillopacity{0.750000}%
\pgfsetlinewidth{1.003750pt}%
\definecolor{currentstroke}{rgb}{1.000000,1.000000,1.000000}%
\pgfsetstrokecolor{currentstroke}%
\pgfsetdash{}{0pt}%
\pgfpathmoveto{\pgfqpoint{9.884685in}{4.317881in}}%
\pgfpathlineto{\pgfqpoint{9.948979in}{4.317881in}}%
\pgfpathlineto{\pgfqpoint{9.948979in}{4.492881in}}%
\pgfpathlineto{\pgfqpoint{9.884685in}{4.492881in}}%
\pgfpathlineto{\pgfqpoint{9.884685in}{4.317881in}}%
\pgfpathclose%
\pgfusepath{stroke,fill}%
\end{pgfscope}%
\begin{pgfscope}%
\pgfpathrectangle{\pgfqpoint{9.884685in}{0.782881in}}{\pgfqpoint{2.835366in}{4.620000in}}%
\pgfusepath{clip}%
\pgfsetbuttcap%
\pgfsetmiterjoin%
\definecolor{currentfill}{rgb}{0.956863,0.309804,0.223529}%
\pgfsetfillcolor{currentfill}%
\pgfsetfillopacity{0.750000}%
\pgfsetlinewidth{1.003750pt}%
\definecolor{currentstroke}{rgb}{1.000000,1.000000,1.000000}%
\pgfsetstrokecolor{currentstroke}%
\pgfsetdash{}{0pt}%
\pgfpathmoveto{\pgfqpoint{9.884685in}{4.492881in}}%
\pgfpathlineto{\pgfqpoint{9.884685in}{4.492881in}}%
\pgfpathlineto{\pgfqpoint{9.884685in}{4.667881in}}%
\pgfpathlineto{\pgfqpoint{9.884685in}{4.667881in}}%
\pgfpathlineto{\pgfqpoint{9.884685in}{4.492881in}}%
\pgfpathclose%
\pgfusepath{stroke,fill}%
\end{pgfscope}%
\begin{pgfscope}%
\pgfpathrectangle{\pgfqpoint{9.884685in}{0.782881in}}{\pgfqpoint{2.835366in}{4.620000in}}%
\pgfusepath{clip}%
\pgfsetbuttcap%
\pgfsetmiterjoin%
\definecolor{currentfill}{rgb}{0.956863,0.309804,0.223529}%
\pgfsetfillcolor{currentfill}%
\pgfsetfillopacity{0.750000}%
\pgfsetlinewidth{1.003750pt}%
\definecolor{currentstroke}{rgb}{1.000000,1.000000,1.000000}%
\pgfsetstrokecolor{currentstroke}%
\pgfsetdash{}{0pt}%
\pgfpathmoveto{\pgfqpoint{9.884685in}{4.667881in}}%
\pgfpathlineto{\pgfqpoint{9.948979in}{4.667881in}}%
\pgfpathlineto{\pgfqpoint{9.948979in}{4.842881in}}%
\pgfpathlineto{\pgfqpoint{9.884685in}{4.842881in}}%
\pgfpathlineto{\pgfqpoint{9.884685in}{4.667881in}}%
\pgfpathclose%
\pgfusepath{stroke,fill}%
\end{pgfscope}%
\begin{pgfscope}%
\pgfpathrectangle{\pgfqpoint{9.884685in}{0.782881in}}{\pgfqpoint{2.835366in}{4.620000in}}%
\pgfusepath{clip}%
\pgfsetbuttcap%
\pgfsetmiterjoin%
\definecolor{currentfill}{rgb}{0.956863,0.309804,0.223529}%
\pgfsetfillcolor{currentfill}%
\pgfsetfillopacity{0.750000}%
\pgfsetlinewidth{1.003750pt}%
\definecolor{currentstroke}{rgb}{1.000000,1.000000,1.000000}%
\pgfsetstrokecolor{currentstroke}%
\pgfsetdash{}{0pt}%
\pgfpathmoveto{\pgfqpoint{9.884685in}{4.842881in}}%
\pgfpathlineto{\pgfqpoint{9.884685in}{4.842881in}}%
\pgfpathlineto{\pgfqpoint{9.884685in}{5.017881in}}%
\pgfpathlineto{\pgfqpoint{9.884685in}{5.017881in}}%
\pgfpathlineto{\pgfqpoint{9.884685in}{4.842881in}}%
\pgfpathclose%
\pgfusepath{stroke,fill}%
\end{pgfscope}%
\begin{pgfscope}%
\pgfpathrectangle{\pgfqpoint{9.884685in}{0.782881in}}{\pgfqpoint{2.835366in}{4.620000in}}%
\pgfusepath{clip}%
\pgfsetbuttcap%
\pgfsetmiterjoin%
\definecolor{currentfill}{rgb}{0.956863,0.309804,0.223529}%
\pgfsetfillcolor{currentfill}%
\pgfsetfillopacity{0.750000}%
\pgfsetlinewidth{1.003750pt}%
\definecolor{currentstroke}{rgb}{1.000000,1.000000,1.000000}%
\pgfsetstrokecolor{currentstroke}%
\pgfsetdash{}{0pt}%
\pgfpathmoveto{\pgfqpoint{9.884685in}{5.017881in}}%
\pgfpathlineto{\pgfqpoint{9.948979in}{5.017881in}}%
\pgfpathlineto{\pgfqpoint{9.948979in}{5.192881in}}%
\pgfpathlineto{\pgfqpoint{9.884685in}{5.192881in}}%
\pgfpathlineto{\pgfqpoint{9.884685in}{5.017881in}}%
\pgfpathclose%
\pgfusepath{stroke,fill}%
\end{pgfscope}%
\begin{pgfscope}%
\pgfsetrectcap%
\pgfsetmiterjoin%
\pgfsetlinewidth{1.254687pt}%
\definecolor{currentstroke}{rgb}{0.150000,0.150000,0.150000}%
\pgfsetstrokecolor{currentstroke}%
\pgfsetdash{}{0pt}%
\pgfpathmoveto{\pgfqpoint{9.884685in}{0.782881in}}%
\pgfpathlineto{\pgfqpoint{9.884685in}{5.402881in}}%
\pgfusepath{stroke}%
\end{pgfscope}%
\begin{pgfscope}%
\pgfsetrectcap%
\pgfsetmiterjoin%
\pgfsetlinewidth{1.254687pt}%
\definecolor{currentstroke}{rgb}{0.150000,0.150000,0.150000}%
\pgfsetstrokecolor{currentstroke}%
\pgfsetdash{}{0pt}%
\pgfpathmoveto{\pgfqpoint{12.720050in}{0.782881in}}%
\pgfpathlineto{\pgfqpoint{12.720050in}{5.402881in}}%
\pgfusepath{stroke}%
\end{pgfscope}%
\begin{pgfscope}%
\pgfsetrectcap%
\pgfsetmiterjoin%
\pgfsetlinewidth{1.254687pt}%
\definecolor{currentstroke}{rgb}{0.150000,0.150000,0.150000}%
\pgfsetstrokecolor{currentstroke}%
\pgfsetdash{}{0pt}%
\pgfpathmoveto{\pgfqpoint{9.884685in}{0.782881in}}%
\pgfpathlineto{\pgfqpoint{12.720050in}{0.782881in}}%
\pgfusepath{stroke}%
\end{pgfscope}%
\begin{pgfscope}%
\pgfsetrectcap%
\pgfsetmiterjoin%
\pgfsetlinewidth{1.254687pt}%
\definecolor{currentstroke}{rgb}{0.150000,0.150000,0.150000}%
\pgfsetstrokecolor{currentstroke}%
\pgfsetdash{}{0pt}%
\pgfpathmoveto{\pgfqpoint{9.884685in}{5.402881in}}%
\pgfpathlineto{\pgfqpoint{12.720050in}{5.402881in}}%
\pgfusepath{stroke}%
\end{pgfscope}%
\begin{pgfscope}%
\definecolor{textcolor}{rgb}{0.150000,0.150000,0.150000}%
\pgfsetstrokecolor{textcolor}%
\pgfsetfillcolor{textcolor}%
\pgftext[x=10.784801in,y=5.121308in,left,base]{\color{textcolor}\sffamily\fontsize{14.000000}{16.800000}\selectfont Skewness: 0.085836}%
\end{pgfscope}%
\begin{pgfscope}%
\definecolor{textcolor}{rgb}{0.150000,0.150000,0.150000}%
\pgfsetstrokecolor{textcolor}%
\pgfsetfillcolor{textcolor}%
\pgftext[x=10.784801in,y=4.893418in,left,base]{\color{textcolor}\sffamily\fontsize{14.000000}{16.800000}\selectfont Kurtosis: 2.097944}%
\end{pgfscope}%
\end{pgfpicture}%
\makeatother%
\endgroup%
}
    \caption{Distribution des rendements du cours du blé de 2003 à 2022}
\end{figure}
\subsubsection{Le Nickel}
Le second cours matière première choisie dans le cadre de l'analyse est le cours du contrat à terme du nickel côté en Dollars au \textit{London Metal Exchange}. Le 
sous-jacent doit être du nickel d'une pureté d'au minimum 99,80\% et la taille du lot est de 6 tonnes\footcite{lme_nickel}.\\[11pt]
Découvert en 1751, le nickel est un élément chimique métallique qui est représenté par le symbole chimique Ni. Il est fréquemment associé au cobalt dans les dépôts 
miniers, particulièrement apprécié pour ses propriétés physiques et chimiques, en particulier pour sa résistance à la corrosion ou bien sa conductivité électrique/
thermique. \\[11pt]
Aujourd'hui le nickel est utilisé dans un nombre important d'industries, en particulier dans la production  
d'acier, de batteries, de composants électroniques, mais aussi dans la frappe de monnaie, la confection de bijoux ou bien dans la chimie\footcite{about_nickel}.
\begin{figure}[H]
    \centering
    \resizebox{0.8\textwidth}{!}{%% Creator: Matplotlib, PGF backend
%%
%% To include the figure in your LaTeX document, write
%%   \input{<filename>.pgf}
%%
%% Make sure the required packages are loaded in your preamble
%%   \usepackage{pgf}
%%
%% Also ensure that all the required font packages are loaded; for instance,
%% the lmodern package is sometimes necessary when using math font.
%%   \usepackage{lmodern}
%%
%% Figures using additional raster images can only be included by \input if
%% they are in the same directory as the main LaTeX file. For loading figures
%% from other directories you can use the `import` package
%%   \usepackage{import}
%%
%% and then include the figures with
%%   \import{<path to file>}{<filename>.pgf}
%%
%% Matplotlib used the following preamble
%%   \usepackage{fontspec}
%%   \setmainfont{DejaVuSerif.ttf}[Path=\detokenize{C:/Users/Joseph/miniconda3/Lib/site-packages/matplotlib/mpl-data/fonts/ttf/}]
%%   \setsansfont{arial.ttf}[Path=\detokenize{C:/Windows/Fonts/}]
%%   \setmonofont{DejaVuSansMono.ttf}[Path=\detokenize{C:/Users/Joseph/miniconda3/Lib/site-packages/matplotlib/mpl-data/fonts/ttf/}]
%%
\begingroup%
\makeatletter%
\begin{pgfpicture}%
\pgfpathrectangle{\pgfpointorigin}{\pgfqpoint{8.234294in}{4.145023in}}%
\pgfusepath{use as bounding box, clip}%
\begin{pgfscope}%
\pgfsetbuttcap%
\pgfsetmiterjoin%
\definecolor{currentfill}{rgb}{1.000000,1.000000,1.000000}%
\pgfsetfillcolor{currentfill}%
\pgfsetlinewidth{0.000000pt}%
\definecolor{currentstroke}{rgb}{1.000000,1.000000,1.000000}%
\pgfsetstrokecolor{currentstroke}%
\pgfsetdash{}{0pt}%
\pgfpathmoveto{\pgfqpoint{0.000000in}{0.000000in}}%
\pgfpathlineto{\pgfqpoint{8.234294in}{0.000000in}}%
\pgfpathlineto{\pgfqpoint{8.234294in}{4.145023in}}%
\pgfpathlineto{\pgfqpoint{0.000000in}{4.145023in}}%
\pgfpathlineto{\pgfqpoint{0.000000in}{0.000000in}}%
\pgfpathclose%
\pgfusepath{fill}%
\end{pgfscope}%
\begin{pgfscope}%
\pgfsetbuttcap%
\pgfsetmiterjoin%
\definecolor{currentfill}{rgb}{0.031373,0.188235,0.419608}%
\pgfsetfillcolor{currentfill}%
\pgfsetlinewidth{1.003750pt}%
\definecolor{currentstroke}{rgb}{1.000000,1.000000,1.000000}%
\pgfsetstrokecolor{currentstroke}%
\pgfsetdash{}{0pt}%
\pgfpathmoveto{\pgfqpoint{3.565000in}{2.120023in}}%
\pgfpathcurveto{\pgfqpoint{3.565000in}{2.402763in}}{\pgfqpoint{3.487134in}{2.680117in}}{\pgfqpoint{3.339977in}{2.921544in}}%
\pgfpathcurveto{\pgfqpoint{3.192820in}{3.162970in}}{\pgfqpoint{2.981978in}{3.359271in}}{\pgfqpoint{2.730669in}{3.488830in}}%
\pgfpathcurveto{\pgfqpoint{2.479359in}{3.618389in}}{\pgfqpoint{2.197158in}{3.676270in}}{\pgfqpoint{1.915138in}{3.656100in}}%
\pgfpathcurveto{\pgfqpoint{1.633118in}{3.635929in}}{\pgfqpoint{1.362026in}{3.538476in}}{\pgfqpoint{1.131712in}{3.374470in}}%
\pgfpathcurveto{\pgfqpoint{0.901399in}{3.210465in}}{\pgfqpoint{0.720640in}{2.986157in}}{\pgfqpoint{0.609339in}{2.726245in}}%
\pgfpathcurveto{\pgfqpoint{0.498038in}{2.466333in}}{\pgfqpoint{0.460437in}{2.180721in}}{\pgfqpoint{0.500675in}{1.900858in}}%
\pgfpathcurveto{\pgfqpoint{0.540913in}{1.620996in}}{\pgfqpoint{0.657458in}{1.357547in}}{\pgfqpoint{0.837476in}{1.139520in}}%
\pgfpathcurveto{\pgfqpoint{1.017493in}{0.921494in}}{\pgfqpoint{1.254126in}{0.757197in}}{\pgfqpoint{1.521315in}{0.664722in}}%
\pgfpathlineto{\pgfqpoint{2.025000in}{2.120023in}}%
\pgfpathlineto{\pgfqpoint{3.565000in}{2.120023in}}%
\pgfpathlineto{\pgfqpoint{3.565000in}{2.120023in}}%
\pgfpathclose%
\pgfusepath{stroke,fill}%
\end{pgfscope}%
\begin{pgfscope}%
\pgfsetbuttcap%
\pgfsetmiterjoin%
\definecolor{currentfill}{rgb}{0.044060,0.333887,0.624452}%
\pgfsetfillcolor{currentfill}%
\pgfsetlinewidth{1.003750pt}%
\definecolor{currentstroke}{rgb}{1.000000,1.000000,1.000000}%
\pgfsetstrokecolor{currentstroke}%
\pgfsetdash{}{0pt}%
\pgfpathmoveto{\pgfqpoint{1.523759in}{0.510741in}}%
\pgfpathcurveto{\pgfqpoint{1.693519in}{0.451987in}}{\pgfqpoint{1.872259in}{0.423386in}}{\pgfqpoint{2.051877in}{0.426236in}}%
\pgfpathcurveto{\pgfqpoint{2.231495in}{0.429087in}}{\pgfqpoint{2.409237in}{0.463344in}}{\pgfqpoint{2.577048in}{0.527455in}}%
\pgfpathlineto{\pgfqpoint{2.027443in}{1.966043in}}%
\pgfpathlineto{\pgfqpoint{1.523759in}{0.510741in}}%
\pgfpathlineto{\pgfqpoint{1.523759in}{0.510741in}}%
\pgfpathclose%
\pgfusepath{stroke,fill}%
\end{pgfscope}%
\begin{pgfscope}%
\pgfsetbuttcap%
\pgfsetmiterjoin%
\definecolor{currentfill}{rgb}{0.166967,0.480692,0.729150}%
\pgfsetfillcolor{currentfill}%
\pgfsetlinewidth{1.003750pt}%
\definecolor{currentstroke}{rgb}{1.000000,1.000000,1.000000}%
\pgfsetstrokecolor{currentstroke}%
\pgfsetdash{}{0pt}%
\pgfpathmoveto{\pgfqpoint{2.659908in}{0.553220in}}%
\pgfpathcurveto{\pgfqpoint{2.766537in}{0.593957in}}{\pgfqpoint{2.868307in}{0.646423in}}{\pgfqpoint{2.963341in}{0.709650in}}%
\pgfpathcurveto{\pgfqpoint{3.058374in}{0.772877in}}{\pgfqpoint{3.146085in}{0.846475in}}{\pgfqpoint{3.224854in}{0.929086in}}%
\pgfpathlineto{\pgfqpoint{2.110304in}{1.991807in}}%
\pgfpathlineto{\pgfqpoint{2.659908in}{0.553220in}}%
\pgfpathlineto{\pgfqpoint{2.659908in}{0.553220in}}%
\pgfpathclose%
\pgfusepath{stroke,fill}%
\end{pgfscope}%
\begin{pgfscope}%
\pgfsetbuttcap%
\pgfsetmiterjoin%
\definecolor{currentfill}{rgb}{0.326290,0.618624,0.802799}%
\pgfsetfillcolor{currentfill}%
\pgfsetlinewidth{1.003750pt}%
\definecolor{currentstroke}{rgb}{1.000000,1.000000,1.000000}%
\pgfsetstrokecolor{currentstroke}%
\pgfsetdash{}{0pt}%
\pgfpathmoveto{\pgfqpoint{3.398656in}{0.890784in}}%
\pgfpathcurveto{\pgfqpoint{3.466155in}{0.961574in}}{\pgfqpoint{3.526755in}{1.038634in}}{\pgfqpoint{3.579636in}{1.120919in}}%
\pgfpathcurveto{\pgfqpoint{3.632518in}{1.203204in}}{\pgfqpoint{3.677440in}{1.290340in}}{\pgfqpoint{3.713793in}{1.381146in}}%
\pgfpathlineto{\pgfqpoint{2.284106in}{1.953506in}}%
\pgfpathlineto{\pgfqpoint{3.398656in}{0.890784in}}%
\pgfpathlineto{\pgfqpoint{3.398656in}{0.890784in}}%
\pgfpathclose%
\pgfusepath{stroke,fill}%
\end{pgfscope}%
\begin{pgfscope}%
\pgfsetbuttcap%
\pgfsetmiterjoin%
\definecolor{currentfill}{rgb}{0.535686,0.746082,0.864252}%
\pgfsetfillcolor{currentfill}%
\pgfsetlinewidth{1.003750pt}%
\definecolor{currentstroke}{rgb}{1.000000,1.000000,1.000000}%
\pgfsetstrokecolor{currentstroke}%
\pgfsetdash{}{0pt}%
\pgfpathmoveto{\pgfqpoint{3.750210in}{1.460889in}}%
\pgfpathcurveto{\pgfqpoint{3.768377in}{1.506267in}}{\pgfqpoint{3.784372in}{1.552483in}}{\pgfqpoint{3.798143in}{1.599381in}}%
\pgfpathcurveto{\pgfqpoint{3.811914in}{1.646280in}}{\pgfqpoint{3.823444in}{1.693807in}}{\pgfqpoint{3.832694in}{1.741803in}}%
\pgfpathlineto{\pgfqpoint{2.320524in}{2.033250in}}%
\pgfpathlineto{\pgfqpoint{3.750210in}{1.460889in}}%
\pgfpathlineto{\pgfqpoint{3.750210in}{1.460889in}}%
\pgfpathclose%
\pgfusepath{stroke,fill}%
\end{pgfscope}%
\begin{pgfscope}%
\pgfsetbuttcap%
\pgfsetmiterjoin%
\definecolor{currentfill}{rgb}{0.730950,0.839477,0.921323}%
\pgfsetfillcolor{currentfill}%
\pgfsetlinewidth{1.003750pt}%
\definecolor{currentstroke}{rgb}{1.000000,1.000000,1.000000}%
\pgfsetstrokecolor{currentstroke}%
\pgfsetdash{}{0pt}%
\pgfpathmoveto{\pgfqpoint{3.919073in}{1.779838in}}%
\pgfpathcurveto{\pgfqpoint{3.925239in}{1.811832in}}{\pgfqpoint{3.930388in}{1.844013in}}{\pgfqpoint{3.934513in}{1.876333in}}%
\pgfpathcurveto{\pgfqpoint{3.938638in}{1.908653in}}{\pgfqpoint{3.941736in}{1.941096in}}{\pgfqpoint{3.943802in}{1.973612in}}%
\pgfpathlineto{\pgfqpoint{2.406903in}{2.071285in}}%
\pgfpathlineto{\pgfqpoint{3.919073in}{1.779838in}}%
\pgfpathlineto{\pgfqpoint{3.919073in}{1.779838in}}%
\pgfpathclose%
\pgfusepath{stroke,fill}%
\end{pgfscope}%
\begin{pgfscope}%
\pgfsetbuttcap%
\pgfsetmiterjoin%
\definecolor{currentfill}{rgb}{0.858408,0.913449,0.964567}%
\pgfsetfillcolor{currentfill}%
\pgfsetlinewidth{1.003750pt}%
\definecolor{currentstroke}{rgb}{1.000000,1.000000,1.000000}%
\pgfsetstrokecolor{currentstroke}%
\pgfsetdash{}{0pt}%
\pgfpathmoveto{\pgfqpoint{3.946706in}{2.010135in}}%
\pgfpathcurveto{\pgfqpoint{3.947739in}{2.026392in}}{\pgfqpoint{3.948514in}{2.042665in}}{\pgfqpoint{3.949031in}{2.058947in}}%
\pgfpathcurveto{\pgfqpoint{3.949548in}{2.075229in}}{\pgfqpoint{3.949806in}{2.091518in}}{\pgfqpoint{3.949806in}{2.107808in}}%
\pgfpathlineto{\pgfqpoint{2.409806in}{2.107808in}}%
\pgfpathlineto{\pgfqpoint{3.946706in}{2.010135in}}%
\pgfpathlineto{\pgfqpoint{3.946706in}{2.010135in}}%
\pgfpathclose%
\pgfusepath{stroke,fill}%
\end{pgfscope}%
\begin{pgfscope}%
\definecolor{textcolor}{rgb}{0.150000,0.150000,0.150000}%
\pgfsetstrokecolor{textcolor}%
\pgfsetfillcolor{textcolor}%
\pgftext[x=0.997719in,y=3.562637in,,]{\color{textcolor}\sffamily\fontsize{15.000000}{18.000000}\selectfont 69.7\%}%
\end{pgfscope}%
\begin{pgfscope}%
\definecolor{textcolor}{rgb}{0.150000,0.150000,0.150000}%
\pgfsetstrokecolor{textcolor}%
\pgfsetfillcolor{textcolor}%
\pgftext[x=2.055542in,y=0.195266in,,]{\color{textcolor}\sffamily\fontsize{15.000000}{18.000000}\selectfont 11.1\%}%
\end{pgfscope}%
\begin{pgfscope}%
\definecolor{textcolor}{rgb}{0.150000,0.150000,0.150000}%
\pgfsetstrokecolor{textcolor}%
\pgfsetfillcolor{textcolor}%
\pgftext[x=3.091296in,y=0.517326in,,]{\color{textcolor}\sffamily\fontsize{15.000000}{18.000000}\selectfont 7.1\%}%
\end{pgfscope}%
\begin{pgfscope}%
\definecolor{textcolor}{rgb}{0.150000,0.150000,0.150000}%
\pgfsetstrokecolor{textcolor}%
\pgfsetfillcolor{textcolor}%
\pgftext[x=3.773966in,y=0.996031in,,]{\color{textcolor}\sffamily\fontsize{15.000000}{18.000000}\selectfont 6.1\%}%
\end{pgfscope}%
\begin{pgfscope}%
\definecolor{textcolor}{rgb}{0.150000,0.150000,0.150000}%
\pgfsetstrokecolor{textcolor}%
\pgfsetfillcolor{textcolor}%
\pgftext[x=4.019786in,y=1.534301in,,]{\color{textcolor}\sffamily\fontsize{15.000000}{18.000000}\selectfont 3.0\%}%
\end{pgfscope}%
\begin{pgfscope}%
\definecolor{textcolor}{rgb}{0.150000,0.150000,0.150000}%
\pgfsetstrokecolor{textcolor}%
\pgfsetfillcolor{textcolor}%
\pgftext[x=4.163655in,y=1.847090in,,]{\color{textcolor}\sffamily\fontsize{15.000000}{18.000000}\selectfont 2.0\%}%
\end{pgfscope}%
\begin{pgfscope}%
\definecolor{textcolor}{rgb}{0.150000,0.150000,0.150000}%
\pgfsetstrokecolor{textcolor}%
\pgfsetfillcolor{textcolor}%
\pgftext[x=4.179915in,y=2.051618in,,]{\color{textcolor}\sffamily\fontsize{15.000000}{18.000000}\selectfont 1.0\%}%
\end{pgfscope}%
\begin{pgfscope}%
\pgfsetbuttcap%
\pgfsetmiterjoin%
\definecolor{currentfill}{rgb}{1.000000,1.000000,1.000000}%
\pgfsetfillcolor{currentfill}%
\pgfsetfillopacity{0.800000}%
\pgfsetlinewidth{1.003750pt}%
\definecolor{currentstroke}{rgb}{0.800000,0.800000,0.800000}%
\pgfsetstrokecolor{currentstroke}%
\pgfsetstrokeopacity{0.800000}%
\pgfsetdash{}{0pt}%
\pgfpathmoveto{\pgfqpoint{4.904722in}{0.767696in}}%
\pgfpathlineto{\pgfqpoint{8.081517in}{0.767696in}}%
\pgfpathquadraticcurveto{\pgfqpoint{8.134294in}{0.767696in}}{\pgfqpoint{8.134294in}{0.820474in}}%
\pgfpathlineto{\pgfqpoint{8.134294in}{3.419572in}}%
\pgfpathquadraticcurveto{\pgfqpoint{8.134294in}{3.472350in}}{\pgfqpoint{8.081517in}{3.472350in}}%
\pgfpathlineto{\pgfqpoint{4.904722in}{3.472350in}}%
\pgfpathquadraticcurveto{\pgfqpoint{4.851944in}{3.472350in}}{\pgfqpoint{4.851944in}{3.419572in}}%
\pgfpathlineto{\pgfqpoint{4.851944in}{0.820474in}}%
\pgfpathquadraticcurveto{\pgfqpoint{4.851944in}{0.767696in}}{\pgfqpoint{4.904722in}{0.767696in}}%
\pgfpathlineto{\pgfqpoint{4.904722in}{0.767696in}}%
\pgfpathclose%
\pgfusepath{stroke,fill}%
\end{pgfscope}%
\begin{pgfscope}%
\pgfsetbuttcap%
\pgfsetmiterjoin%
\definecolor{currentfill}{rgb}{0.031373,0.188235,0.419608}%
\pgfsetfillcolor{currentfill}%
\pgfsetlinewidth{1.003750pt}%
\definecolor{currentstroke}{rgb}{1.000000,1.000000,1.000000}%
\pgfsetstrokecolor{currentstroke}%
\pgfsetdash{}{0pt}%
\pgfpathmoveto{\pgfqpoint{4.957500in}{3.177897in}}%
\pgfpathlineto{\pgfqpoint{5.485278in}{3.177897in}}%
\pgfpathlineto{\pgfqpoint{5.485278in}{3.362620in}}%
\pgfpathlineto{\pgfqpoint{4.957500in}{3.362620in}}%
\pgfpathlineto{\pgfqpoint{4.957500in}{3.177897in}}%
\pgfpathclose%
\pgfusepath{stroke,fill}%
\end{pgfscope}%
\begin{pgfscope}%
\definecolor{textcolor}{rgb}{0.150000,0.150000,0.150000}%
\pgfsetstrokecolor{textcolor}%
\pgfsetfillcolor{textcolor}%
\pgftext[x=5.696389in,y=3.177897in,left,base]{\color{textcolor}\sffamily\fontsize{19.000000}{22.800000}\selectfont Acier inoxydable}%
\end{pgfscope}%
\begin{pgfscope}%
\pgfsetbuttcap%
\pgfsetmiterjoin%
\definecolor{currentfill}{rgb}{0.044060,0.333887,0.624452}%
\pgfsetfillcolor{currentfill}%
\pgfsetlinewidth{1.003750pt}%
\definecolor{currentstroke}{rgb}{1.000000,1.000000,1.000000}%
\pgfsetstrokecolor{currentstroke}%
\pgfsetdash{}{0pt}%
\pgfpathmoveto{\pgfqpoint{4.957500in}{2.801521in}}%
\pgfpathlineto{\pgfqpoint{5.485278in}{2.801521in}}%
\pgfpathlineto{\pgfqpoint{5.485278in}{2.986243in}}%
\pgfpathlineto{\pgfqpoint{4.957500in}{2.986243in}}%
\pgfpathlineto{\pgfqpoint{4.957500in}{2.801521in}}%
\pgfpathclose%
\pgfusepath{stroke,fill}%
\end{pgfscope}%
\begin{pgfscope}%
\definecolor{textcolor}{rgb}{0.150000,0.150000,0.150000}%
\pgfsetstrokecolor{textcolor}%
\pgfsetfillcolor{textcolor}%
\pgftext[x=5.696389in,y=2.801521in,left,base]{\color{textcolor}\sffamily\fontsize{19.000000}{22.800000}\selectfont Batteries}%
\end{pgfscope}%
\begin{pgfscope}%
\pgfsetbuttcap%
\pgfsetmiterjoin%
\definecolor{currentfill}{rgb}{0.166967,0.480692,0.729150}%
\pgfsetfillcolor{currentfill}%
\pgfsetlinewidth{1.003750pt}%
\definecolor{currentstroke}{rgb}{1.000000,1.000000,1.000000}%
\pgfsetstrokecolor{currentstroke}%
\pgfsetdash{}{0pt}%
\pgfpathmoveto{\pgfqpoint{4.957500in}{2.425016in}}%
\pgfpathlineto{\pgfqpoint{5.485278in}{2.425016in}}%
\pgfpathlineto{\pgfqpoint{5.485278in}{2.609738in}}%
\pgfpathlineto{\pgfqpoint{4.957500in}{2.609738in}}%
\pgfpathlineto{\pgfqpoint{4.957500in}{2.425016in}}%
\pgfpathclose%
\pgfusepath{stroke,fill}%
\end{pgfscope}%
\begin{pgfscope}%
\definecolor{textcolor}{rgb}{0.150000,0.150000,0.150000}%
\pgfsetstrokecolor{textcolor}%
\pgfsetfillcolor{textcolor}%
\pgftext[x=5.696389in,y=2.425016in,left,base]{\color{textcolor}\sffamily\fontsize{19.000000}{22.800000}\selectfont Alliages non ferreux}%
\end{pgfscope}%
\begin{pgfscope}%
\pgfsetbuttcap%
\pgfsetmiterjoin%
\definecolor{currentfill}{rgb}{0.326290,0.618624,0.802799}%
\pgfsetfillcolor{currentfill}%
\pgfsetlinewidth{1.003750pt}%
\definecolor{currentstroke}{rgb}{1.000000,1.000000,1.000000}%
\pgfsetstrokecolor{currentstroke}%
\pgfsetdash{}{0pt}%
\pgfpathmoveto{\pgfqpoint{4.957500in}{2.048639in}}%
\pgfpathlineto{\pgfqpoint{5.485278in}{2.048639in}}%
\pgfpathlineto{\pgfqpoint{5.485278in}{2.233361in}}%
\pgfpathlineto{\pgfqpoint{4.957500in}{2.233361in}}%
\pgfpathlineto{\pgfqpoint{4.957500in}{2.048639in}}%
\pgfpathclose%
\pgfusepath{stroke,fill}%
\end{pgfscope}%
\begin{pgfscope}%
\definecolor{textcolor}{rgb}{0.150000,0.150000,0.150000}%
\pgfsetstrokecolor{textcolor}%
\pgfsetfillcolor{textcolor}%
\pgftext[x=5.696389in,y=2.048639in,left,base]{\color{textcolor}\sffamily\fontsize{19.000000}{22.800000}\selectfont Placage}%
\end{pgfscope}%
\begin{pgfscope}%
\pgfsetbuttcap%
\pgfsetmiterjoin%
\definecolor{currentfill}{rgb}{0.535686,0.746082,0.864252}%
\pgfsetfillcolor{currentfill}%
\pgfsetlinewidth{1.003750pt}%
\definecolor{currentstroke}{rgb}{1.000000,1.000000,1.000000}%
\pgfsetstrokecolor{currentstroke}%
\pgfsetdash{}{0pt}%
\pgfpathmoveto{\pgfqpoint{4.957500in}{1.672263in}}%
\pgfpathlineto{\pgfqpoint{5.485278in}{1.672263in}}%
\pgfpathlineto{\pgfqpoint{5.485278in}{1.856985in}}%
\pgfpathlineto{\pgfqpoint{4.957500in}{1.856985in}}%
\pgfpathlineto{\pgfqpoint{4.957500in}{1.672263in}}%
\pgfpathclose%
\pgfusepath{stroke,fill}%
\end{pgfscope}%
\begin{pgfscope}%
\definecolor{textcolor}{rgb}{0.150000,0.150000,0.150000}%
\pgfsetstrokecolor{textcolor}%
\pgfsetfillcolor{textcolor}%
\pgftext[x=5.696389in,y=1.672263in,left,base]{\color{textcolor}\sffamily\fontsize{19.000000}{22.800000}\selectfont Aciers alliés}%
\end{pgfscope}%
\begin{pgfscope}%
\pgfsetbuttcap%
\pgfsetmiterjoin%
\definecolor{currentfill}{rgb}{0.730950,0.839477,0.921323}%
\pgfsetfillcolor{currentfill}%
\pgfsetlinewidth{1.003750pt}%
\definecolor{currentstroke}{rgb}{1.000000,1.000000,1.000000}%
\pgfsetstrokecolor{currentstroke}%
\pgfsetdash{}{0pt}%
\pgfpathmoveto{\pgfqpoint{4.957500in}{1.298979in}}%
\pgfpathlineto{\pgfqpoint{5.485278in}{1.298979in}}%
\pgfpathlineto{\pgfqpoint{5.485278in}{1.483701in}}%
\pgfpathlineto{\pgfqpoint{4.957500in}{1.483701in}}%
\pgfpathlineto{\pgfqpoint{4.957500in}{1.298979in}}%
\pgfpathclose%
\pgfusepath{stroke,fill}%
\end{pgfscope}%
\begin{pgfscope}%
\definecolor{textcolor}{rgb}{0.150000,0.150000,0.150000}%
\pgfsetstrokecolor{textcolor}%
\pgfsetfillcolor{textcolor}%
\pgftext[x=5.696389in,y=1.298979in,left,base]{\color{textcolor}\sffamily\fontsize{19.000000}{22.800000}\selectfont Fonderies}%
\end{pgfscope}%
\begin{pgfscope}%
\pgfsetbuttcap%
\pgfsetmiterjoin%
\definecolor{currentfill}{rgb}{0.858408,0.913449,0.964567}%
\pgfsetfillcolor{currentfill}%
\pgfsetlinewidth{1.003750pt}%
\definecolor{currentstroke}{rgb}{1.000000,1.000000,1.000000}%
\pgfsetstrokecolor{currentstroke}%
\pgfsetdash{}{0pt}%
\pgfpathmoveto{\pgfqpoint{4.957500in}{0.925695in}}%
\pgfpathlineto{\pgfqpoint{5.485278in}{0.925695in}}%
\pgfpathlineto{\pgfqpoint{5.485278in}{1.110417in}}%
\pgfpathlineto{\pgfqpoint{4.957500in}{1.110417in}}%
\pgfpathlineto{\pgfqpoint{4.957500in}{0.925695in}}%
\pgfpathclose%
\pgfusepath{stroke,fill}%
\end{pgfscope}%
\begin{pgfscope}%
\definecolor{textcolor}{rgb}{0.150000,0.150000,0.150000}%
\pgfsetstrokecolor{textcolor}%
\pgfsetfillcolor{textcolor}%
\pgftext[x=5.696389in,y=0.925695in,left,base]{\color{textcolor}\sffamily\fontsize{19.000000}{22.800000}\selectfont Autres}%
\end{pgfscope}%
\end{pgfpicture}%
\makeatother%
\endgroup%
}
    \caption{Principales utilisations du nickel}
\end{figure}
Il est important de souligner qu'il existe un risque potentiel de conflits d'utilisation à long terme, notamment en raison de l'augmentation considérable de l'utilisation 
du nickel dans les technologies liées à la transition énergétique, en particulier dans la production de batteries. Cette utilisation accrue pourrait épuiser les ressources 
en nickel et engendrer des conflits géopolitiques.\\[11pt]
Concernant l'occurrence du nickel, il est relativement répandu sur le globe, les ressources terrestres étant estimées à 300 millions de tonnes avec 60\% correspondent à 
des dépôts de latérite (roche rouge) et 40\% à des gisements de sulfure. Quand aux réserves, elles sont estimées a 100 millions de tonnes et se 
situent principalement en Indonésie, en Australie, au Brésil, et en Russie. Enfin, concernant la production de nickel dans le monde, la production en 2022 de nickel était
de 3,3 millions de tones avec quasiment la moitié produite par l'Indonésie\footcite{info_nickel}.\\[11pt]
Comme pour le cours du blé, des statistiques sont faite sur une longue période (de 2006 à 2022) pour le cours du contrat sur le nickel.
\begin{table}[H]
    \centering
    \caption{Statistiques descriptives sur le cours du nickel de 2006 à 2022}
    \sffamily
    \resizebox{\textwidth}{!}{\begin{tabular}{ccccccc}
    \toprule
    Moyenne & Écart-Type & Minimum & Maximum & Médiane & Q1 & Q3 \\
    \midrule
    18 161,44 \$ & 7 636,12 \$ & 8 435,00 \$ & 49 675,00 \$ & 16 332,50 \$ & 12 742,50 \$ & 21 219,75 \$ \\
    \bottomrule
\end{tabular}}
\end{table}
En moyenne le cours du contrat de nickel sur les 17 ans est de 18 161,44 \$. Son plus bas prix date de mai 2016 tandis que le maximum qu'ait atteint le cours date de 
avril 2007, l'étendue sur la période est de 41 240\$.
\begin{table}[H]
    \centering
    \caption{Statistiques sur les rendements du cours du nickel de 2006 à 2022}
    \sffamily
    \resizebox{\textwidth}{!}{\begin{tabular}{ccccccc}
    \toprule
    Moyenne & Écart-Type & Minimum & Maximum & Skewness & Kurtosis \\
    \midrule
    0,33 \% & 10,03\% & -26,93\%  & 27,93\% & -0,23 & 0,17 \\
    \bottomrule
\end{tabular}

}
\end{table}
Ici la volatilité mensuelle du contrat est modérée à 10\%. Le skewness de -0,23 suggère que la distribution est légèrement asymétrique négative, les rendements sont donc plus souvent négatifs cependant le skewness est quand meme proche de 0, et le kurtosis inférieur à 3. Ces résultats peuvent laisser à penser que les rendements sont 
distribués de manière normale.
\begin{figure}[H]
    \centering
    \resizebox{\textwidth}{!}{%% Creator: Matplotlib, PGF backend
%%
%% To include the figure in your LaTeX document, write
%%   \input{<filename>.pgf}
%%
%% Make sure the required packages are loaded in your preamble
%%   \usepackage{pgf}
%%
%% Also ensure that all the required font packages are loaded; for instance,
%% the lmodern package is sometimes necessary when using math font.
%%   \usepackage{lmodern}
%%
%% Figures using additional raster images can only be included by \input if
%% they are in the same directory as the main LaTeX file. For loading figures
%% from other directories you can use the `import` package
%%   \usepackage{import}
%%
%% and then include the figures with
%%   \import{<path to file>}{<filename>.pgf}
%%
%% Matplotlib used the following preamble
%%   \usepackage{fontspec}
%%   \setmainfont{DejaVuSerif.ttf}[Path=\detokenize{C:/Users/Joseph/miniconda3/Lib/site-packages/matplotlib/mpl-data/fonts/ttf/}]
%%   \setsansfont{arial.ttf}[Path=\detokenize{C:/Windows/Fonts/}]
%%   \setmonofont{DejaVuSansMono.ttf}[Path=\detokenize{C:/Users/Joseph/miniconda3/Lib/site-packages/matplotlib/mpl-data/fonts/ttf/}]
%%
\begingroup%
\makeatletter%
\begin{pgfpicture}%
\pgfpathrectangle{\pgfpointorigin}{\pgfqpoint{12.831795in}{5.545477in}}%
\pgfusepath{use as bounding box, clip}%
\begin{pgfscope}%
\pgfsetbuttcap%
\pgfsetmiterjoin%
\definecolor{currentfill}{rgb}{1.000000,1.000000,1.000000}%
\pgfsetfillcolor{currentfill}%
\pgfsetlinewidth{0.000000pt}%
\definecolor{currentstroke}{rgb}{1.000000,1.000000,1.000000}%
\pgfsetstrokecolor{currentstroke}%
\pgfsetdash{}{0pt}%
\pgfpathmoveto{\pgfqpoint{0.000000in}{0.000000in}}%
\pgfpathlineto{\pgfqpoint{12.831795in}{0.000000in}}%
\pgfpathlineto{\pgfqpoint{12.831795in}{5.545477in}}%
\pgfpathlineto{\pgfqpoint{0.000000in}{5.545477in}}%
\pgfpathlineto{\pgfqpoint{0.000000in}{0.000000in}}%
\pgfpathclose%
\pgfusepath{fill}%
\end{pgfscope}%
\begin{pgfscope}%
\pgfsetbuttcap%
\pgfsetmiterjoin%
\definecolor{currentfill}{rgb}{1.000000,1.000000,1.000000}%
\pgfsetfillcolor{currentfill}%
\pgfsetlinewidth{0.000000pt}%
\definecolor{currentstroke}{rgb}{0.000000,0.000000,0.000000}%
\pgfsetstrokecolor{currentstroke}%
\pgfsetstrokeopacity{0.000000}%
\pgfsetdash{}{0pt}%
\pgfpathmoveto{\pgfqpoint{1.095050in}{0.782881in}}%
\pgfpathlineto{\pgfqpoint{9.601148in}{0.782881in}}%
\pgfpathlineto{\pgfqpoint{9.601148in}{5.402881in}}%
\pgfpathlineto{\pgfqpoint{1.095050in}{5.402881in}}%
\pgfpathlineto{\pgfqpoint{1.095050in}{0.782881in}}%
\pgfpathclose%
\pgfusepath{fill}%
\end{pgfscope}%
\begin{pgfscope}%
\pgfpathrectangle{\pgfqpoint{1.095050in}{0.782881in}}{\pgfqpoint{8.506098in}{4.620000in}}%
\pgfusepath{clip}%
\pgfsetroundcap%
\pgfsetroundjoin%
\pgfsetlinewidth{1.003750pt}%
\definecolor{currentstroke}{rgb}{0.800000,0.800000,0.800000}%
\pgfsetstrokecolor{currentstroke}%
\pgfsetstrokeopacity{0.000000}%
\pgfsetdash{}{0pt}%
\pgfpathmoveto{\pgfqpoint{1.442694in}{0.782881in}}%
\pgfpathlineto{\pgfqpoint{1.442694in}{5.402881in}}%
\pgfusepath{stroke}%
\end{pgfscope}%
\begin{pgfscope}%
\definecolor{textcolor}{rgb}{0.150000,0.150000,0.150000}%
\pgfsetstrokecolor{textcolor}%
\pgfsetfillcolor{textcolor}%
\pgftext[x=1.442694in,y=0.650937in,,top]{\color{textcolor}\sffamily\fontsize{19.000000}{22.800000}\selectfont 2006}%
\end{pgfscope}%
\begin{pgfscope}%
\pgfpathrectangle{\pgfqpoint{1.095050in}{0.782881in}}{\pgfqpoint{8.506098in}{4.620000in}}%
\pgfusepath{clip}%
\pgfsetroundcap%
\pgfsetroundjoin%
\pgfsetlinewidth{1.003750pt}%
\definecolor{currentstroke}{rgb}{0.800000,0.800000,0.800000}%
\pgfsetstrokecolor{currentstroke}%
\pgfsetstrokeopacity{0.000000}%
\pgfsetdash{}{0pt}%
\pgfpathmoveto{\pgfqpoint{2.361021in}{0.782881in}}%
\pgfpathlineto{\pgfqpoint{2.361021in}{5.402881in}}%
\pgfusepath{stroke}%
\end{pgfscope}%
\begin{pgfscope}%
\definecolor{textcolor}{rgb}{0.150000,0.150000,0.150000}%
\pgfsetstrokecolor{textcolor}%
\pgfsetfillcolor{textcolor}%
\pgftext[x=2.361021in,y=0.650937in,,top]{\color{textcolor}\sffamily\fontsize{19.000000}{22.800000}\selectfont 2008}%
\end{pgfscope}%
\begin{pgfscope}%
\pgfpathrectangle{\pgfqpoint{1.095050in}{0.782881in}}{\pgfqpoint{8.506098in}{4.620000in}}%
\pgfusepath{clip}%
\pgfsetroundcap%
\pgfsetroundjoin%
\pgfsetlinewidth{1.003750pt}%
\definecolor{currentstroke}{rgb}{0.800000,0.800000,0.800000}%
\pgfsetstrokecolor{currentstroke}%
\pgfsetstrokeopacity{0.000000}%
\pgfsetdash{}{0pt}%
\pgfpathmoveto{\pgfqpoint{3.280606in}{0.782881in}}%
\pgfpathlineto{\pgfqpoint{3.280606in}{5.402881in}}%
\pgfusepath{stroke}%
\end{pgfscope}%
\begin{pgfscope}%
\definecolor{textcolor}{rgb}{0.150000,0.150000,0.150000}%
\pgfsetstrokecolor{textcolor}%
\pgfsetfillcolor{textcolor}%
\pgftext[x=3.280606in,y=0.650937in,,top]{\color{textcolor}\sffamily\fontsize{19.000000}{22.800000}\selectfont 2010}%
\end{pgfscope}%
\begin{pgfscope}%
\pgfpathrectangle{\pgfqpoint{1.095050in}{0.782881in}}{\pgfqpoint{8.506098in}{4.620000in}}%
\pgfusepath{clip}%
\pgfsetroundcap%
\pgfsetroundjoin%
\pgfsetlinewidth{1.003750pt}%
\definecolor{currentstroke}{rgb}{0.800000,0.800000,0.800000}%
\pgfsetstrokecolor{currentstroke}%
\pgfsetstrokeopacity{0.000000}%
\pgfsetdash{}{0pt}%
\pgfpathmoveto{\pgfqpoint{4.198933in}{0.782881in}}%
\pgfpathlineto{\pgfqpoint{4.198933in}{5.402881in}}%
\pgfusepath{stroke}%
\end{pgfscope}%
\begin{pgfscope}%
\definecolor{textcolor}{rgb}{0.150000,0.150000,0.150000}%
\pgfsetstrokecolor{textcolor}%
\pgfsetfillcolor{textcolor}%
\pgftext[x=4.198933in,y=0.650937in,,top]{\color{textcolor}\sffamily\fontsize{19.000000}{22.800000}\selectfont 2012}%
\end{pgfscope}%
\begin{pgfscope}%
\pgfpathrectangle{\pgfqpoint{1.095050in}{0.782881in}}{\pgfqpoint{8.506098in}{4.620000in}}%
\pgfusepath{clip}%
\pgfsetroundcap%
\pgfsetroundjoin%
\pgfsetlinewidth{1.003750pt}%
\definecolor{currentstroke}{rgb}{0.800000,0.800000,0.800000}%
\pgfsetstrokecolor{currentstroke}%
\pgfsetstrokeopacity{0.000000}%
\pgfsetdash{}{0pt}%
\pgfpathmoveto{\pgfqpoint{5.118517in}{0.782881in}}%
\pgfpathlineto{\pgfqpoint{5.118517in}{5.402881in}}%
\pgfusepath{stroke}%
\end{pgfscope}%
\begin{pgfscope}%
\definecolor{textcolor}{rgb}{0.150000,0.150000,0.150000}%
\pgfsetstrokecolor{textcolor}%
\pgfsetfillcolor{textcolor}%
\pgftext[x=5.118517in,y=0.650937in,,top]{\color{textcolor}\sffamily\fontsize{19.000000}{22.800000}\selectfont 2014}%
\end{pgfscope}%
\begin{pgfscope}%
\pgfpathrectangle{\pgfqpoint{1.095050in}{0.782881in}}{\pgfqpoint{8.506098in}{4.620000in}}%
\pgfusepath{clip}%
\pgfsetroundcap%
\pgfsetroundjoin%
\pgfsetlinewidth{1.003750pt}%
\definecolor{currentstroke}{rgb}{0.800000,0.800000,0.800000}%
\pgfsetstrokecolor{currentstroke}%
\pgfsetstrokeopacity{0.000000}%
\pgfsetdash{}{0pt}%
\pgfpathmoveto{\pgfqpoint{6.036844in}{0.782881in}}%
\pgfpathlineto{\pgfqpoint{6.036844in}{5.402881in}}%
\pgfusepath{stroke}%
\end{pgfscope}%
\begin{pgfscope}%
\definecolor{textcolor}{rgb}{0.150000,0.150000,0.150000}%
\pgfsetstrokecolor{textcolor}%
\pgfsetfillcolor{textcolor}%
\pgftext[x=6.036844in,y=0.650937in,,top]{\color{textcolor}\sffamily\fontsize{19.000000}{22.800000}\selectfont 2016}%
\end{pgfscope}%
\begin{pgfscope}%
\pgfpathrectangle{\pgfqpoint{1.095050in}{0.782881in}}{\pgfqpoint{8.506098in}{4.620000in}}%
\pgfusepath{clip}%
\pgfsetroundcap%
\pgfsetroundjoin%
\pgfsetlinewidth{1.003750pt}%
\definecolor{currentstroke}{rgb}{0.800000,0.800000,0.800000}%
\pgfsetstrokecolor{currentstroke}%
\pgfsetstrokeopacity{0.000000}%
\pgfsetdash{}{0pt}%
\pgfpathmoveto{\pgfqpoint{6.956429in}{0.782881in}}%
\pgfpathlineto{\pgfqpoint{6.956429in}{5.402881in}}%
\pgfusepath{stroke}%
\end{pgfscope}%
\begin{pgfscope}%
\definecolor{textcolor}{rgb}{0.150000,0.150000,0.150000}%
\pgfsetstrokecolor{textcolor}%
\pgfsetfillcolor{textcolor}%
\pgftext[x=6.956429in,y=0.650937in,,top]{\color{textcolor}\sffamily\fontsize{19.000000}{22.800000}\selectfont 2018}%
\end{pgfscope}%
\begin{pgfscope}%
\pgfpathrectangle{\pgfqpoint{1.095050in}{0.782881in}}{\pgfqpoint{8.506098in}{4.620000in}}%
\pgfusepath{clip}%
\pgfsetroundcap%
\pgfsetroundjoin%
\pgfsetlinewidth{1.003750pt}%
\definecolor{currentstroke}{rgb}{0.800000,0.800000,0.800000}%
\pgfsetstrokecolor{currentstroke}%
\pgfsetstrokeopacity{0.000000}%
\pgfsetdash{}{0pt}%
\pgfpathmoveto{\pgfqpoint{7.874756in}{0.782881in}}%
\pgfpathlineto{\pgfqpoint{7.874756in}{5.402881in}}%
\pgfusepath{stroke}%
\end{pgfscope}%
\begin{pgfscope}%
\definecolor{textcolor}{rgb}{0.150000,0.150000,0.150000}%
\pgfsetstrokecolor{textcolor}%
\pgfsetfillcolor{textcolor}%
\pgftext[x=7.874756in,y=0.650937in,,top]{\color{textcolor}\sffamily\fontsize{19.000000}{22.800000}\selectfont 2020}%
\end{pgfscope}%
\begin{pgfscope}%
\pgfpathrectangle{\pgfqpoint{1.095050in}{0.782881in}}{\pgfqpoint{8.506098in}{4.620000in}}%
\pgfusepath{clip}%
\pgfsetroundcap%
\pgfsetroundjoin%
\pgfsetlinewidth{1.003750pt}%
\definecolor{currentstroke}{rgb}{0.800000,0.800000,0.800000}%
\pgfsetstrokecolor{currentstroke}%
\pgfsetstrokeopacity{0.000000}%
\pgfsetdash{}{0pt}%
\pgfpathmoveto{\pgfqpoint{8.794341in}{0.782881in}}%
\pgfpathlineto{\pgfqpoint{8.794341in}{5.402881in}}%
\pgfusepath{stroke}%
\end{pgfscope}%
\begin{pgfscope}%
\definecolor{textcolor}{rgb}{0.150000,0.150000,0.150000}%
\pgfsetstrokecolor{textcolor}%
\pgfsetfillcolor{textcolor}%
\pgftext[x=8.794341in,y=0.650937in,,top]{\color{textcolor}\sffamily\fontsize{19.000000}{22.800000}\selectfont 2022}%
\end{pgfscope}%
\begin{pgfscope}%
\definecolor{textcolor}{rgb}{0.150000,0.150000,0.150000}%
\pgfsetstrokecolor{textcolor}%
\pgfsetfillcolor{textcolor}%
\pgftext[x=5.348099in,y=0.354042in,,top]{\color{textcolor}\sffamily\fontsize{20.000000}{24.000000}\selectfont Date}%
\end{pgfscope}%
\begin{pgfscope}%
\pgfpathrectangle{\pgfqpoint{1.095050in}{0.782881in}}{\pgfqpoint{8.506098in}{4.620000in}}%
\pgfusepath{clip}%
\pgfsetroundcap%
\pgfsetroundjoin%
\pgfsetlinewidth{1.003750pt}%
\definecolor{currentstroke}{rgb}{0.800000,0.800000,0.800000}%
\pgfsetstrokecolor{currentstroke}%
\pgfsetstrokeopacity{0.400000}%
\pgfsetdash{}{0pt}%
\pgfpathmoveto{\pgfqpoint{1.095050in}{1.523894in}}%
\pgfpathlineto{\pgfqpoint{9.601148in}{1.523894in}}%
\pgfusepath{stroke}%
\end{pgfscope}%
\begin{pgfscope}%
\definecolor{textcolor}{rgb}{0.150000,0.150000,0.150000}%
\pgfsetstrokecolor{textcolor}%
\pgfsetfillcolor{textcolor}%
\pgftext[x=0.409597in, y=1.429446in, left, base]{\color{textcolor}\sffamily\fontsize{19.000000}{22.800000}\selectfont \ensuremath{-}0.2}%
\end{pgfscope}%
\begin{pgfscope}%
\pgfpathrectangle{\pgfqpoint{1.095050in}{0.782881in}}{\pgfqpoint{8.506098in}{4.620000in}}%
\pgfusepath{clip}%
\pgfsetroundcap%
\pgfsetroundjoin%
\pgfsetlinewidth{1.003750pt}%
\definecolor{currentstroke}{rgb}{0.800000,0.800000,0.800000}%
\pgfsetstrokecolor{currentstroke}%
\pgfsetstrokeopacity{0.400000}%
\pgfsetdash{}{0pt}%
\pgfpathmoveto{\pgfqpoint{1.095050in}{2.289321in}}%
\pgfpathlineto{\pgfqpoint{9.601148in}{2.289321in}}%
\pgfusepath{stroke}%
\end{pgfscope}%
\begin{pgfscope}%
\definecolor{textcolor}{rgb}{0.150000,0.150000,0.150000}%
\pgfsetstrokecolor{textcolor}%
\pgfsetfillcolor{textcolor}%
\pgftext[x=0.409597in, y=2.194872in, left, base]{\color{textcolor}\sffamily\fontsize{19.000000}{22.800000}\selectfont \ensuremath{-}0.1}%
\end{pgfscope}%
\begin{pgfscope}%
\pgfpathrectangle{\pgfqpoint{1.095050in}{0.782881in}}{\pgfqpoint{8.506098in}{4.620000in}}%
\pgfusepath{clip}%
\pgfsetroundcap%
\pgfsetroundjoin%
\pgfsetlinewidth{1.003750pt}%
\definecolor{currentstroke}{rgb}{0.800000,0.800000,0.800000}%
\pgfsetstrokecolor{currentstroke}%
\pgfsetstrokeopacity{0.400000}%
\pgfsetdash{}{0pt}%
\pgfpathmoveto{\pgfqpoint{1.095050in}{3.054748in}}%
\pgfpathlineto{\pgfqpoint{9.601148in}{3.054748in}}%
\pgfusepath{stroke}%
\end{pgfscope}%
\begin{pgfscope}%
\definecolor{textcolor}{rgb}{0.150000,0.150000,0.150000}%
\pgfsetstrokecolor{textcolor}%
\pgfsetfillcolor{textcolor}%
\pgftext[x=0.596264in, y=2.960299in, left, base]{\color{textcolor}\sffamily\fontsize{19.000000}{22.800000}\selectfont 0.0}%
\end{pgfscope}%
\begin{pgfscope}%
\pgfpathrectangle{\pgfqpoint{1.095050in}{0.782881in}}{\pgfqpoint{8.506098in}{4.620000in}}%
\pgfusepath{clip}%
\pgfsetroundcap%
\pgfsetroundjoin%
\pgfsetlinewidth{1.003750pt}%
\definecolor{currentstroke}{rgb}{0.800000,0.800000,0.800000}%
\pgfsetstrokecolor{currentstroke}%
\pgfsetstrokeopacity{0.400000}%
\pgfsetdash{}{0pt}%
\pgfpathmoveto{\pgfqpoint{1.095050in}{3.820174in}}%
\pgfpathlineto{\pgfqpoint{9.601148in}{3.820174in}}%
\pgfusepath{stroke}%
\end{pgfscope}%
\begin{pgfscope}%
\definecolor{textcolor}{rgb}{0.150000,0.150000,0.150000}%
\pgfsetstrokecolor{textcolor}%
\pgfsetfillcolor{textcolor}%
\pgftext[x=0.596264in, y=3.725726in, left, base]{\color{textcolor}\sffamily\fontsize{19.000000}{22.800000}\selectfont 0.1}%
\end{pgfscope}%
\begin{pgfscope}%
\pgfpathrectangle{\pgfqpoint{1.095050in}{0.782881in}}{\pgfqpoint{8.506098in}{4.620000in}}%
\pgfusepath{clip}%
\pgfsetroundcap%
\pgfsetroundjoin%
\pgfsetlinewidth{1.003750pt}%
\definecolor{currentstroke}{rgb}{0.800000,0.800000,0.800000}%
\pgfsetstrokecolor{currentstroke}%
\pgfsetstrokeopacity{0.400000}%
\pgfsetdash{}{0pt}%
\pgfpathmoveto{\pgfqpoint{1.095050in}{4.585601in}}%
\pgfpathlineto{\pgfqpoint{9.601148in}{4.585601in}}%
\pgfusepath{stroke}%
\end{pgfscope}%
\begin{pgfscope}%
\definecolor{textcolor}{rgb}{0.150000,0.150000,0.150000}%
\pgfsetstrokecolor{textcolor}%
\pgfsetfillcolor{textcolor}%
\pgftext[x=0.596264in, y=4.491153in, left, base]{\color{textcolor}\sffamily\fontsize{19.000000}{22.800000}\selectfont 0.2}%
\end{pgfscope}%
\begin{pgfscope}%
\pgfpathrectangle{\pgfqpoint{1.095050in}{0.782881in}}{\pgfqpoint{8.506098in}{4.620000in}}%
\pgfusepath{clip}%
\pgfsetroundcap%
\pgfsetroundjoin%
\pgfsetlinewidth{1.003750pt}%
\definecolor{currentstroke}{rgb}{0.800000,0.800000,0.800000}%
\pgfsetstrokecolor{currentstroke}%
\pgfsetstrokeopacity{0.400000}%
\pgfsetdash{}{0pt}%
\pgfpathmoveto{\pgfqpoint{1.095050in}{5.351028in}}%
\pgfpathlineto{\pgfqpoint{9.601148in}{5.351028in}}%
\pgfusepath{stroke}%
\end{pgfscope}%
\begin{pgfscope}%
\definecolor{textcolor}{rgb}{0.150000,0.150000,0.150000}%
\pgfsetstrokecolor{textcolor}%
\pgfsetfillcolor{textcolor}%
\pgftext[x=0.596264in, y=5.256580in, left, base]{\color{textcolor}\sffamily\fontsize{19.000000}{22.800000}\selectfont 0.3}%
\end{pgfscope}%
\begin{pgfscope}%
\definecolor{textcolor}{rgb}{0.150000,0.150000,0.150000}%
\pgfsetstrokecolor{textcolor}%
\pgfsetfillcolor{textcolor}%
\pgftext[x=0.354042in,y=3.092881in,,bottom,rotate=90.000000]{\color{textcolor}\sffamily\fontsize{20.000000}{24.000000}\selectfont Rendements mensuels}%
\end{pgfscope}%
\begin{pgfscope}%
\pgfpathrectangle{\pgfqpoint{1.095050in}{0.782881in}}{\pgfqpoint{8.506098in}{4.620000in}}%
\pgfusepath{clip}%
\pgfsetroundcap%
\pgfsetroundjoin%
\pgfsetlinewidth{1.505625pt}%
\definecolor{currentstroke}{rgb}{0.417086,0.680631,0.838231}%
\pgfsetstrokecolor{currentstroke}%
\pgfsetdash{}{0pt}%
\pgfpathmoveto{\pgfqpoint{1.481691in}{2.920862in}}%
\pgfpathlineto{\pgfqpoint{1.516915in}{3.227036in}}%
\pgfpathlineto{\pgfqpoint{1.555912in}{4.927298in}}%
\pgfpathlineto{\pgfqpoint{1.593652in}{4.210094in}}%
\pgfpathlineto{\pgfqpoint{1.632649in}{2.907414in}}%
\pgfpathlineto{\pgfqpoint{1.670389in}{4.543413in}}%
\pgfpathlineto{\pgfqpoint{1.709386in}{4.221424in}}%
\pgfpathlineto{\pgfqpoint{1.748383in}{2.981030in}}%
\pgfpathlineto{\pgfqpoint{1.786123in}{3.386938in}}%
\pgfpathlineto{\pgfqpoint{1.825120in}{3.552436in}}%
\pgfpathlineto{\pgfqpoint{1.862860in}{2.937545in}}%
\pgfpathlineto{\pgfqpoint{1.901857in}{4.104201in}}%
\pgfpathlineto{\pgfqpoint{1.940855in}{3.937416in}}%
\pgfpathlineto{\pgfqpoint{1.976078in}{3.554530in}}%
\pgfpathlineto{\pgfqpoint{2.015076in}{3.519266in}}%
\pgfpathlineto{\pgfqpoint{2.052815in}{2.703989in}}%
\pgfpathlineto{\pgfqpoint{2.091813in}{1.009754in}}%
\pgfpathlineto{\pgfqpoint{2.129552in}{1.972372in}}%
\pgfpathlineto{\pgfqpoint{2.168549in}{2.570051in}}%
\pgfpathlineto{\pgfqpoint{2.207547in}{3.234917in}}%
\pgfpathlineto{\pgfqpoint{2.245286in}{3.417308in}}%
\pgfpathlineto{\pgfqpoint{2.284284in}{1.757105in}}%
\pgfpathlineto{\pgfqpoint{2.322023in}{2.834716in}}%
\pgfpathlineto{\pgfqpoint{2.361021in}{3.388094in}}%
\pgfpathlineto{\pgfqpoint{2.400018in}{4.139668in}}%
\pgfpathlineto{\pgfqpoint{2.436500in}{2.603664in}}%
\pgfpathlineto{\pgfqpoint{2.475497in}{2.752471in}}%
\pgfpathlineto{\pgfqpoint{2.513237in}{1.092825in}}%
\pgfpathlineto{\pgfqpoint{2.552234in}{2.994314in}}%
\pgfpathlineto{\pgfqpoint{2.589973in}{1.693851in}}%
\pgfpathlineto{\pgfqpoint{2.628971in}{3.821511in}}%
\pgfpathlineto{\pgfqpoint{2.667968in}{1.108073in}}%
\pgfpathlineto{\pgfqpoint{2.705708in}{0.992881in}}%
\pgfpathlineto{\pgfqpoint{2.744705in}{1.767981in}}%
\pgfpathlineto{\pgfqpoint{2.782445in}{2.854570in}}%
\pgfpathlineto{\pgfqpoint{2.821442in}{4.105918in}}%
\pgfpathlineto{\pgfqpoint{2.860440in}{2.061638in}}%
\pgfpathlineto{\pgfqpoint{2.895663in}{2.957734in}}%
\pgfpathlineto{\pgfqpoint{2.934661in}{4.365201in}}%
\pgfpathlineto{\pgfqpoint{2.972400in}{4.308577in}}%
\pgfpathlineto{\pgfqpoint{3.011397in}{3.955546in}}%
\pgfpathlineto{\pgfqpoint{3.049137in}{4.179328in}}%
\pgfpathlineto{\pgfqpoint{3.088134in}{3.580836in}}%
\pgfpathlineto{\pgfqpoint{3.127132in}{2.507188in}}%
\pgfpathlineto{\pgfqpoint{3.164871in}{3.224850in}}%
\pgfpathlineto{\pgfqpoint{3.203869in}{2.236599in}}%
\pgfpathlineto{\pgfqpoint{3.241608in}{4.140683in}}%
\pgfpathlineto{\pgfqpoint{3.280606in}{2.953365in}}%
\pgfpathlineto{\pgfqpoint{3.319603in}{4.040326in}}%
\pgfpathlineto{\pgfqpoint{3.354827in}{4.347938in}}%
\pgfpathlineto{\pgfqpoint{3.393824in}{3.471005in}}%
\pgfpathlineto{\pgfqpoint{3.431563in}{1.306853in}}%
\pgfpathlineto{\pgfqpoint{3.470561in}{2.527956in}}%
\pgfpathlineto{\pgfqpoint{3.508300in}{3.581539in}}%
\pgfpathlineto{\pgfqpoint{3.547298in}{2.931954in}}%
\pgfpathlineto{\pgfqpoint{3.586295in}{3.961421in}}%
\pgfpathlineto{\pgfqpoint{3.624035in}{2.952315in}}%
\pgfpathlineto{\pgfqpoint{3.663032in}{2.999663in}}%
\pgfpathlineto{\pgfqpoint{3.700772in}{3.729407in}}%
\pgfpathlineto{\pgfqpoint{3.739769in}{3.743730in}}%
\pgfpathlineto{\pgfqpoint{3.778767in}{3.510534in}}%
\pgfpathlineto{\pgfqpoint{3.813990in}{2.265149in}}%
\pgfpathlineto{\pgfqpoint{3.852987in}{3.261339in}}%
\pgfpathlineto{\pgfqpoint{3.890727in}{2.018396in}}%
\pgfpathlineto{\pgfqpoint{3.929724in}{3.046583in}}%
\pgfpathlineto{\pgfqpoint{3.967464in}{3.477754in}}%
\pgfpathlineto{\pgfqpoint{4.006461in}{2.236119in}}%
\pgfpathlineto{\pgfqpoint{4.045459in}{1.272842in}}%
\pgfpathlineto{\pgfqpoint{4.083198in}{3.857947in}}%
\pgfpathlineto{\pgfqpoint{4.122196in}{2.197069in}}%
\pgfpathlineto{\pgfqpoint{4.159935in}{3.566491in}}%
\pgfpathlineto{\pgfqpoint{4.198933in}{3.885508in}}%
\pgfpathlineto{\pgfqpoint{4.237930in}{2.443761in}}%
\pgfpathlineto{\pgfqpoint{4.274411in}{2.464077in}}%
\pgfpathlineto{\pgfqpoint{4.313409in}{3.084748in}}%
\pgfpathlineto{\pgfqpoint{4.351148in}{2.307232in}}%
\pgfpathlineto{\pgfqpoint{4.390146in}{3.286994in}}%
\pgfpathlineto{\pgfqpoint{4.427885in}{2.648397in}}%
\pgfpathlineto{\pgfqpoint{4.466883in}{3.095648in}}%
\pgfpathlineto{\pgfqpoint{4.505880in}{4.179616in}}%
\pgfpathlineto{\pgfqpoint{4.543620in}{2.046559in}}%
\pgfpathlineto{\pgfqpoint{4.582617in}{3.713269in}}%
\pgfpathlineto{\pgfqpoint{4.620357in}{2.794508in}}%
\pgfpathlineto{\pgfqpoint{4.659354in}{3.602255in}}%
\pgfpathlineto{\pgfqpoint{4.698351in}{2.298020in}}%
\pgfpathlineto{\pgfqpoint{4.733575in}{3.082364in}}%
\pgfpathlineto{\pgfqpoint{4.772572in}{2.447820in}}%
\pgfpathlineto{\pgfqpoint{4.810312in}{2.768455in}}%
\pgfpathlineto{\pgfqpoint{4.849309in}{2.456265in}}%
\pgfpathlineto{\pgfqpoint{4.887049in}{3.146317in}}%
\pgfpathlineto{\pgfqpoint{4.926046in}{3.013261in}}%
\pgfpathlineto{\pgfqpoint{4.965044in}{3.140240in}}%
\pgfpathlineto{\pgfqpoint{5.002783in}{3.408455in}}%
\pgfpathlineto{\pgfqpoint{5.041781in}{2.455815in}}%
\pgfpathlineto{\pgfqpoint{5.079520in}{3.269746in}}%
\pgfpathlineto{\pgfqpoint{5.118517in}{3.098675in}}%
\pgfpathlineto{\pgfqpoint{5.157515in}{3.449550in}}%
\pgfpathlineto{\pgfqpoint{5.192738in}{3.644984in}}%
\pgfpathlineto{\pgfqpoint{5.231736in}{4.141249in}}%
\pgfpathlineto{\pgfqpoint{5.269475in}{3.431681in}}%
\pgfpathlineto{\pgfqpoint{5.308473in}{2.970788in}}%
\pgfpathlineto{\pgfqpoint{5.346212in}{2.836593in}}%
\pgfpathlineto{\pgfqpoint{5.385210in}{3.175807in}}%
\pgfpathlineto{\pgfqpoint{5.424207in}{1.967241in}}%
\pgfpathlineto{\pgfqpoint{5.461947in}{2.801888in}}%
\pgfpathlineto{\pgfqpoint{5.500944in}{3.291164in}}%
\pgfpathlineto{\pgfqpoint{5.538683in}{2.506475in}}%
\pgfpathlineto{\pgfqpoint{5.577681in}{3.062322in}}%
\pgfpathlineto{\pgfqpoint{5.616678in}{2.494685in}}%
\pgfpathlineto{\pgfqpoint{5.651902in}{2.070968in}}%
\pgfpathlineto{\pgfqpoint{5.690899in}{3.959378in}}%
\pgfpathlineto{\pgfqpoint{5.728639in}{2.287816in}}%
\pgfpathlineto{\pgfqpoint{5.767636in}{2.656387in}}%
\pgfpathlineto{\pgfqpoint{5.805376in}{2.429290in}}%
\pgfpathlineto{\pgfqpoint{5.844373in}{2.343223in}}%
\pgfpathlineto{\pgfqpoint{5.883371in}{3.309165in}}%
\pgfpathlineto{\pgfqpoint{5.921110in}{2.800330in}}%
\pgfpathlineto{\pgfqpoint{5.960107in}{2.116978in}}%
\pgfpathlineto{\pgfqpoint{5.997847in}{2.985634in}}%
\pgfpathlineto{\pgfqpoint{6.036844in}{2.879183in}}%
\pgfpathlineto{\pgfqpoint{6.075842in}{2.965432in}}%
\pgfpathlineto{\pgfqpoint{6.112323in}{3.027748in}}%
\pgfpathlineto{\pgfqpoint{6.151321in}{3.870666in}}%
\pgfpathlineto{\pgfqpoint{6.189060in}{2.189082in}}%
\pgfpathlineto{\pgfqpoint{6.228058in}{3.920413in}}%
\pgfpathlineto{\pgfqpoint{6.265797in}{3.959442in}}%
\pgfpathlineto{\pgfqpoint{6.304795in}{2.405087in}}%
\pgfpathlineto{\pgfqpoint{6.343792in}{3.664702in}}%
\pgfpathlineto{\pgfqpoint{6.381531in}{2.982022in}}%
\pgfpathlineto{\pgfqpoint{6.420529in}{3.601084in}}%
\pgfpathlineto{\pgfqpoint{6.458268in}{2.168498in}}%
\pgfpathlineto{\pgfqpoint{6.497266in}{3.004932in}}%
\pgfpathlineto{\pgfqpoint{6.536263in}{3.804870in}}%
\pgfpathlineto{\pgfqpoint{6.571487in}{2.358259in}}%
\pgfpathlineto{\pgfqpoint{6.610484in}{2.602631in}}%
\pgfpathlineto{\pgfqpoint{6.648224in}{2.655738in}}%
\pgfpathlineto{\pgfqpoint{6.687221in}{3.405004in}}%
\pgfpathlineto{\pgfqpoint{6.724961in}{3.699328in}}%
\pgfpathlineto{\pgfqpoint{6.763958in}{4.158817in}}%
\pgfpathlineto{\pgfqpoint{6.802955in}{2.161309in}}%
\pgfpathlineto{\pgfqpoint{6.840695in}{4.262724in}}%
\pgfpathlineto{\pgfqpoint{6.879692in}{2.279010in}}%
\pgfpathlineto{\pgfqpoint{6.917432in}{4.114632in}}%
\pgfpathlineto{\pgfqpoint{6.956429in}{3.542742in}}%
\pgfpathlineto{\pgfqpoint{6.995427in}{3.160942in}}%
\pgfpathlineto{\pgfqpoint{7.030650in}{2.777819in}}%
\pgfpathlineto{\pgfqpoint{7.069648in}{3.253571in}}%
\pgfpathlineto{\pgfqpoint{7.107387in}{3.888074in}}%
\pgfpathlineto{\pgfqpoint{7.146385in}{2.892101in}}%
\pgfpathlineto{\pgfqpoint{7.184124in}{2.594241in}}%
\pgfpathlineto{\pgfqpoint{7.223122in}{2.352448in}}%
\pgfpathlineto{\pgfqpoint{7.262119in}{2.934206in}}%
\pgfpathlineto{\pgfqpoint{7.299858in}{2.355532in}}%
\pgfpathlineto{\pgfqpoint{7.338856in}{2.852420in}}%
\pgfpathlineto{\pgfqpoint{7.376595in}{2.698020in}}%
\pgfpathlineto{\pgfqpoint{7.415593in}{4.239771in}}%
\pgfpathlineto{\pgfqpoint{7.454590in}{3.396593in}}%
\pgfpathlineto{\pgfqpoint{7.489814in}{3.015938in}}%
\pgfpathlineto{\pgfqpoint{7.528811in}{2.578653in}}%
\pgfpathlineto{\pgfqpoint{7.566551in}{2.938436in}}%
\pgfpathlineto{\pgfqpoint{7.605548in}{3.471844in}}%
\pgfpathlineto{\pgfqpoint{7.682285in}{4.672416in}}%
\pgfpathlineto{\pgfqpoint{7.721282in}{2.682364in}}%
\pgfpathlineto{\pgfqpoint{7.759022in}{2.870737in}}%
\pgfpathlineto{\pgfqpoint{7.798019in}{1.547575in}}%
\pgfpathlineto{\pgfqpoint{7.835759in}{3.250986in}}%
\pgfpathlineto{\pgfqpoint{7.874756in}{2.385017in}}%
\pgfpathlineto{\pgfqpoint{7.913754in}{2.691860in}}%
\pgfpathlineto{\pgfqpoint{7.950235in}{2.557379in}}%
\pgfpathlineto{\pgfqpoint{7.989233in}{3.512666in}}%
\pgfpathlineto{\pgfqpoint{8.026972in}{3.137173in}}%
\pgfpathlineto{\pgfqpoint{8.065970in}{3.347808in}}%
\pgfpathlineto{\pgfqpoint{8.103709in}{3.619769in}}%
\pgfpathlineto{\pgfqpoint{8.142706in}{3.885762in}}%
\pgfpathlineto{\pgfqpoint{8.181704in}{2.619205in}}%
\pgfpathlineto{\pgfqpoint{8.219443in}{3.384464in}}%
\pgfpathlineto{\pgfqpoint{8.258441in}{3.485320in}}%
\pgfpathlineto{\pgfqpoint{8.296180in}{3.326753in}}%
\pgfpathlineto{\pgfqpoint{8.335178in}{3.535975in}}%
\pgfpathlineto{\pgfqpoint{8.374175in}{3.428798in}}%
\pgfpathlineto{\pgfqpoint{8.409399in}{1.944155in}}%
\pgfpathlineto{\pgfqpoint{8.448396in}{3.783931in}}%
\pgfpathlineto{\pgfqpoint{8.486136in}{3.242547in}}%
\pgfpathlineto{\pgfqpoint{8.525133in}{3.097310in}}%
\pgfpathlineto{\pgfqpoint{8.562872in}{3.597336in}}%
\pgfpathlineto{\pgfqpoint{8.601870in}{3.052790in}}%
\pgfpathlineto{\pgfqpoint{8.640867in}{2.396389in}}%
\pgfpathlineto{\pgfqpoint{8.678607in}{3.674241in}}%
\pgfpathlineto{\pgfqpoint{8.717604in}{3.229454in}}%
\pgfpathlineto{\pgfqpoint{8.755344in}{3.378635in}}%
\pgfpathlineto{\pgfqpoint{8.794341in}{3.613187in}}%
\pgfpathlineto{\pgfqpoint{8.833339in}{3.696894in}}%
\pgfpathlineto{\pgfqpoint{8.868562in}{5.192881in}}%
\pgfpathlineto{\pgfqpoint{8.907560in}{2.974224in}}%
\pgfpathlineto{\pgfqpoint{8.945299in}{2.194052in}}%
\pgfpathlineto{\pgfqpoint{8.984296in}{1.341488in}}%
\pgfpathlineto{\pgfqpoint{9.022036in}{3.359194in}}%
\pgfpathlineto{\pgfqpoint{9.061033in}{2.303507in}}%
\pgfpathlineto{\pgfqpoint{9.100031in}{2.945291in}}%
\pgfpathlineto{\pgfqpoint{9.137770in}{3.305180in}}%
\pgfpathlineto{\pgfqpoint{9.176768in}{4.685356in}}%
\pgfpathlineto{\pgfqpoint{9.214507in}{3.877129in}}%
\pgfpathlineto{\pgfqpoint{9.214507in}{3.877129in}}%
\pgfusepath{stroke}%
\end{pgfscope}%
\begin{pgfscope}%
\pgfsetrectcap%
\pgfsetmiterjoin%
\pgfsetlinewidth{1.254687pt}%
\definecolor{currentstroke}{rgb}{0.150000,0.150000,0.150000}%
\pgfsetstrokecolor{currentstroke}%
\pgfsetdash{}{0pt}%
\pgfpathmoveto{\pgfqpoint{1.095050in}{0.782881in}}%
\pgfpathlineto{\pgfqpoint{1.095050in}{5.402881in}}%
\pgfusepath{stroke}%
\end{pgfscope}%
\begin{pgfscope}%
\pgfsetrectcap%
\pgfsetmiterjoin%
\pgfsetlinewidth{1.254687pt}%
\definecolor{currentstroke}{rgb}{0.150000,0.150000,0.150000}%
\pgfsetstrokecolor{currentstroke}%
\pgfsetdash{}{0pt}%
\pgfpathmoveto{\pgfqpoint{9.601148in}{0.782881in}}%
\pgfpathlineto{\pgfqpoint{9.601148in}{5.402881in}}%
\pgfusepath{stroke}%
\end{pgfscope}%
\begin{pgfscope}%
\pgfsetrectcap%
\pgfsetmiterjoin%
\pgfsetlinewidth{1.254687pt}%
\definecolor{currentstroke}{rgb}{0.150000,0.150000,0.150000}%
\pgfsetstrokecolor{currentstroke}%
\pgfsetdash{}{0pt}%
\pgfpathmoveto{\pgfqpoint{1.095050in}{0.782881in}}%
\pgfpathlineto{\pgfqpoint{9.601148in}{0.782881in}}%
\pgfusepath{stroke}%
\end{pgfscope}%
\begin{pgfscope}%
\pgfsetrectcap%
\pgfsetmiterjoin%
\pgfsetlinewidth{1.254687pt}%
\definecolor{currentstroke}{rgb}{0.150000,0.150000,0.150000}%
\pgfsetstrokecolor{currentstroke}%
\pgfsetdash{}{0pt}%
\pgfpathmoveto{\pgfqpoint{1.095050in}{5.402881in}}%
\pgfpathlineto{\pgfqpoint{9.601148in}{5.402881in}}%
\pgfusepath{stroke}%
\end{pgfscope}%
\begin{pgfscope}%
\pgfsetbuttcap%
\pgfsetmiterjoin%
\definecolor{currentfill}{rgb}{1.000000,1.000000,1.000000}%
\pgfsetfillcolor{currentfill}%
\pgfsetlinewidth{0.000000pt}%
\definecolor{currentstroke}{rgb}{0.000000,0.000000,0.000000}%
\pgfsetstrokecolor{currentstroke}%
\pgfsetstrokeopacity{0.000000}%
\pgfsetdash{}{0pt}%
\pgfpathmoveto{\pgfqpoint{9.884685in}{0.782881in}}%
\pgfpathlineto{\pgfqpoint{12.720050in}{0.782881in}}%
\pgfpathlineto{\pgfqpoint{12.720050in}{5.402881in}}%
\pgfpathlineto{\pgfqpoint{9.884685in}{5.402881in}}%
\pgfpathlineto{\pgfqpoint{9.884685in}{0.782881in}}%
\pgfpathclose%
\pgfusepath{fill}%
\end{pgfscope}%
\begin{pgfscope}%
\pgfpathrectangle{\pgfqpoint{9.884685in}{0.782881in}}{\pgfqpoint{2.835366in}{4.620000in}}%
\pgfusepath{clip}%
\pgfsetroundcap%
\pgfsetroundjoin%
\pgfsetlinewidth{1.003750pt}%
\definecolor{currentstroke}{rgb}{0.800000,0.800000,0.800000}%
\pgfsetstrokecolor{currentstroke}%
\pgfsetstrokeopacity{0.400000}%
\pgfsetdash{}{0pt}%
\pgfpathmoveto{\pgfqpoint{9.884685in}{0.782881in}}%
\pgfpathlineto{\pgfqpoint{9.884685in}{5.402881in}}%
\pgfusepath{stroke}%
\end{pgfscope}%
\begin{pgfscope}%
\definecolor{textcolor}{rgb}{0.150000,0.150000,0.150000}%
\pgfsetstrokecolor{textcolor}%
\pgfsetfillcolor{textcolor}%
\pgftext[x=9.884685in,y=0.650937in,,top]{\color{textcolor}\sffamily\fontsize{19.000000}{22.800000}\selectfont 0}%
\end{pgfscope}%
\begin{pgfscope}%
\pgfpathrectangle{\pgfqpoint{9.884685in}{0.782881in}}{\pgfqpoint{2.835366in}{4.620000in}}%
\pgfusepath{clip}%
\pgfsetroundcap%
\pgfsetroundjoin%
\pgfsetlinewidth{1.003750pt}%
\definecolor{currentstroke}{rgb}{0.800000,0.800000,0.800000}%
\pgfsetstrokecolor{currentstroke}%
\pgfsetstrokeopacity{0.400000}%
\pgfsetdash{}{0pt}%
\pgfpathmoveto{\pgfqpoint{11.234859in}{0.782881in}}%
\pgfpathlineto{\pgfqpoint{11.234859in}{5.402881in}}%
\pgfusepath{stroke}%
\end{pgfscope}%
\begin{pgfscope}%
\definecolor{textcolor}{rgb}{0.150000,0.150000,0.150000}%
\pgfsetstrokecolor{textcolor}%
\pgfsetfillcolor{textcolor}%
\pgftext[x=11.234859in,y=0.650937in,,top]{\color{textcolor}\sffamily\fontsize{19.000000}{22.800000}\selectfont 20}%
\end{pgfscope}%
\begin{pgfscope}%
\pgfpathrectangle{\pgfqpoint{9.884685in}{0.782881in}}{\pgfqpoint{2.835366in}{4.620000in}}%
\pgfusepath{clip}%
\pgfsetroundcap%
\pgfsetroundjoin%
\pgfsetlinewidth{1.003750pt}%
\definecolor{currentstroke}{rgb}{0.800000,0.800000,0.800000}%
\pgfsetstrokecolor{currentstroke}%
\pgfsetstrokeopacity{0.400000}%
\pgfsetdash{}{0pt}%
\pgfpathmoveto{\pgfqpoint{12.585033in}{0.782881in}}%
\pgfpathlineto{\pgfqpoint{12.585033in}{5.402881in}}%
\pgfusepath{stroke}%
\end{pgfscope}%
\begin{pgfscope}%
\definecolor{textcolor}{rgb}{0.150000,0.150000,0.150000}%
\pgfsetstrokecolor{textcolor}%
\pgfsetfillcolor{textcolor}%
\pgftext[x=12.585033in,y=0.650937in,,top]{\color{textcolor}\sffamily\fontsize{19.000000}{22.800000}\selectfont 40}%
\end{pgfscope}%
\begin{pgfscope}%
\definecolor{textcolor}{rgb}{0.150000,0.150000,0.150000}%
\pgfsetstrokecolor{textcolor}%
\pgfsetfillcolor{textcolor}%
\pgftext[x=11.302367in,y=0.354042in,,top]{\color{textcolor}\sffamily\fontsize{20.000000}{24.000000}\selectfont Nombre d'occurences}%
\end{pgfscope}%
\begin{pgfscope}%
\pgfpathrectangle{\pgfqpoint{9.884685in}{0.782881in}}{\pgfqpoint{2.835366in}{4.620000in}}%
\pgfusepath{clip}%
\pgfsetbuttcap%
\pgfsetmiterjoin%
\definecolor{currentfill}{rgb}{0.325490,0.619608,0.803922}%
\pgfsetfillcolor{currentfill}%
\pgfsetfillopacity{0.750000}%
\pgfsetlinewidth{1.003750pt}%
\definecolor{currentstroke}{rgb}{1.000000,1.000000,1.000000}%
\pgfsetstrokecolor{currentstroke}%
\pgfsetdash{}{0pt}%
\pgfpathmoveto{\pgfqpoint{9.884685in}{0.992881in}}%
\pgfpathlineto{\pgfqpoint{10.289737in}{0.992881in}}%
\pgfpathlineto{\pgfqpoint{10.289737in}{1.315958in}}%
\pgfpathlineto{\pgfqpoint{9.884685in}{1.315958in}}%
\pgfpathlineto{\pgfqpoint{9.884685in}{0.992881in}}%
\pgfpathclose%
\pgfusepath{stroke,fill}%
\end{pgfscope}%
\begin{pgfscope}%
\pgfpathrectangle{\pgfqpoint{9.884685in}{0.782881in}}{\pgfqpoint{2.835366in}{4.620000in}}%
\pgfusepath{clip}%
\pgfsetbuttcap%
\pgfsetmiterjoin%
\definecolor{currentfill}{rgb}{0.325490,0.619608,0.803922}%
\pgfsetfillcolor{currentfill}%
\pgfsetfillopacity{0.750000}%
\pgfsetlinewidth{1.003750pt}%
\definecolor{currentstroke}{rgb}{1.000000,1.000000,1.000000}%
\pgfsetstrokecolor{currentstroke}%
\pgfsetdash{}{0pt}%
\pgfpathmoveto{\pgfqpoint{9.884685in}{1.315958in}}%
\pgfpathlineto{\pgfqpoint{10.019702in}{1.315958in}}%
\pgfpathlineto{\pgfqpoint{10.019702in}{1.639035in}}%
\pgfpathlineto{\pgfqpoint{9.884685in}{1.639035in}}%
\pgfpathlineto{\pgfqpoint{9.884685in}{1.315958in}}%
\pgfpathclose%
\pgfusepath{stroke,fill}%
\end{pgfscope}%
\begin{pgfscope}%
\pgfpathrectangle{\pgfqpoint{9.884685in}{0.782881in}}{\pgfqpoint{2.835366in}{4.620000in}}%
\pgfusepath{clip}%
\pgfsetbuttcap%
\pgfsetmiterjoin%
\definecolor{currentfill}{rgb}{0.325490,0.619608,0.803922}%
\pgfsetfillcolor{currentfill}%
\pgfsetfillopacity{0.750000}%
\pgfsetlinewidth{1.003750pt}%
\definecolor{currentstroke}{rgb}{1.000000,1.000000,1.000000}%
\pgfsetstrokecolor{currentstroke}%
\pgfsetdash{}{0pt}%
\pgfpathmoveto{\pgfqpoint{9.884685in}{1.639035in}}%
\pgfpathlineto{\pgfqpoint{10.154719in}{1.639035in}}%
\pgfpathlineto{\pgfqpoint{10.154719in}{1.962112in}}%
\pgfpathlineto{\pgfqpoint{9.884685in}{1.962112in}}%
\pgfpathlineto{\pgfqpoint{9.884685in}{1.639035in}}%
\pgfpathclose%
\pgfusepath{stroke,fill}%
\end{pgfscope}%
\begin{pgfscope}%
\pgfpathrectangle{\pgfqpoint{9.884685in}{0.782881in}}{\pgfqpoint{2.835366in}{4.620000in}}%
\pgfusepath{clip}%
\pgfsetbuttcap%
\pgfsetmiterjoin%
\definecolor{currentfill}{rgb}{0.325490,0.619608,0.803922}%
\pgfsetfillcolor{currentfill}%
\pgfsetfillopacity{0.750000}%
\pgfsetlinewidth{1.003750pt}%
\definecolor{currentstroke}{rgb}{1.000000,1.000000,1.000000}%
\pgfsetstrokecolor{currentstroke}%
\pgfsetdash{}{0pt}%
\pgfpathmoveto{\pgfqpoint{9.884685in}{1.962112in}}%
\pgfpathlineto{\pgfqpoint{10.964824in}{1.962112in}}%
\pgfpathlineto{\pgfqpoint{10.964824in}{2.285189in}}%
\pgfpathlineto{\pgfqpoint{9.884685in}{2.285189in}}%
\pgfpathlineto{\pgfqpoint{9.884685in}{1.962112in}}%
\pgfpathclose%
\pgfusepath{stroke,fill}%
\end{pgfscope}%
\begin{pgfscope}%
\pgfpathrectangle{\pgfqpoint{9.884685in}{0.782881in}}{\pgfqpoint{2.835366in}{4.620000in}}%
\pgfusepath{clip}%
\pgfsetbuttcap%
\pgfsetmiterjoin%
\definecolor{currentfill}{rgb}{0.325490,0.619608,0.803922}%
\pgfsetfillcolor{currentfill}%
\pgfsetfillopacity{0.750000}%
\pgfsetlinewidth{1.003750pt}%
\definecolor{currentstroke}{rgb}{1.000000,1.000000,1.000000}%
\pgfsetstrokecolor{currentstroke}%
\pgfsetdash{}{0pt}%
\pgfpathmoveto{\pgfqpoint{9.884685in}{2.285189in}}%
\pgfpathlineto{\pgfqpoint{11.707420in}{2.285189in}}%
\pgfpathlineto{\pgfqpoint{11.707420in}{2.608266in}}%
\pgfpathlineto{\pgfqpoint{9.884685in}{2.608266in}}%
\pgfpathlineto{\pgfqpoint{9.884685in}{2.285189in}}%
\pgfpathclose%
\pgfusepath{stroke,fill}%
\end{pgfscope}%
\begin{pgfscope}%
\pgfpathrectangle{\pgfqpoint{9.884685in}{0.782881in}}{\pgfqpoint{2.835366in}{4.620000in}}%
\pgfusepath{clip}%
\pgfsetbuttcap%
\pgfsetmiterjoin%
\definecolor{currentfill}{rgb}{0.325490,0.619608,0.803922}%
\pgfsetfillcolor{currentfill}%
\pgfsetfillopacity{0.750000}%
\pgfsetlinewidth{1.003750pt}%
\definecolor{currentstroke}{rgb}{1.000000,1.000000,1.000000}%
\pgfsetstrokecolor{currentstroke}%
\pgfsetdash{}{0pt}%
\pgfpathmoveto{\pgfqpoint{9.884685in}{2.608266in}}%
\pgfpathlineto{\pgfqpoint{11.437385in}{2.608266in}}%
\pgfpathlineto{\pgfqpoint{11.437385in}{2.931343in}}%
\pgfpathlineto{\pgfqpoint{9.884685in}{2.931343in}}%
\pgfpathlineto{\pgfqpoint{9.884685in}{2.608266in}}%
\pgfpathclose%
\pgfusepath{stroke,fill}%
\end{pgfscope}%
\begin{pgfscope}%
\pgfpathrectangle{\pgfqpoint{9.884685in}{0.782881in}}{\pgfqpoint{2.835366in}{4.620000in}}%
\pgfusepath{clip}%
\pgfsetbuttcap%
\pgfsetmiterjoin%
\definecolor{currentfill}{rgb}{0.325490,0.619608,0.803922}%
\pgfsetfillcolor{currentfill}%
\pgfsetfillopacity{0.750000}%
\pgfsetlinewidth{1.003750pt}%
\definecolor{currentstroke}{rgb}{1.000000,1.000000,1.000000}%
\pgfsetstrokecolor{currentstroke}%
\pgfsetdash{}{0pt}%
\pgfpathmoveto{\pgfqpoint{9.884685in}{2.931343in}}%
\pgfpathlineto{\pgfqpoint{12.585033in}{2.931343in}}%
\pgfpathlineto{\pgfqpoint{12.585033in}{3.254420in}}%
\pgfpathlineto{\pgfqpoint{9.884685in}{3.254420in}}%
\pgfpathlineto{\pgfqpoint{9.884685in}{2.931343in}}%
\pgfpathclose%
\pgfusepath{stroke,fill}%
\end{pgfscope}%
\begin{pgfscope}%
\pgfpathrectangle{\pgfqpoint{9.884685in}{0.782881in}}{\pgfqpoint{2.835366in}{4.620000in}}%
\pgfusepath{clip}%
\pgfsetbuttcap%
\pgfsetmiterjoin%
\definecolor{currentfill}{rgb}{0.325490,0.619608,0.803922}%
\pgfsetfillcolor{currentfill}%
\pgfsetfillopacity{0.750000}%
\pgfsetlinewidth{1.003750pt}%
\definecolor{currentstroke}{rgb}{1.000000,1.000000,1.000000}%
\pgfsetstrokecolor{currentstroke}%
\pgfsetdash{}{0pt}%
\pgfpathmoveto{\pgfqpoint{9.884685in}{3.254420in}}%
\pgfpathlineto{\pgfqpoint{12.044963in}{3.254420in}}%
\pgfpathlineto{\pgfqpoint{12.044963in}{3.577497in}}%
\pgfpathlineto{\pgfqpoint{9.884685in}{3.577497in}}%
\pgfpathlineto{\pgfqpoint{9.884685in}{3.254420in}}%
\pgfpathclose%
\pgfusepath{stroke,fill}%
\end{pgfscope}%
\begin{pgfscope}%
\pgfpathrectangle{\pgfqpoint{9.884685in}{0.782881in}}{\pgfqpoint{2.835366in}{4.620000in}}%
\pgfusepath{clip}%
\pgfsetbuttcap%
\pgfsetmiterjoin%
\definecolor{currentfill}{rgb}{0.325490,0.619608,0.803922}%
\pgfsetfillcolor{currentfill}%
\pgfsetfillopacity{0.750000}%
\pgfsetlinewidth{1.003750pt}%
\definecolor{currentstroke}{rgb}{1.000000,1.000000,1.000000}%
\pgfsetstrokecolor{currentstroke}%
\pgfsetdash{}{0pt}%
\pgfpathmoveto{\pgfqpoint{9.884685in}{3.577497in}}%
\pgfpathlineto{\pgfqpoint{11.504894in}{3.577497in}}%
\pgfpathlineto{\pgfqpoint{11.504894in}{3.900574in}}%
\pgfpathlineto{\pgfqpoint{9.884685in}{3.900574in}}%
\pgfpathlineto{\pgfqpoint{9.884685in}{3.577497in}}%
\pgfpathclose%
\pgfusepath{stroke,fill}%
\end{pgfscope}%
\begin{pgfscope}%
\pgfpathrectangle{\pgfqpoint{9.884685in}{0.782881in}}{\pgfqpoint{2.835366in}{4.620000in}}%
\pgfusepath{clip}%
\pgfsetbuttcap%
\pgfsetmiterjoin%
\definecolor{currentfill}{rgb}{0.325490,0.619608,0.803922}%
\pgfsetfillcolor{currentfill}%
\pgfsetfillopacity{0.750000}%
\pgfsetlinewidth{1.003750pt}%
\definecolor{currentstroke}{rgb}{1.000000,1.000000,1.000000}%
\pgfsetstrokecolor{currentstroke}%
\pgfsetdash{}{0pt}%
\pgfpathmoveto{\pgfqpoint{9.884685in}{3.900574in}}%
\pgfpathlineto{\pgfqpoint{11.167350in}{3.900574in}}%
\pgfpathlineto{\pgfqpoint{11.167350in}{4.223650in}}%
\pgfpathlineto{\pgfqpoint{9.884685in}{4.223650in}}%
\pgfpathlineto{\pgfqpoint{9.884685in}{3.900574in}}%
\pgfpathclose%
\pgfusepath{stroke,fill}%
\end{pgfscope}%
\begin{pgfscope}%
\pgfpathrectangle{\pgfqpoint{9.884685in}{0.782881in}}{\pgfqpoint{2.835366in}{4.620000in}}%
\pgfusepath{clip}%
\pgfsetbuttcap%
\pgfsetmiterjoin%
\definecolor{currentfill}{rgb}{0.325490,0.619608,0.803922}%
\pgfsetfillcolor{currentfill}%
\pgfsetfillopacity{0.750000}%
\pgfsetlinewidth{1.003750pt}%
\definecolor{currentstroke}{rgb}{1.000000,1.000000,1.000000}%
\pgfsetstrokecolor{currentstroke}%
\pgfsetdash{}{0pt}%
\pgfpathmoveto{\pgfqpoint{9.884685in}{4.223650in}}%
\pgfpathlineto{\pgfqpoint{10.289737in}{4.223650in}}%
\pgfpathlineto{\pgfqpoint{10.289737in}{4.546727in}}%
\pgfpathlineto{\pgfqpoint{9.884685in}{4.546727in}}%
\pgfpathlineto{\pgfqpoint{9.884685in}{4.223650in}}%
\pgfpathclose%
\pgfusepath{stroke,fill}%
\end{pgfscope}%
\begin{pgfscope}%
\pgfpathrectangle{\pgfqpoint{9.884685in}{0.782881in}}{\pgfqpoint{2.835366in}{4.620000in}}%
\pgfusepath{clip}%
\pgfsetbuttcap%
\pgfsetmiterjoin%
\definecolor{currentfill}{rgb}{0.325490,0.619608,0.803922}%
\pgfsetfillcolor{currentfill}%
\pgfsetfillopacity{0.750000}%
\pgfsetlinewidth{1.003750pt}%
\definecolor{currentstroke}{rgb}{1.000000,1.000000,1.000000}%
\pgfsetstrokecolor{currentstroke}%
\pgfsetdash{}{0pt}%
\pgfpathmoveto{\pgfqpoint{9.884685in}{4.546727in}}%
\pgfpathlineto{\pgfqpoint{10.019702in}{4.546727in}}%
\pgfpathlineto{\pgfqpoint{10.019702in}{4.869804in}}%
\pgfpathlineto{\pgfqpoint{9.884685in}{4.869804in}}%
\pgfpathlineto{\pgfqpoint{9.884685in}{4.546727in}}%
\pgfpathclose%
\pgfusepath{stroke,fill}%
\end{pgfscope}%
\begin{pgfscope}%
\pgfpathrectangle{\pgfqpoint{9.884685in}{0.782881in}}{\pgfqpoint{2.835366in}{4.620000in}}%
\pgfusepath{clip}%
\pgfsetbuttcap%
\pgfsetmiterjoin%
\definecolor{currentfill}{rgb}{0.325490,0.619608,0.803922}%
\pgfsetfillcolor{currentfill}%
\pgfsetfillopacity{0.750000}%
\pgfsetlinewidth{1.003750pt}%
\definecolor{currentstroke}{rgb}{1.000000,1.000000,1.000000}%
\pgfsetstrokecolor{currentstroke}%
\pgfsetdash{}{0pt}%
\pgfpathmoveto{\pgfqpoint{9.884685in}{4.869804in}}%
\pgfpathlineto{\pgfqpoint{10.019702in}{4.869804in}}%
\pgfpathlineto{\pgfqpoint{10.019702in}{5.192881in}}%
\pgfpathlineto{\pgfqpoint{9.884685in}{5.192881in}}%
\pgfpathlineto{\pgfqpoint{9.884685in}{4.869804in}}%
\pgfpathclose%
\pgfusepath{stroke,fill}%
\end{pgfscope}%
\begin{pgfscope}%
\pgfsetrectcap%
\pgfsetmiterjoin%
\pgfsetlinewidth{1.254687pt}%
\definecolor{currentstroke}{rgb}{0.150000,0.150000,0.150000}%
\pgfsetstrokecolor{currentstroke}%
\pgfsetdash{}{0pt}%
\pgfpathmoveto{\pgfqpoint{9.884685in}{0.782881in}}%
\pgfpathlineto{\pgfqpoint{9.884685in}{5.402881in}}%
\pgfusepath{stroke}%
\end{pgfscope}%
\begin{pgfscope}%
\pgfsetrectcap%
\pgfsetmiterjoin%
\pgfsetlinewidth{1.254687pt}%
\definecolor{currentstroke}{rgb}{0.150000,0.150000,0.150000}%
\pgfsetstrokecolor{currentstroke}%
\pgfsetdash{}{0pt}%
\pgfpathmoveto{\pgfqpoint{12.720050in}{0.782881in}}%
\pgfpathlineto{\pgfqpoint{12.720050in}{5.402881in}}%
\pgfusepath{stroke}%
\end{pgfscope}%
\begin{pgfscope}%
\pgfsetrectcap%
\pgfsetmiterjoin%
\pgfsetlinewidth{1.254687pt}%
\definecolor{currentstroke}{rgb}{0.150000,0.150000,0.150000}%
\pgfsetstrokecolor{currentstroke}%
\pgfsetdash{}{0pt}%
\pgfpathmoveto{\pgfqpoint{9.884685in}{0.782881in}}%
\pgfpathlineto{\pgfqpoint{12.720050in}{0.782881in}}%
\pgfusepath{stroke}%
\end{pgfscope}%
\begin{pgfscope}%
\pgfsetrectcap%
\pgfsetmiterjoin%
\pgfsetlinewidth{1.254687pt}%
\definecolor{currentstroke}{rgb}{0.150000,0.150000,0.150000}%
\pgfsetstrokecolor{currentstroke}%
\pgfsetdash{}{0pt}%
\pgfpathmoveto{\pgfqpoint{9.884685in}{5.402881in}}%
\pgfpathlineto{\pgfqpoint{12.720050in}{5.402881in}}%
\pgfusepath{stroke}%
\end{pgfscope}%
\begin{pgfscope}%
\definecolor{textcolor}{rgb}{0.150000,0.150000,0.150000}%
\pgfsetstrokecolor{textcolor}%
\pgfsetfillcolor{textcolor}%
\pgftext[x=10.762298in,y=5.121400in,left,base]{\color{textcolor}\sffamily\fontsize{14.000000}{16.800000}\selectfont Skewness: -0.226152}%
\end{pgfscope}%
\begin{pgfscope}%
\definecolor{textcolor}{rgb}{0.150000,0.150000,0.150000}%
\pgfsetstrokecolor{textcolor}%
\pgfsetfillcolor{textcolor}%
\pgftext[x=10.762298in,y=4.891772in,left,base]{\color{textcolor}\sffamily\fontsize{14.000000}{16.800000}\selectfont Kurtosis: 0.168668}%
\end{pgfscope}%
\end{pgfpicture}%
\makeatother%
\endgroup%
}
    \caption{Distribution des rendements du cours du nickel de 2006 à 2022}
\end{figure}
\subsection{Analyse macroéconomique des cours}
Un contrat à terme, ou future est un contrat à terme par lequel deux parties s'engagent à acheter ou vendre une quantité déterminée d'un actif sous-jacent (une 
action ou un indice boursier par exemple ici des matières premières), à une date d'échéance et à un prix convenus à l'avance. Il permet d'anticiper les variations futures 
d'un actif sous-jacent et peut donc servir à couvrir un portefeuille contre les fluctuations à venir du marché. Il permet aussi de dynamiser les performances d'un 
portefeuille. Le contrat a terme constitue un engagement ferme, il doit être exécuté à sa date d'échéance par ses contreparties : l'acheteur contrat doit acheter l'actif 
sous-jacent au prix convenu et le vendeur doit livrer l'actif.\\[11pt]
Il est possible d'apercevoir de fortes augmentations des cours lors des périodes de crise. En effet, à chaque crise ou détection de crise par les investisseurs, le prix 
des contrats augmentent, les investisseurs étant plongés dans un climat d'incertitude. Afin d'avoir plus de détails sur les fluctuations, un historique des évènements 
ayant marqué les deux cours des matières premières est dressé.
\begin{itemize}
    \item \textbf{2003 : Sécheresse}\\
    La France subit un fort climat de sécheresse ce qui réduit les récoltes de blé meunier de 20\% par rapport à 2002. S'ajoute à cela, l'invasion des États-Unis en Irak, 
    ces deux éléments peuvent expliquer l'augmentation du prix, la sécheresse étant le principal élément explicatif.
    Le cours du blé était de 110,01 \euro\ en mai 2003 et est passé à 161,56  en octobre 2003 (+46,86\%).

    \item \textbf{2007-2008 : Crise des subprimes}
    \begin{itemize}
        \item Le blé : Le 1 juin 2006 le cours était de 111,23 \euro\, dès 2007 le cours va augmenter fortement jusqu'à 281,32 \euro\ le 1 février 2008 (+152,92\%).
        \item Le Nickel :  Tout comme le blé, le cours des contrats futures sur le nickel a été largement impacté par la crise des subprimes. Ce dernier est passé de 14 645 \$\ en février 2006, à 49 675 \$\ en avril 2007 le point culminant pendant la crise des subprimes. Soit une multiplication du cours sur 14 mois de 3,39 fois.
    \end{itemize}
    
    \item \textbf{2010-2013 : Crise de la dette Européenne}
    \begin{itemize}
        \item Le blé : La crise de la dette européennes a été longue sur les marchés financiers et à donc longuement impacter le cours du blé.
        En mai 2010 le cours était de : 132,65 \euro\ et a atteint deux points culminants : en janvier 2011 270,75 \euro\ et en novembre 2012 269,7 \euro\. Preuve 
        de la persistance du climat d'incertitude.
        \item Le nickel : Comme un grand nombre de matière première, le nickel a subi l'impact de la crise de la dette souveraine, en effet le cours est passé de 9 837 \$\ en avril 2009 à 28 875 \$\ en février 2011, soit une augmentation de 193,53\%.
    \end{itemize}
    \item \textbf{2020-2023 : Crise liée à la pandémie de Covid-19, guerre russo-ukrainienne, et réchauffement climatique}
    \begin{itemize}
        \item Le blé : La pandémie de Covid19 et les différents confinements qui lui sont associés ont provoqué des disruptions sur les chaînes de valeur mondiales et par 
        conséquent sur les marchés aussi. De plus, les récoltes de blé sont menacées par la sécheresse due au réchauffement climatique, et le commerce du blé s'est 
        retrouvé bouleversé par la guerre en Ukraine, faisant flamber le cours du contrat sur le blé à des prix inégalés (400 \euro).
        \item Le nickel : Le nickel est surnommé le "métal du diable" de part sa volatilité. En plus des causes citées précédemment la demande est forte augmentation suite à l'engouement autour des véhicules électriques. En 2022 le cours a franchi à nouveau la barre symbolique des 24 000 \$\. Le cours a quadruplé en l'espace de deux ans, il à grimpé de 10 887 \$\ en mars 2020 à 42 100 \$\ en mars 2022.
    \end{itemize}
\end{itemize}
\begin{figure}[H]
    \centering
    \resizebox{\textwidth}{!}{%% Creator: Matplotlib, PGF backend
%%
%% To include the figure in your LaTeX document, write
%%   \input{<filename>.pgf}
%%
%% Make sure the required packages are loaded in your preamble
%%   \usepackage{pgf}
%%
%% Also ensure that all the required font packages are loaded; for instance,
%% the lmodern package is sometimes necessary when using math font.
%%   \usepackage{lmodern}
%%
%% Figures using additional raster images can only be included by \input if
%% they are in the same directory as the main LaTeX file. For loading figures
%% from other directories you can use the `import` package
%%   \usepackage{import}
%%
%% and then include the figures with
%%   \import{<path to file>}{<filename>.pgf}
%%
%% Matplotlib used the following preamble
%%   \usepackage{fontspec}
%%   \setmainfont{DejaVuSerif.ttf}[Path=\detokenize{C:/Users/Joseph/miniconda3/Lib/site-packages/matplotlib/mpl-data/fonts/ttf/}]
%%   \setsansfont{arial.ttf}[Path=\detokenize{C:/Windows/Fonts/}]
%%   \setmonofont{DejaVuSansMono.ttf}[Path=\detokenize{C:/Users/Joseph/miniconda3/Lib/site-packages/matplotlib/mpl-data/fonts/ttf/}]
%%
\begingroup%
\makeatletter%
\begin{pgfpicture}%
\pgfpathrectangle{\pgfpointorigin}{\pgfqpoint{16.588475in}{5.502881in}}%
\pgfusepath{use as bounding box, clip}%
\begin{pgfscope}%
\pgfsetbuttcap%
\pgfsetmiterjoin%
\definecolor{currentfill}{rgb}{1.000000,1.000000,1.000000}%
\pgfsetfillcolor{currentfill}%
\pgfsetlinewidth{0.000000pt}%
\definecolor{currentstroke}{rgb}{1.000000,1.000000,1.000000}%
\pgfsetstrokecolor{currentstroke}%
\pgfsetdash{}{0pt}%
\pgfpathmoveto{\pgfqpoint{0.000000in}{0.000000in}}%
\pgfpathlineto{\pgfqpoint{16.588475in}{0.000000in}}%
\pgfpathlineto{\pgfqpoint{16.588475in}{5.502881in}}%
\pgfpathlineto{\pgfqpoint{0.000000in}{5.502881in}}%
\pgfpathlineto{\pgfqpoint{0.000000in}{0.000000in}}%
\pgfpathclose%
\pgfusepath{fill}%
\end{pgfscope}%
\begin{pgfscope}%
\pgfsetbuttcap%
\pgfsetmiterjoin%
\definecolor{currentfill}{rgb}{1.000000,1.000000,1.000000}%
\pgfsetfillcolor{currentfill}%
\pgfsetlinewidth{0.000000pt}%
\definecolor{currentstroke}{rgb}{0.000000,0.000000,0.000000}%
\pgfsetstrokecolor{currentstroke}%
\pgfsetstrokeopacity{0.000000}%
\pgfsetdash{}{0pt}%
\pgfpathmoveto{\pgfqpoint{0.988475in}{0.782881in}}%
\pgfpathlineto{\pgfqpoint{8.033930in}{0.782881in}}%
\pgfpathlineto{\pgfqpoint{8.033930in}{5.402881in}}%
\pgfpathlineto{\pgfqpoint{0.988475in}{5.402881in}}%
\pgfpathlineto{\pgfqpoint{0.988475in}{0.782881in}}%
\pgfpathclose%
\pgfusepath{fill}%
\end{pgfscope}%
\begin{pgfscope}%
\pgfpathrectangle{\pgfqpoint{0.988475in}{0.782881in}}{\pgfqpoint{7.045455in}{4.620000in}}%
\pgfusepath{clip}%
\pgfsetroundcap%
\pgfsetroundjoin%
\pgfsetlinewidth{1.003750pt}%
\definecolor{currentstroke}{rgb}{0.800000,0.800000,0.800000}%
\pgfsetstrokecolor{currentstroke}%
\pgfsetstrokeopacity{0.000000}%
\pgfsetdash{}{0pt}%
\pgfpathmoveto{\pgfqpoint{1.342007in}{0.782881in}}%
\pgfpathlineto{\pgfqpoint{1.342007in}{5.402881in}}%
\pgfusepath{stroke}%
\end{pgfscope}%
\begin{pgfscope}%
\definecolor{textcolor}{rgb}{0.150000,0.150000,0.150000}%
\pgfsetstrokecolor{textcolor}%
\pgfsetfillcolor{textcolor}%
\pgftext[x=1.342007in,y=0.650937in,,top]{\color{textcolor}\sffamily\fontsize{19.000000}{22.800000}\selectfont 2004}%
\end{pgfscope}%
\begin{pgfscope}%
\pgfpathrectangle{\pgfqpoint{0.988475in}{0.782881in}}{\pgfqpoint{7.045455in}{4.620000in}}%
\pgfusepath{clip}%
\pgfsetroundcap%
\pgfsetroundjoin%
\pgfsetlinewidth{1.003750pt}%
\definecolor{currentstroke}{rgb}{0.800000,0.800000,0.800000}%
\pgfsetstrokecolor{currentstroke}%
\pgfsetstrokeopacity{0.000000}%
\pgfsetdash{}{0pt}%
\pgfpathmoveto{\pgfqpoint{2.050039in}{0.782881in}}%
\pgfpathlineto{\pgfqpoint{2.050039in}{5.402881in}}%
\pgfusepath{stroke}%
\end{pgfscope}%
\begin{pgfscope}%
\definecolor{textcolor}{rgb}{0.150000,0.150000,0.150000}%
\pgfsetstrokecolor{textcolor}%
\pgfsetfillcolor{textcolor}%
\pgftext[x=2.050039in,y=0.650937in,,top]{\color{textcolor}\sffamily\fontsize{19.000000}{22.800000}\selectfont 2006}%
\end{pgfscope}%
\begin{pgfscope}%
\pgfpathrectangle{\pgfqpoint{0.988475in}{0.782881in}}{\pgfqpoint{7.045455in}{4.620000in}}%
\pgfusepath{clip}%
\pgfsetroundcap%
\pgfsetroundjoin%
\pgfsetlinewidth{1.003750pt}%
\definecolor{currentstroke}{rgb}{0.800000,0.800000,0.800000}%
\pgfsetstrokecolor{currentstroke}%
\pgfsetstrokeopacity{0.000000}%
\pgfsetdash{}{0pt}%
\pgfpathmoveto{\pgfqpoint{2.757103in}{0.782881in}}%
\pgfpathlineto{\pgfqpoint{2.757103in}{5.402881in}}%
\pgfusepath{stroke}%
\end{pgfscope}%
\begin{pgfscope}%
\definecolor{textcolor}{rgb}{0.150000,0.150000,0.150000}%
\pgfsetstrokecolor{textcolor}%
\pgfsetfillcolor{textcolor}%
\pgftext[x=2.757103in,y=0.650937in,,top]{\color{textcolor}\sffamily\fontsize{19.000000}{22.800000}\selectfont 2008}%
\end{pgfscope}%
\begin{pgfscope}%
\pgfpathrectangle{\pgfqpoint{0.988475in}{0.782881in}}{\pgfqpoint{7.045455in}{4.620000in}}%
\pgfusepath{clip}%
\pgfsetroundcap%
\pgfsetroundjoin%
\pgfsetlinewidth{1.003750pt}%
\definecolor{currentstroke}{rgb}{0.800000,0.800000,0.800000}%
\pgfsetstrokecolor{currentstroke}%
\pgfsetstrokeopacity{0.000000}%
\pgfsetdash{}{0pt}%
\pgfpathmoveto{\pgfqpoint{3.465135in}{0.782881in}}%
\pgfpathlineto{\pgfqpoint{3.465135in}{5.402881in}}%
\pgfusepath{stroke}%
\end{pgfscope}%
\begin{pgfscope}%
\definecolor{textcolor}{rgb}{0.150000,0.150000,0.150000}%
\pgfsetstrokecolor{textcolor}%
\pgfsetfillcolor{textcolor}%
\pgftext[x=3.465135in,y=0.650937in,,top]{\color{textcolor}\sffamily\fontsize{19.000000}{22.800000}\selectfont 2010}%
\end{pgfscope}%
\begin{pgfscope}%
\pgfpathrectangle{\pgfqpoint{0.988475in}{0.782881in}}{\pgfqpoint{7.045455in}{4.620000in}}%
\pgfusepath{clip}%
\pgfsetroundcap%
\pgfsetroundjoin%
\pgfsetlinewidth{1.003750pt}%
\definecolor{currentstroke}{rgb}{0.800000,0.800000,0.800000}%
\pgfsetstrokecolor{currentstroke}%
\pgfsetstrokeopacity{0.000000}%
\pgfsetdash{}{0pt}%
\pgfpathmoveto{\pgfqpoint{4.172199in}{0.782881in}}%
\pgfpathlineto{\pgfqpoint{4.172199in}{5.402881in}}%
\pgfusepath{stroke}%
\end{pgfscope}%
\begin{pgfscope}%
\definecolor{textcolor}{rgb}{0.150000,0.150000,0.150000}%
\pgfsetstrokecolor{textcolor}%
\pgfsetfillcolor{textcolor}%
\pgftext[x=4.172199in,y=0.650937in,,top]{\color{textcolor}\sffamily\fontsize{19.000000}{22.800000}\selectfont 2012}%
\end{pgfscope}%
\begin{pgfscope}%
\pgfpathrectangle{\pgfqpoint{0.988475in}{0.782881in}}{\pgfqpoint{7.045455in}{4.620000in}}%
\pgfusepath{clip}%
\pgfsetroundcap%
\pgfsetroundjoin%
\pgfsetlinewidth{1.003750pt}%
\definecolor{currentstroke}{rgb}{0.800000,0.800000,0.800000}%
\pgfsetstrokecolor{currentstroke}%
\pgfsetstrokeopacity{0.000000}%
\pgfsetdash{}{0pt}%
\pgfpathmoveto{\pgfqpoint{4.880231in}{0.782881in}}%
\pgfpathlineto{\pgfqpoint{4.880231in}{5.402881in}}%
\pgfusepath{stroke}%
\end{pgfscope}%
\begin{pgfscope}%
\definecolor{textcolor}{rgb}{0.150000,0.150000,0.150000}%
\pgfsetstrokecolor{textcolor}%
\pgfsetfillcolor{textcolor}%
\pgftext[x=4.880231in,y=0.650937in,,top]{\color{textcolor}\sffamily\fontsize{19.000000}{22.800000}\selectfont 2014}%
\end{pgfscope}%
\begin{pgfscope}%
\pgfpathrectangle{\pgfqpoint{0.988475in}{0.782881in}}{\pgfqpoint{7.045455in}{4.620000in}}%
\pgfusepath{clip}%
\pgfsetroundcap%
\pgfsetroundjoin%
\pgfsetlinewidth{1.003750pt}%
\definecolor{currentstroke}{rgb}{0.800000,0.800000,0.800000}%
\pgfsetstrokecolor{currentstroke}%
\pgfsetstrokeopacity{0.000000}%
\pgfsetdash{}{0pt}%
\pgfpathmoveto{\pgfqpoint{5.587295in}{0.782881in}}%
\pgfpathlineto{\pgfqpoint{5.587295in}{5.402881in}}%
\pgfusepath{stroke}%
\end{pgfscope}%
\begin{pgfscope}%
\definecolor{textcolor}{rgb}{0.150000,0.150000,0.150000}%
\pgfsetstrokecolor{textcolor}%
\pgfsetfillcolor{textcolor}%
\pgftext[x=5.587295in,y=0.650937in,,top]{\color{textcolor}\sffamily\fontsize{19.000000}{22.800000}\selectfont 2016}%
\end{pgfscope}%
\begin{pgfscope}%
\pgfpathrectangle{\pgfqpoint{0.988475in}{0.782881in}}{\pgfqpoint{7.045455in}{4.620000in}}%
\pgfusepath{clip}%
\pgfsetroundcap%
\pgfsetroundjoin%
\pgfsetlinewidth{1.003750pt}%
\definecolor{currentstroke}{rgb}{0.800000,0.800000,0.800000}%
\pgfsetstrokecolor{currentstroke}%
\pgfsetstrokeopacity{0.000000}%
\pgfsetdash{}{0pt}%
\pgfpathmoveto{\pgfqpoint{6.295328in}{0.782881in}}%
\pgfpathlineto{\pgfqpoint{6.295328in}{5.402881in}}%
\pgfusepath{stroke}%
\end{pgfscope}%
\begin{pgfscope}%
\definecolor{textcolor}{rgb}{0.150000,0.150000,0.150000}%
\pgfsetstrokecolor{textcolor}%
\pgfsetfillcolor{textcolor}%
\pgftext[x=6.295328in,y=0.650937in,,top]{\color{textcolor}\sffamily\fontsize{19.000000}{22.800000}\selectfont 2018}%
\end{pgfscope}%
\begin{pgfscope}%
\pgfpathrectangle{\pgfqpoint{0.988475in}{0.782881in}}{\pgfqpoint{7.045455in}{4.620000in}}%
\pgfusepath{clip}%
\pgfsetroundcap%
\pgfsetroundjoin%
\pgfsetlinewidth{1.003750pt}%
\definecolor{currentstroke}{rgb}{0.800000,0.800000,0.800000}%
\pgfsetstrokecolor{currentstroke}%
\pgfsetstrokeopacity{0.000000}%
\pgfsetdash{}{0pt}%
\pgfpathmoveto{\pgfqpoint{7.002391in}{0.782881in}}%
\pgfpathlineto{\pgfqpoint{7.002391in}{5.402881in}}%
\pgfusepath{stroke}%
\end{pgfscope}%
\begin{pgfscope}%
\definecolor{textcolor}{rgb}{0.150000,0.150000,0.150000}%
\pgfsetstrokecolor{textcolor}%
\pgfsetfillcolor{textcolor}%
\pgftext[x=7.002391in,y=0.650937in,,top]{\color{textcolor}\sffamily\fontsize{19.000000}{22.800000}\selectfont 2020}%
\end{pgfscope}%
\begin{pgfscope}%
\pgfpathrectangle{\pgfqpoint{0.988475in}{0.782881in}}{\pgfqpoint{7.045455in}{4.620000in}}%
\pgfusepath{clip}%
\pgfsetroundcap%
\pgfsetroundjoin%
\pgfsetlinewidth{1.003750pt}%
\definecolor{currentstroke}{rgb}{0.800000,0.800000,0.800000}%
\pgfsetstrokecolor{currentstroke}%
\pgfsetstrokeopacity{0.000000}%
\pgfsetdash{}{0pt}%
\pgfpathmoveto{\pgfqpoint{7.710424in}{0.782881in}}%
\pgfpathlineto{\pgfqpoint{7.710424in}{5.402881in}}%
\pgfusepath{stroke}%
\end{pgfscope}%
\begin{pgfscope}%
\definecolor{textcolor}{rgb}{0.150000,0.150000,0.150000}%
\pgfsetstrokecolor{textcolor}%
\pgfsetfillcolor{textcolor}%
\pgftext[x=7.710424in,y=0.650937in,,top]{\color{textcolor}\sffamily\fontsize{19.000000}{22.800000}\selectfont 2022}%
\end{pgfscope}%
\begin{pgfscope}%
\definecolor{textcolor}{rgb}{0.150000,0.150000,0.150000}%
\pgfsetstrokecolor{textcolor}%
\pgfsetfillcolor{textcolor}%
\pgftext[x=4.511202in,y=0.354042in,,top]{\color{textcolor}\sffamily\fontsize{20.000000}{24.000000}\selectfont Date}%
\end{pgfscope}%
\begin{pgfscope}%
\pgfpathrectangle{\pgfqpoint{0.988475in}{0.782881in}}{\pgfqpoint{7.045455in}{4.620000in}}%
\pgfusepath{clip}%
\pgfsetroundcap%
\pgfsetroundjoin%
\pgfsetlinewidth{1.003750pt}%
\definecolor{currentstroke}{rgb}{0.800000,0.800000,0.800000}%
\pgfsetstrokecolor{currentstroke}%
\pgfsetstrokeopacity{0.400000}%
\pgfsetdash{}{0pt}%
\pgfpathmoveto{\pgfqpoint{0.988475in}{0.971829in}}%
\pgfpathlineto{\pgfqpoint{8.033930in}{0.971829in}}%
\pgfusepath{stroke}%
\end{pgfscope}%
\begin{pgfscope}%
\definecolor{textcolor}{rgb}{0.150000,0.150000,0.150000}%
\pgfsetstrokecolor{textcolor}%
\pgfsetfillcolor{textcolor}%
\pgftext[x=0.416243in, y=0.877380in, left, base]{\color{textcolor}\sffamily\fontsize{19.000000}{22.800000}\selectfont 100}%
\end{pgfscope}%
\begin{pgfscope}%
\pgfpathrectangle{\pgfqpoint{0.988475in}{0.782881in}}{\pgfqpoint{7.045455in}{4.620000in}}%
\pgfusepath{clip}%
\pgfsetroundcap%
\pgfsetroundjoin%
\pgfsetlinewidth{1.003750pt}%
\definecolor{currentstroke}{rgb}{0.800000,0.800000,0.800000}%
\pgfsetstrokecolor{currentstroke}%
\pgfsetstrokeopacity{0.400000}%
\pgfsetdash{}{0pt}%
\pgfpathmoveto{\pgfqpoint{0.988475in}{1.673583in}}%
\pgfpathlineto{\pgfqpoint{8.033930in}{1.673583in}}%
\pgfusepath{stroke}%
\end{pgfscope}%
\begin{pgfscope}%
\definecolor{textcolor}{rgb}{0.150000,0.150000,0.150000}%
\pgfsetstrokecolor{textcolor}%
\pgfsetfillcolor{textcolor}%
\pgftext[x=0.416243in, y=1.579134in, left, base]{\color{textcolor}\sffamily\fontsize{19.000000}{22.800000}\selectfont 150}%
\end{pgfscope}%
\begin{pgfscope}%
\pgfpathrectangle{\pgfqpoint{0.988475in}{0.782881in}}{\pgfqpoint{7.045455in}{4.620000in}}%
\pgfusepath{clip}%
\pgfsetroundcap%
\pgfsetroundjoin%
\pgfsetlinewidth{1.003750pt}%
\definecolor{currentstroke}{rgb}{0.800000,0.800000,0.800000}%
\pgfsetstrokecolor{currentstroke}%
\pgfsetstrokeopacity{0.400000}%
\pgfsetdash{}{0pt}%
\pgfpathmoveto{\pgfqpoint{0.988475in}{2.375337in}}%
\pgfpathlineto{\pgfqpoint{8.033930in}{2.375337in}}%
\pgfusepath{stroke}%
\end{pgfscope}%
\begin{pgfscope}%
\definecolor{textcolor}{rgb}{0.150000,0.150000,0.150000}%
\pgfsetstrokecolor{textcolor}%
\pgfsetfillcolor{textcolor}%
\pgftext[x=0.416243in, y=2.280889in, left, base]{\color{textcolor}\sffamily\fontsize{19.000000}{22.800000}\selectfont 200}%
\end{pgfscope}%
\begin{pgfscope}%
\pgfpathrectangle{\pgfqpoint{0.988475in}{0.782881in}}{\pgfqpoint{7.045455in}{4.620000in}}%
\pgfusepath{clip}%
\pgfsetroundcap%
\pgfsetroundjoin%
\pgfsetlinewidth{1.003750pt}%
\definecolor{currentstroke}{rgb}{0.800000,0.800000,0.800000}%
\pgfsetstrokecolor{currentstroke}%
\pgfsetstrokeopacity{0.400000}%
\pgfsetdash{}{0pt}%
\pgfpathmoveto{\pgfqpoint{0.988475in}{3.077092in}}%
\pgfpathlineto{\pgfqpoint{8.033930in}{3.077092in}}%
\pgfusepath{stroke}%
\end{pgfscope}%
\begin{pgfscope}%
\definecolor{textcolor}{rgb}{0.150000,0.150000,0.150000}%
\pgfsetstrokecolor{textcolor}%
\pgfsetfillcolor{textcolor}%
\pgftext[x=0.416243in, y=2.982643in, left, base]{\color{textcolor}\sffamily\fontsize{19.000000}{22.800000}\selectfont 250}%
\end{pgfscope}%
\begin{pgfscope}%
\pgfpathrectangle{\pgfqpoint{0.988475in}{0.782881in}}{\pgfqpoint{7.045455in}{4.620000in}}%
\pgfusepath{clip}%
\pgfsetroundcap%
\pgfsetroundjoin%
\pgfsetlinewidth{1.003750pt}%
\definecolor{currentstroke}{rgb}{0.800000,0.800000,0.800000}%
\pgfsetstrokecolor{currentstroke}%
\pgfsetstrokeopacity{0.400000}%
\pgfsetdash{}{0pt}%
\pgfpathmoveto{\pgfqpoint{0.988475in}{3.778846in}}%
\pgfpathlineto{\pgfqpoint{8.033930in}{3.778846in}}%
\pgfusepath{stroke}%
\end{pgfscope}%
\begin{pgfscope}%
\definecolor{textcolor}{rgb}{0.150000,0.150000,0.150000}%
\pgfsetstrokecolor{textcolor}%
\pgfsetfillcolor{textcolor}%
\pgftext[x=0.416243in, y=3.684398in, left, base]{\color{textcolor}\sffamily\fontsize{19.000000}{22.800000}\selectfont 300}%
\end{pgfscope}%
\begin{pgfscope}%
\pgfpathrectangle{\pgfqpoint{0.988475in}{0.782881in}}{\pgfqpoint{7.045455in}{4.620000in}}%
\pgfusepath{clip}%
\pgfsetroundcap%
\pgfsetroundjoin%
\pgfsetlinewidth{1.003750pt}%
\definecolor{currentstroke}{rgb}{0.800000,0.800000,0.800000}%
\pgfsetstrokecolor{currentstroke}%
\pgfsetstrokeopacity{0.400000}%
\pgfsetdash{}{0pt}%
\pgfpathmoveto{\pgfqpoint{0.988475in}{4.480601in}}%
\pgfpathlineto{\pgfqpoint{8.033930in}{4.480601in}}%
\pgfusepath{stroke}%
\end{pgfscope}%
\begin{pgfscope}%
\definecolor{textcolor}{rgb}{0.150000,0.150000,0.150000}%
\pgfsetstrokecolor{textcolor}%
\pgfsetfillcolor{textcolor}%
\pgftext[x=0.416243in, y=4.386152in, left, base]{\color{textcolor}\sffamily\fontsize{19.000000}{22.800000}\selectfont 350}%
\end{pgfscope}%
\begin{pgfscope}%
\pgfpathrectangle{\pgfqpoint{0.988475in}{0.782881in}}{\pgfqpoint{7.045455in}{4.620000in}}%
\pgfusepath{clip}%
\pgfsetroundcap%
\pgfsetroundjoin%
\pgfsetlinewidth{1.003750pt}%
\definecolor{currentstroke}{rgb}{0.800000,0.800000,0.800000}%
\pgfsetstrokecolor{currentstroke}%
\pgfsetstrokeopacity{0.400000}%
\pgfsetdash{}{0pt}%
\pgfpathmoveto{\pgfqpoint{0.988475in}{5.182355in}}%
\pgfpathlineto{\pgfqpoint{8.033930in}{5.182355in}}%
\pgfusepath{stroke}%
\end{pgfscope}%
\begin{pgfscope}%
\definecolor{textcolor}{rgb}{0.150000,0.150000,0.150000}%
\pgfsetstrokecolor{textcolor}%
\pgfsetfillcolor{textcolor}%
\pgftext[x=0.416243in, y=5.087906in, left, base]{\color{textcolor}\sffamily\fontsize{19.000000}{22.800000}\selectfont 400}%
\end{pgfscope}%
\begin{pgfscope}%
\definecolor{textcolor}{rgb}{0.150000,0.150000,0.150000}%
\pgfsetstrokecolor{textcolor}%
\pgfsetfillcolor{textcolor}%
\pgftext[x=0.360688in,y=3.092881in,,bottom,rotate=90.000000]{\color{textcolor}\sffamily\fontsize{20.000000}{24.000000}\selectfont Cours mensuel du blé (en €)}%
\end{pgfscope}%
\begin{pgfscope}%
\pgfpathrectangle{\pgfqpoint{0.988475in}{0.782881in}}{\pgfqpoint{7.045455in}{4.620000in}}%
\pgfusepath{clip}%
\pgfsetroundcap%
\pgfsetroundjoin%
\pgfsetlinewidth{1.505625pt}%
\definecolor{currentstroke}{rgb}{0.956863,0.309804,0.223529}%
\pgfsetstrokecolor{currentstroke}%
\pgfsetdash{}{0pt}%
\pgfpathmoveto{\pgfqpoint{0.988475in}{1.122706in}}%
\pgfpathlineto{\pgfqpoint{1.018501in}{1.084109in}}%
\pgfpathlineto{\pgfqpoint{1.045621in}{1.122706in}}%
\pgfpathlineto{\pgfqpoint{1.075647in}{1.154285in}}%
\pgfpathlineto{\pgfqpoint{1.104705in}{1.112179in}}%
\pgfpathlineto{\pgfqpoint{1.134731in}{1.168320in}}%
\pgfpathlineto{\pgfqpoint{1.163788in}{1.266565in}}%
\pgfpathlineto{\pgfqpoint{1.193814in}{1.434987in}}%
\pgfpathlineto{\pgfqpoint{1.223840in}{1.491127in}}%
\pgfpathlineto{\pgfqpoint{1.252897in}{1.659548in}}%
\pgfpathlineto{\pgfqpoint{1.282923in}{1.842004in}}%
\pgfpathlineto{\pgfqpoint{1.311981in}{1.722706in}}%
\pgfpathlineto{\pgfqpoint{1.342007in}{1.771829in}}%
\pgfpathlineto{\pgfqpoint{1.372033in}{1.715688in}}%
\pgfpathlineto{\pgfqpoint{1.400122in}{1.750776in}}%
\pgfpathlineto{\pgfqpoint{1.430148in}{1.743758in}}%
\pgfpathlineto{\pgfqpoint{1.459205in}{1.322706in}}%
\pgfpathlineto{\pgfqpoint{1.489231in}{1.266565in}}%
\pgfpathlineto{\pgfqpoint{1.518289in}{1.115688in}}%
\pgfpathlineto{\pgfqpoint{1.548315in}{1.136741in}}%
\pgfpathlineto{\pgfqpoint{1.578341in}{1.091127in}}%
\pgfpathlineto{\pgfqpoint{1.607398in}{1.091127in}}%
\pgfpathlineto{\pgfqpoint{1.637424in}{1.077092in}}%
\pgfpathlineto{\pgfqpoint{1.666481in}{1.056039in}}%
\pgfpathlineto{\pgfqpoint{1.696507in}{1.073583in}}%
\pgfpathlineto{\pgfqpoint{1.726533in}{1.098144in}}%
\pgfpathlineto{\pgfqpoint{1.753654in}{1.031478in}}%
\pgfpathlineto{\pgfqpoint{1.783680in}{0.992881in}}%
\pgfpathlineto{\pgfqpoint{1.812737in}{1.056039in}}%
\pgfpathlineto{\pgfqpoint{1.842763in}{1.024460in}}%
\pgfpathlineto{\pgfqpoint{1.871820in}{1.045513in}}%
\pgfpathlineto{\pgfqpoint{1.901846in}{1.052530in}}%
\pgfpathlineto{\pgfqpoint{1.931872in}{1.087618in}}%
\pgfpathlineto{\pgfqpoint{1.960930in}{1.101653in}}%
\pgfpathlineto{\pgfqpoint{1.990956in}{1.112179in}}%
\pgfpathlineto{\pgfqpoint{2.020013in}{1.098144in}}%
\pgfpathlineto{\pgfqpoint{2.050039in}{1.136741in}}%
\pgfpathlineto{\pgfqpoint{2.080065in}{1.133232in}}%
\pgfpathlineto{\pgfqpoint{2.107185in}{1.112179in}}%
\pgfpathlineto{\pgfqpoint{2.137211in}{1.168320in}}%
\pgfpathlineto{\pgfqpoint{2.166269in}{1.175337in}}%
\pgfpathlineto{\pgfqpoint{2.196295in}{1.140250in}}%
\pgfpathlineto{\pgfqpoint{2.225352in}{1.259548in}}%
\pgfpathlineto{\pgfqpoint{2.255378in}{1.526215in}}%
\pgfpathlineto{\pgfqpoint{2.285404in}{1.638495in}}%
\pgfpathlineto{\pgfqpoint{2.314462in}{1.757794in}}%
\pgfpathlineto{\pgfqpoint{2.344488in}{1.684109in}}%
\pgfpathlineto{\pgfqpoint{2.373545in}{1.687618in}}%
\pgfpathlineto{\pgfqpoint{2.403571in}{1.645513in}}%
\pgfpathlineto{\pgfqpoint{2.433597in}{1.708671in}}%
\pgfpathlineto{\pgfqpoint{2.460717in}{1.652530in}}%
\pgfpathlineto{\pgfqpoint{2.490743in}{1.831478in}}%
\pgfpathlineto{\pgfqpoint{2.519801in}{1.771829in}}%
\pgfpathlineto{\pgfqpoint{2.549827in}{2.157794in}}%
\pgfpathlineto{\pgfqpoint{2.578884in}{2.382355in}}%
\pgfpathlineto{\pgfqpoint{2.608910in}{3.164811in}}%
\pgfpathlineto{\pgfqpoint{2.638936in}{3.333232in}}%
\pgfpathlineto{\pgfqpoint{2.667994in}{2.824460in}}%
\pgfpathlineto{\pgfqpoint{2.698020in}{3.077092in}}%
\pgfpathlineto{\pgfqpoint{2.727077in}{3.091127in}}%
\pgfpathlineto{\pgfqpoint{2.757103in}{2.999899in}}%
\pgfpathlineto{\pgfqpoint{2.787129in}{3.554285in}}%
\pgfpathlineto{\pgfqpoint{2.815218in}{2.845513in}}%
\pgfpathlineto{\pgfqpoint{2.845244in}{2.536741in}}%
\pgfpathlineto{\pgfqpoint{2.874301in}{2.122706in}}%
\pgfpathlineto{\pgfqpoint{2.904327in}{2.340250in}}%
\pgfpathlineto{\pgfqpoint{2.933385in}{2.266565in}}%
\pgfpathlineto{\pgfqpoint{2.963411in}{2.178846in}}%
\pgfpathlineto{\pgfqpoint{2.993437in}{1.831478in}}%
\pgfpathlineto{\pgfqpoint{3.022494in}{1.589372in}}%
\pgfpathlineto{\pgfqpoint{3.052520in}{1.452530in}}%
\pgfpathlineto{\pgfqpoint{3.081577in}{1.494636in}}%
\pgfpathlineto{\pgfqpoint{3.111603in}{1.687618in}}%
\pgfpathlineto{\pgfqpoint{3.141629in}{1.505162in}}%
\pgfpathlineto{\pgfqpoint{3.168750in}{1.420951in}}%
\pgfpathlineto{\pgfqpoint{3.227833in}{1.694636in}}%
\pgfpathlineto{\pgfqpoint{3.257859in}{1.449022in}}%
\pgfpathlineto{\pgfqpoint{3.286917in}{1.357794in}}%
\pgfpathlineto{\pgfqpoint{3.316943in}{1.350776in}}%
\pgfpathlineto{\pgfqpoint{3.346969in}{1.287618in}}%
\pgfpathlineto{\pgfqpoint{3.376026in}{1.347267in}}%
\pgfpathlineto{\pgfqpoint{3.406052in}{1.413934in}}%
\pgfpathlineto{\pgfqpoint{3.435109in}{1.410425in}}%
\pgfpathlineto{\pgfqpoint{3.465135in}{1.329723in}}%
\pgfpathlineto{\pgfqpoint{3.495161in}{1.284109in}}%
\pgfpathlineto{\pgfqpoint{3.522282in}{1.326215in}}%
\pgfpathlineto{\pgfqpoint{3.552308in}{1.420951in}}%
\pgfpathlineto{\pgfqpoint{3.581365in}{1.417443in}}%
\pgfpathlineto{\pgfqpoint{3.611391in}{1.484109in}}%
\pgfpathlineto{\pgfqpoint{3.640448in}{2.178846in}}%
\pgfpathlineto{\pgfqpoint{3.670474in}{2.764811in}}%
\pgfpathlineto{\pgfqpoint{3.700500in}{2.487618in}}%
\pgfpathlineto{\pgfqpoint{3.729558in}{2.726215in}}%
\pgfpathlineto{\pgfqpoint{3.759584in}{2.712179in}}%
\pgfpathlineto{\pgfqpoint{3.788641in}{3.112179in}}%
\pgfpathlineto{\pgfqpoint{3.818667in}{3.343758in}}%
\pgfpathlineto{\pgfqpoint{3.848693in}{3.182355in}}%
\pgfpathlineto{\pgfqpoint{3.875813in}{2.936741in}}%
\pgfpathlineto{\pgfqpoint{3.905839in}{2.999899in}}%
\pgfpathlineto{\pgfqpoint{3.934897in}{2.880601in}}%
\pgfpathlineto{\pgfqpoint{3.964923in}{2.164811in}}%
\pgfpathlineto{\pgfqpoint{3.993980in}{2.357794in}}%
\pgfpathlineto{\pgfqpoint{4.024006in}{2.533232in}}%
\pgfpathlineto{\pgfqpoint{4.054032in}{2.143758in}}%
\pgfpathlineto{\pgfqpoint{4.083090in}{2.192881in}}%
\pgfpathlineto{\pgfqpoint{4.113116in}{2.073583in}}%
\pgfpathlineto{\pgfqpoint{4.142173in}{2.410425in}}%
\pgfpathlineto{\pgfqpoint{4.172199in}{2.592881in}}%
\pgfpathlineto{\pgfqpoint{4.202225in}{2.557794in}}%
\pgfpathlineto{\pgfqpoint{4.230314in}{2.554285in}}%
\pgfpathlineto{\pgfqpoint{4.260340in}{2.606916in}}%
\pgfpathlineto{\pgfqpoint{4.289397in}{2.487618in}}%
\pgfpathlineto{\pgfqpoint{4.319423in}{2.782355in}}%
\pgfpathlineto{\pgfqpoint{4.348481in}{3.217443in}}%
\pgfpathlineto{\pgfqpoint{4.378507in}{3.270074in}}%
\pgfpathlineto{\pgfqpoint{4.408533in}{3.298144in}}%
\pgfpathlineto{\pgfqpoint{4.437590in}{3.291127in}}%
\pgfpathlineto{\pgfqpoint{4.467616in}{3.350776in}}%
\pgfpathlineto{\pgfqpoint{4.496674in}{3.080601in}}%
\pgfpathlineto{\pgfqpoint{4.526700in}{3.045513in}}%
\pgfpathlineto{\pgfqpoint{4.556726in}{3.052530in}}%
\pgfpathlineto{\pgfqpoint{4.583846in}{2.919197in}}%
\pgfpathlineto{\pgfqpoint{4.613872in}{3.108671in}}%
\pgfpathlineto{\pgfqpoint{4.642929in}{2.459548in}}%
\pgfpathlineto{\pgfqpoint{4.672955in}{2.287618in}}%
\pgfpathlineto{\pgfqpoint{4.702013in}{2.231478in}}%
\pgfpathlineto{\pgfqpoint{4.732039in}{2.196390in}}%
\pgfpathlineto{\pgfqpoint{4.762065in}{2.280601in}}%
\pgfpathlineto{\pgfqpoint{4.791122in}{2.431478in}}%
\pgfpathlineto{\pgfqpoint{4.821148in}{2.512179in}}%
\pgfpathlineto{\pgfqpoint{4.850205in}{2.501653in}}%
\pgfpathlineto{\pgfqpoint{4.880231in}{2.270074in}}%
\pgfpathlineto{\pgfqpoint{4.910257in}{2.392881in}}%
\pgfpathlineto{\pgfqpoint{4.937378in}{2.484109in}}%
\pgfpathlineto{\pgfqpoint{4.967404in}{2.589372in}}%
\pgfpathlineto{\pgfqpoint{4.996461in}{2.256039in}}%
\pgfpathlineto{\pgfqpoint{5.026487in}{2.175337in}}%
\pgfpathlineto{\pgfqpoint{5.055545in}{1.961302in}}%
\pgfpathlineto{\pgfqpoint{5.085571in}{2.013934in}}%
\pgfpathlineto{\pgfqpoint{5.115597in}{1.712179in}}%
\pgfpathlineto{\pgfqpoint{5.144654in}{1.985864in}}%
\pgfpathlineto{\pgfqpoint{5.174680in}{2.154285in}}%
\pgfpathlineto{\pgfqpoint{5.203737in}{2.375337in}}%
\pgfpathlineto{\pgfqpoint{5.233763in}{2.171829in}}%
\pgfpathlineto{\pgfqpoint{5.263789in}{2.199899in}}%
\pgfpathlineto{\pgfqpoint{5.290910in}{2.199899in}}%
\pgfpathlineto{\pgfqpoint{5.320936in}{2.066565in}}%
\pgfpathlineto{\pgfqpoint{5.349993in}{2.038495in}}%
\pgfpathlineto{\pgfqpoint{5.380019in}{2.392881in}}%
\pgfpathlineto{\pgfqpoint{5.409076in}{2.101653in}}%
\pgfpathlineto{\pgfqpoint{5.439102in}{1.817443in}}%
\pgfpathlineto{\pgfqpoint{5.469128in}{2.017443in}}%
\pgfpathlineto{\pgfqpoint{5.498186in}{2.105162in}}%
\pgfpathlineto{\pgfqpoint{5.528212in}{2.049022in}}%
\pgfpathlineto{\pgfqpoint{5.557269in}{2.003408in}}%
\pgfpathlineto{\pgfqpoint{5.587295in}{1.866565in}}%
\pgfpathlineto{\pgfqpoint{5.617321in}{1.634987in}}%
\pgfpathlineto{\pgfqpoint{5.645410in}{1.712179in}}%
\pgfpathlineto{\pgfqpoint{5.675436in}{1.694636in}}%
\pgfpathlineto{\pgfqpoint{5.704493in}{1.880601in}}%
\pgfpathlineto{\pgfqpoint{5.734519in}{1.754285in}}%
\pgfpathlineto{\pgfqpoint{5.763577in}{1.898144in}}%
\pgfpathlineto{\pgfqpoint{5.793603in}{1.740250in}}%
\pgfpathlineto{\pgfqpoint{5.823629in}{1.820951in}}%
\pgfpathlineto{\pgfqpoint{5.852686in}{1.894636in}}%
\pgfpathlineto{\pgfqpoint{5.882712in}{1.842004in}}%
\pgfpathlineto{\pgfqpoint{5.911770in}{1.926215in}}%
\pgfpathlineto{\pgfqpoint{5.941796in}{1.891127in}}%
\pgfpathlineto{\pgfqpoint{5.971822in}{1.985864in}}%
\pgfpathlineto{\pgfqpoint{5.998942in}{1.877092in}}%
\pgfpathlineto{\pgfqpoint{6.028968in}{1.929723in}}%
\pgfpathlineto{\pgfqpoint{6.058025in}{1.908671in}}%
\pgfpathlineto{\pgfqpoint{6.088051in}{2.045513in}}%
\pgfpathlineto{\pgfqpoint{6.117109in}{1.929723in}}%
\pgfpathlineto{\pgfqpoint{6.147135in}{1.754285in}}%
\pgfpathlineto{\pgfqpoint{6.177161in}{1.898144in}}%
\pgfpathlineto{\pgfqpoint{6.206218in}{1.842004in}}%
\pgfpathlineto{\pgfqpoint{6.236244in}{1.806916in}}%
\pgfpathlineto{\pgfqpoint{6.265302in}{1.799899in}}%
\pgfpathlineto{\pgfqpoint{6.295328in}{1.789372in}}%
\pgfpathlineto{\pgfqpoint{6.325354in}{1.908671in}}%
\pgfpathlineto{\pgfqpoint{6.352474in}{1.866565in}}%
\pgfpathlineto{\pgfqpoint{6.382500in}{1.929723in}}%
\pgfpathlineto{\pgfqpoint{6.411557in}{2.133232in}}%
\pgfpathlineto{\pgfqpoint{6.441583in}{2.073583in}}%
\pgfpathlineto{\pgfqpoint{6.470641in}{2.406916in}}%
\pgfpathlineto{\pgfqpoint{6.500667in}{2.442004in}}%
\pgfpathlineto{\pgfqpoint{6.530693in}{2.396390in}}%
\pgfpathlineto{\pgfqpoint{6.559750in}{2.354285in}}%
\pgfpathlineto{\pgfqpoint{6.589776in}{2.392881in}}%
\pgfpathlineto{\pgfqpoint{6.618833in}{2.420951in}}%
\pgfpathlineto{\pgfqpoint{6.648859in}{2.434987in}}%
\pgfpathlineto{\pgfqpoint{6.678885in}{2.280601in}}%
\pgfpathlineto{\pgfqpoint{6.706006in}{2.175337in}}%
\pgfpathlineto{\pgfqpoint{6.736032in}{2.140250in}}%
\pgfpathlineto{\pgfqpoint{6.765089in}{2.168320in}}%
\pgfpathlineto{\pgfqpoint{6.795115in}{2.098144in}}%
\pgfpathlineto{\pgfqpoint{6.824173in}{2.020951in}}%
\pgfpathlineto{\pgfqpoint{6.854198in}{1.873583in}}%
\pgfpathlineto{\pgfqpoint{6.884224in}{2.020951in}}%
\pgfpathlineto{\pgfqpoint{6.913282in}{2.066565in}}%
\pgfpathlineto{\pgfqpoint{6.943308in}{2.171829in}}%
\pgfpathlineto{\pgfqpoint{6.972365in}{2.217443in}}%
\pgfpathlineto{\pgfqpoint{7.002391in}{2.249022in}}%
\pgfpathlineto{\pgfqpoint{7.032417in}{2.199899in}}%
\pgfpathlineto{\pgfqpoint{7.060506in}{2.322706in}}%
\pgfpathlineto{\pgfqpoint{7.090532in}{2.315688in}}%
\pgfpathlineto{\pgfqpoint{7.149616in}{2.101653in}}%
\pgfpathlineto{\pgfqpoint{7.178673in}{2.133232in}}%
\pgfpathlineto{\pgfqpoint{7.208699in}{2.203408in}}%
\pgfpathlineto{\pgfqpoint{7.238725in}{2.343758in}}%
\pgfpathlineto{\pgfqpoint{7.267782in}{2.449022in}}%
\pgfpathlineto{\pgfqpoint{7.297808in}{2.519197in}}%
\pgfpathlineto{\pgfqpoint{7.326866in}{2.561302in}}%
\pgfpathlineto{\pgfqpoint{7.356892in}{2.754285in}}%
\pgfpathlineto{\pgfqpoint{7.386918in}{3.006916in}}%
\pgfpathlineto{\pgfqpoint{7.414038in}{2.592881in}}%
\pgfpathlineto{\pgfqpoint{7.444064in}{3.185864in}}%
\pgfpathlineto{\pgfqpoint{7.473121in}{2.568320in}}%
\pgfpathlineto{\pgfqpoint{7.503147in}{2.505162in}}%
\pgfpathlineto{\pgfqpoint{7.532205in}{2.701653in}}%
\pgfpathlineto{\pgfqpoint{7.562231in}{3.059548in}}%
\pgfpathlineto{\pgfqpoint{7.592257in}{3.189372in}}%
\pgfpathlineto{\pgfqpoint{7.621314in}{3.543758in}}%
\pgfpathlineto{\pgfqpoint{7.651340in}{3.491127in}}%
\pgfpathlineto{\pgfqpoint{7.680398in}{3.477092in}}%
\pgfpathlineto{\pgfqpoint{7.710424in}{3.301653in}}%
\pgfpathlineto{\pgfqpoint{7.740450in}{4.094636in}}%
\pgfpathlineto{\pgfqpoint{7.767570in}{4.754285in}}%
\pgfpathlineto{\pgfqpoint{7.797596in}{5.192881in}}%
\pgfpathlineto{\pgfqpoint{7.826653in}{5.073583in}}%
\pgfpathlineto{\pgfqpoint{7.856679in}{4.484109in}}%
\pgfpathlineto{\pgfqpoint{7.885737in}{4.382355in}}%
\pgfpathlineto{\pgfqpoint{7.915763in}{4.231478in}}%
\pgfpathlineto{\pgfqpoint{7.945789in}{4.575337in}}%
\pgfpathlineto{\pgfqpoint{7.974846in}{4.512179in}}%
\pgfpathlineto{\pgfqpoint{8.004872in}{4.150776in}}%
\pgfpathlineto{\pgfqpoint{8.033930in}{3.908671in}}%
\pgfpathlineto{\pgfqpoint{8.033930in}{3.908671in}}%
\pgfusepath{stroke}%
\end{pgfscope}%
\begin{pgfscope}%
\pgfsetrectcap%
\pgfsetmiterjoin%
\pgfsetlinewidth{1.254687pt}%
\definecolor{currentstroke}{rgb}{0.150000,0.150000,0.150000}%
\pgfsetstrokecolor{currentstroke}%
\pgfsetdash{}{0pt}%
\pgfpathmoveto{\pgfqpoint{0.988475in}{0.782881in}}%
\pgfpathlineto{\pgfqpoint{0.988475in}{5.402881in}}%
\pgfusepath{stroke}%
\end{pgfscope}%
\begin{pgfscope}%
\pgfsetrectcap%
\pgfsetmiterjoin%
\pgfsetlinewidth{1.254687pt}%
\definecolor{currentstroke}{rgb}{0.150000,0.150000,0.150000}%
\pgfsetstrokecolor{currentstroke}%
\pgfsetdash{}{0pt}%
\pgfpathmoveto{\pgfqpoint{8.033930in}{0.782881in}}%
\pgfpathlineto{\pgfqpoint{8.033930in}{5.402881in}}%
\pgfusepath{stroke}%
\end{pgfscope}%
\begin{pgfscope}%
\pgfsetrectcap%
\pgfsetmiterjoin%
\pgfsetlinewidth{1.254687pt}%
\definecolor{currentstroke}{rgb}{0.150000,0.150000,0.150000}%
\pgfsetstrokecolor{currentstroke}%
\pgfsetdash{}{0pt}%
\pgfpathmoveto{\pgfqpoint{0.988475in}{0.782881in}}%
\pgfpathlineto{\pgfqpoint{8.033930in}{0.782881in}}%
\pgfusepath{stroke}%
\end{pgfscope}%
\begin{pgfscope}%
\pgfsetrectcap%
\pgfsetmiterjoin%
\pgfsetlinewidth{1.254687pt}%
\definecolor{currentstroke}{rgb}{0.150000,0.150000,0.150000}%
\pgfsetstrokecolor{currentstroke}%
\pgfsetdash{}{0pt}%
\pgfpathmoveto{\pgfqpoint{0.988475in}{5.402881in}}%
\pgfpathlineto{\pgfqpoint{8.033930in}{5.402881in}}%
\pgfusepath{stroke}%
\end{pgfscope}%
\begin{pgfscope}%
\pgfsetbuttcap%
\pgfsetmiterjoin%
\definecolor{currentfill}{rgb}{1.000000,1.000000,1.000000}%
\pgfsetfillcolor{currentfill}%
\pgfsetlinewidth{0.000000pt}%
\definecolor{currentstroke}{rgb}{0.000000,0.000000,0.000000}%
\pgfsetstrokecolor{currentstroke}%
\pgfsetstrokeopacity{0.000000}%
\pgfsetdash{}{0pt}%
\pgfpathmoveto{\pgfqpoint{9.443020in}{0.782881in}}%
\pgfpathlineto{\pgfqpoint{16.488475in}{0.782881in}}%
\pgfpathlineto{\pgfqpoint{16.488475in}{5.402881in}}%
\pgfpathlineto{\pgfqpoint{9.443020in}{5.402881in}}%
\pgfpathlineto{\pgfqpoint{9.443020in}{0.782881in}}%
\pgfpathclose%
\pgfusepath{fill}%
\end{pgfscope}%
\begin{pgfscope}%
\pgfpathrectangle{\pgfqpoint{9.443020in}{0.782881in}}{\pgfqpoint{7.045455in}{4.620000in}}%
\pgfusepath{clip}%
\pgfsetroundcap%
\pgfsetroundjoin%
\pgfsetlinewidth{1.003750pt}%
\definecolor{currentstroke}{rgb}{0.800000,0.800000,0.800000}%
\pgfsetstrokecolor{currentstroke}%
\pgfsetstrokeopacity{0.000000}%
\pgfsetdash{}{0pt}%
\pgfpathmoveto{\pgfqpoint{9.763268in}{0.782881in}}%
\pgfpathlineto{\pgfqpoint{9.763268in}{5.402881in}}%
\pgfusepath{stroke}%
\end{pgfscope}%
\begin{pgfscope}%
\definecolor{textcolor}{rgb}{0.150000,0.150000,0.150000}%
\pgfsetstrokecolor{textcolor}%
\pgfsetfillcolor{textcolor}%
\pgftext[x=9.763268in,y=0.650937in,,top]{\color{textcolor}\sffamily\fontsize{19.000000}{22.800000}\selectfont 2006}%
\end{pgfscope}%
\begin{pgfscope}%
\pgfpathrectangle{\pgfqpoint{9.443020in}{0.782881in}}{\pgfqpoint{7.045455in}{4.620000in}}%
\pgfusepath{clip}%
\pgfsetroundcap%
\pgfsetroundjoin%
\pgfsetlinewidth{1.003750pt}%
\definecolor{currentstroke}{rgb}{0.800000,0.800000,0.800000}%
\pgfsetstrokecolor{currentstroke}%
\pgfsetstrokeopacity{0.000000}%
\pgfsetdash{}{0pt}%
\pgfpathmoveto{\pgfqpoint{10.520086in}{0.782881in}}%
\pgfpathlineto{\pgfqpoint{10.520086in}{5.402881in}}%
\pgfusepath{stroke}%
\end{pgfscope}%
\begin{pgfscope}%
\definecolor{textcolor}{rgb}{0.150000,0.150000,0.150000}%
\pgfsetstrokecolor{textcolor}%
\pgfsetfillcolor{textcolor}%
\pgftext[x=10.520086in,y=0.650937in,,top]{\color{textcolor}\sffamily\fontsize{19.000000}{22.800000}\selectfont 2008}%
\end{pgfscope}%
\begin{pgfscope}%
\pgfpathrectangle{\pgfqpoint{9.443020in}{0.782881in}}{\pgfqpoint{7.045455in}{4.620000in}}%
\pgfusepath{clip}%
\pgfsetroundcap%
\pgfsetroundjoin%
\pgfsetlinewidth{1.003750pt}%
\definecolor{currentstroke}{rgb}{0.800000,0.800000,0.800000}%
\pgfsetstrokecolor{currentstroke}%
\pgfsetstrokeopacity{0.000000}%
\pgfsetdash{}{0pt}%
\pgfpathmoveto{\pgfqpoint{11.277941in}{0.782881in}}%
\pgfpathlineto{\pgfqpoint{11.277941in}{5.402881in}}%
\pgfusepath{stroke}%
\end{pgfscope}%
\begin{pgfscope}%
\definecolor{textcolor}{rgb}{0.150000,0.150000,0.150000}%
\pgfsetstrokecolor{textcolor}%
\pgfsetfillcolor{textcolor}%
\pgftext[x=11.277941in,y=0.650937in,,top]{\color{textcolor}\sffamily\fontsize{19.000000}{22.800000}\selectfont 2010}%
\end{pgfscope}%
\begin{pgfscope}%
\pgfpathrectangle{\pgfqpoint{9.443020in}{0.782881in}}{\pgfqpoint{7.045455in}{4.620000in}}%
\pgfusepath{clip}%
\pgfsetroundcap%
\pgfsetroundjoin%
\pgfsetlinewidth{1.003750pt}%
\definecolor{currentstroke}{rgb}{0.800000,0.800000,0.800000}%
\pgfsetstrokecolor{currentstroke}%
\pgfsetstrokeopacity{0.000000}%
\pgfsetdash{}{0pt}%
\pgfpathmoveto{\pgfqpoint{12.034758in}{0.782881in}}%
\pgfpathlineto{\pgfqpoint{12.034758in}{5.402881in}}%
\pgfusepath{stroke}%
\end{pgfscope}%
\begin{pgfscope}%
\definecolor{textcolor}{rgb}{0.150000,0.150000,0.150000}%
\pgfsetstrokecolor{textcolor}%
\pgfsetfillcolor{textcolor}%
\pgftext[x=12.034758in,y=0.650937in,,top]{\color{textcolor}\sffamily\fontsize{19.000000}{22.800000}\selectfont 2012}%
\end{pgfscope}%
\begin{pgfscope}%
\pgfpathrectangle{\pgfqpoint{9.443020in}{0.782881in}}{\pgfqpoint{7.045455in}{4.620000in}}%
\pgfusepath{clip}%
\pgfsetroundcap%
\pgfsetroundjoin%
\pgfsetlinewidth{1.003750pt}%
\definecolor{currentstroke}{rgb}{0.800000,0.800000,0.800000}%
\pgfsetstrokecolor{currentstroke}%
\pgfsetstrokeopacity{0.000000}%
\pgfsetdash{}{0pt}%
\pgfpathmoveto{\pgfqpoint{12.792613in}{0.782881in}}%
\pgfpathlineto{\pgfqpoint{12.792613in}{5.402881in}}%
\pgfusepath{stroke}%
\end{pgfscope}%
\begin{pgfscope}%
\definecolor{textcolor}{rgb}{0.150000,0.150000,0.150000}%
\pgfsetstrokecolor{textcolor}%
\pgfsetfillcolor{textcolor}%
\pgftext[x=12.792613in,y=0.650937in,,top]{\color{textcolor}\sffamily\fontsize{19.000000}{22.800000}\selectfont 2014}%
\end{pgfscope}%
\begin{pgfscope}%
\pgfpathrectangle{\pgfqpoint{9.443020in}{0.782881in}}{\pgfqpoint{7.045455in}{4.620000in}}%
\pgfusepath{clip}%
\pgfsetroundcap%
\pgfsetroundjoin%
\pgfsetlinewidth{1.003750pt}%
\definecolor{currentstroke}{rgb}{0.800000,0.800000,0.800000}%
\pgfsetstrokecolor{currentstroke}%
\pgfsetstrokeopacity{0.000000}%
\pgfsetdash{}{0pt}%
\pgfpathmoveto{\pgfqpoint{13.549430in}{0.782881in}}%
\pgfpathlineto{\pgfqpoint{13.549430in}{5.402881in}}%
\pgfusepath{stroke}%
\end{pgfscope}%
\begin{pgfscope}%
\definecolor{textcolor}{rgb}{0.150000,0.150000,0.150000}%
\pgfsetstrokecolor{textcolor}%
\pgfsetfillcolor{textcolor}%
\pgftext[x=13.549430in,y=0.650937in,,top]{\color{textcolor}\sffamily\fontsize{19.000000}{22.800000}\selectfont 2016}%
\end{pgfscope}%
\begin{pgfscope}%
\pgfpathrectangle{\pgfqpoint{9.443020in}{0.782881in}}{\pgfqpoint{7.045455in}{4.620000in}}%
\pgfusepath{clip}%
\pgfsetroundcap%
\pgfsetroundjoin%
\pgfsetlinewidth{1.003750pt}%
\definecolor{currentstroke}{rgb}{0.800000,0.800000,0.800000}%
\pgfsetstrokecolor{currentstroke}%
\pgfsetstrokeopacity{0.000000}%
\pgfsetdash{}{0pt}%
\pgfpathmoveto{\pgfqpoint{14.307285in}{0.782881in}}%
\pgfpathlineto{\pgfqpoint{14.307285in}{5.402881in}}%
\pgfusepath{stroke}%
\end{pgfscope}%
\begin{pgfscope}%
\definecolor{textcolor}{rgb}{0.150000,0.150000,0.150000}%
\pgfsetstrokecolor{textcolor}%
\pgfsetfillcolor{textcolor}%
\pgftext[x=14.307285in,y=0.650937in,,top]{\color{textcolor}\sffamily\fontsize{19.000000}{22.800000}\selectfont 2018}%
\end{pgfscope}%
\begin{pgfscope}%
\pgfpathrectangle{\pgfqpoint{9.443020in}{0.782881in}}{\pgfqpoint{7.045455in}{4.620000in}}%
\pgfusepath{clip}%
\pgfsetroundcap%
\pgfsetroundjoin%
\pgfsetlinewidth{1.003750pt}%
\definecolor{currentstroke}{rgb}{0.800000,0.800000,0.800000}%
\pgfsetstrokecolor{currentstroke}%
\pgfsetstrokeopacity{0.000000}%
\pgfsetdash{}{0pt}%
\pgfpathmoveto{\pgfqpoint{15.064103in}{0.782881in}}%
\pgfpathlineto{\pgfqpoint{15.064103in}{5.402881in}}%
\pgfusepath{stroke}%
\end{pgfscope}%
\begin{pgfscope}%
\definecolor{textcolor}{rgb}{0.150000,0.150000,0.150000}%
\pgfsetstrokecolor{textcolor}%
\pgfsetfillcolor{textcolor}%
\pgftext[x=15.064103in,y=0.650937in,,top]{\color{textcolor}\sffamily\fontsize{19.000000}{22.800000}\selectfont 2020}%
\end{pgfscope}%
\begin{pgfscope}%
\pgfpathrectangle{\pgfqpoint{9.443020in}{0.782881in}}{\pgfqpoint{7.045455in}{4.620000in}}%
\pgfusepath{clip}%
\pgfsetroundcap%
\pgfsetroundjoin%
\pgfsetlinewidth{1.003750pt}%
\definecolor{currentstroke}{rgb}{0.800000,0.800000,0.800000}%
\pgfsetstrokecolor{currentstroke}%
\pgfsetstrokeopacity{0.000000}%
\pgfsetdash{}{0pt}%
\pgfpathmoveto{\pgfqpoint{15.821957in}{0.782881in}}%
\pgfpathlineto{\pgfqpoint{15.821957in}{5.402881in}}%
\pgfusepath{stroke}%
\end{pgfscope}%
\begin{pgfscope}%
\definecolor{textcolor}{rgb}{0.150000,0.150000,0.150000}%
\pgfsetstrokecolor{textcolor}%
\pgfsetfillcolor{textcolor}%
\pgftext[x=15.821957in,y=0.650937in,,top]{\color{textcolor}\sffamily\fontsize{19.000000}{22.800000}\selectfont 2022}%
\end{pgfscope}%
\begin{pgfscope}%
\definecolor{textcolor}{rgb}{0.150000,0.150000,0.150000}%
\pgfsetstrokecolor{textcolor}%
\pgfsetfillcolor{textcolor}%
\pgftext[x=12.965748in,y=0.354042in,,top]{\color{textcolor}\sffamily\fontsize{20.000000}{24.000000}\selectfont Date}%
\end{pgfscope}%
\begin{pgfscope}%
\pgfpathrectangle{\pgfqpoint{9.443020in}{0.782881in}}{\pgfqpoint{7.045455in}{4.620000in}}%
\pgfusepath{clip}%
\pgfsetroundcap%
\pgfsetroundjoin%
\pgfsetlinewidth{1.003750pt}%
\definecolor{currentstroke}{rgb}{0.800000,0.800000,0.800000}%
\pgfsetstrokecolor{currentstroke}%
\pgfsetstrokeopacity{0.400000}%
\pgfsetdash{}{0pt}%
\pgfpathmoveto{\pgfqpoint{9.443020in}{1.152265in}}%
\pgfpathlineto{\pgfqpoint{16.488475in}{1.152265in}}%
\pgfusepath{stroke}%
\end{pgfscope}%
\begin{pgfscope}%
\definecolor{textcolor}{rgb}{0.150000,0.150000,0.150000}%
\pgfsetstrokecolor{textcolor}%
\pgfsetfillcolor{textcolor}%
\pgftext[x=8.577264in, y=1.057817in, left, base]{\color{textcolor}\sffamily\fontsize{19.000000}{22.800000}\selectfont 10000}%
\end{pgfscope}%
\begin{pgfscope}%
\pgfpathrectangle{\pgfqpoint{9.443020in}{0.782881in}}{\pgfqpoint{7.045455in}{4.620000in}}%
\pgfusepath{clip}%
\pgfsetroundcap%
\pgfsetroundjoin%
\pgfsetlinewidth{1.003750pt}%
\definecolor{currentstroke}{rgb}{0.800000,0.800000,0.800000}%
\pgfsetstrokecolor{currentstroke}%
\pgfsetstrokeopacity{0.400000}%
\pgfsetdash{}{0pt}%
\pgfpathmoveto{\pgfqpoint{9.443020in}{2.170694in}}%
\pgfpathlineto{\pgfqpoint{16.488475in}{2.170694in}}%
\pgfusepath{stroke}%
\end{pgfscope}%
\begin{pgfscope}%
\definecolor{textcolor}{rgb}{0.150000,0.150000,0.150000}%
\pgfsetstrokecolor{textcolor}%
\pgfsetfillcolor{textcolor}%
\pgftext[x=8.577264in, y=2.076246in, left, base]{\color{textcolor}\sffamily\fontsize{19.000000}{22.800000}\selectfont 20000}%
\end{pgfscope}%
\begin{pgfscope}%
\pgfpathrectangle{\pgfqpoint{9.443020in}{0.782881in}}{\pgfqpoint{7.045455in}{4.620000in}}%
\pgfusepath{clip}%
\pgfsetroundcap%
\pgfsetroundjoin%
\pgfsetlinewidth{1.003750pt}%
\definecolor{currentstroke}{rgb}{0.800000,0.800000,0.800000}%
\pgfsetstrokecolor{currentstroke}%
\pgfsetstrokeopacity{0.400000}%
\pgfsetdash{}{0pt}%
\pgfpathmoveto{\pgfqpoint{9.443020in}{3.189123in}}%
\pgfpathlineto{\pgfqpoint{16.488475in}{3.189123in}}%
\pgfusepath{stroke}%
\end{pgfscope}%
\begin{pgfscope}%
\definecolor{textcolor}{rgb}{0.150000,0.150000,0.150000}%
\pgfsetstrokecolor{textcolor}%
\pgfsetfillcolor{textcolor}%
\pgftext[x=8.577264in, y=3.094674in, left, base]{\color{textcolor}\sffamily\fontsize{19.000000}{22.800000}\selectfont 30000}%
\end{pgfscope}%
\begin{pgfscope}%
\pgfpathrectangle{\pgfqpoint{9.443020in}{0.782881in}}{\pgfqpoint{7.045455in}{4.620000in}}%
\pgfusepath{clip}%
\pgfsetroundcap%
\pgfsetroundjoin%
\pgfsetlinewidth{1.003750pt}%
\definecolor{currentstroke}{rgb}{0.800000,0.800000,0.800000}%
\pgfsetstrokecolor{currentstroke}%
\pgfsetstrokeopacity{0.400000}%
\pgfsetdash{}{0pt}%
\pgfpathmoveto{\pgfqpoint{9.443020in}{4.207551in}}%
\pgfpathlineto{\pgfqpoint{16.488475in}{4.207551in}}%
\pgfusepath{stroke}%
\end{pgfscope}%
\begin{pgfscope}%
\definecolor{textcolor}{rgb}{0.150000,0.150000,0.150000}%
\pgfsetstrokecolor{textcolor}%
\pgfsetfillcolor{textcolor}%
\pgftext[x=8.577264in, y=4.113103in, left, base]{\color{textcolor}\sffamily\fontsize{19.000000}{22.800000}\selectfont 40000}%
\end{pgfscope}%
\begin{pgfscope}%
\pgfpathrectangle{\pgfqpoint{9.443020in}{0.782881in}}{\pgfqpoint{7.045455in}{4.620000in}}%
\pgfusepath{clip}%
\pgfsetroundcap%
\pgfsetroundjoin%
\pgfsetlinewidth{1.003750pt}%
\definecolor{currentstroke}{rgb}{0.800000,0.800000,0.800000}%
\pgfsetstrokecolor{currentstroke}%
\pgfsetstrokeopacity{0.400000}%
\pgfsetdash{}{0pt}%
\pgfpathmoveto{\pgfqpoint{9.443020in}{5.225980in}}%
\pgfpathlineto{\pgfqpoint{16.488475in}{5.225980in}}%
\pgfusepath{stroke}%
\end{pgfscope}%
\begin{pgfscope}%
\definecolor{textcolor}{rgb}{0.150000,0.150000,0.150000}%
\pgfsetstrokecolor{textcolor}%
\pgfsetfillcolor{textcolor}%
\pgftext[x=8.577264in, y=5.131532in, left, base]{\color{textcolor}\sffamily\fontsize{19.000000}{22.800000}\selectfont 50000}%
\end{pgfscope}%
\begin{pgfscope}%
\definecolor{textcolor}{rgb}{0.150000,0.150000,0.150000}%
\pgfsetstrokecolor{textcolor}%
\pgfsetfillcolor{textcolor}%
\pgftext[x=8.521708in,y=3.092881in,,bottom,rotate=90.000000]{\color{textcolor}\sffamily\fontsize{20.000000}{24.000000}\selectfont Cours mensuel du Nickel (en \(\displaystyle )\)}%
\end{pgfscope}%
\begin{pgfscope}%
\pgfpathrectangle{\pgfqpoint{9.443020in}{0.782881in}}{\pgfqpoint{7.045455in}{4.620000in}}%
\pgfusepath{clip}%
\pgfsetroundcap%
\pgfsetroundjoin%
\pgfsetlinewidth{1.505625pt}%
\definecolor{currentstroke}{rgb}{0.325490,0.619608,0.803922}%
\pgfsetstrokecolor{currentstroke}%
\pgfsetdash{}{0pt}%
\pgfpathmoveto{\pgfqpoint{9.763268in}{1.672682in}}%
\pgfpathlineto{\pgfqpoint{9.795407in}{1.646000in}}%
\pgfpathlineto{\pgfqpoint{9.824436in}{1.680422in}}%
\pgfpathlineto{\pgfqpoint{9.856575in}{2.109079in}}%
\pgfpathlineto{\pgfqpoint{9.887677in}{2.430903in}}%
\pgfpathlineto{\pgfqpoint{9.919816in}{2.387110in}}%
\pgfpathlineto{\pgfqpoint{9.950918in}{2.870864in}}%
\pgfpathlineto{\pgfqpoint{9.983057in}{3.321518in}}%
\pgfpathlineto{\pgfqpoint{10.015195in}{3.290966in}}%
\pgfpathlineto{\pgfqpoint{10.046297in}{3.431000in}}%
\pgfpathlineto{\pgfqpoint{10.078436in}{3.652508in}}%
\pgfpathlineto{\pgfqpoint{10.109538in}{3.599040in}}%
\pgfpathlineto{\pgfqpoint{10.141677in}{4.108255in}}%
\pgfpathlineto{\pgfqpoint{10.173816in}{4.594045in}}%
\pgfpathlineto{\pgfqpoint{10.202845in}{4.894991in}}%
\pgfpathlineto{\pgfqpoint{10.234984in}{5.192881in}}%
\pgfpathlineto{\pgfqpoint{10.266086in}{4.966281in}}%
\pgfpathlineto{\pgfqpoint{10.298224in}{3.833279in}}%
\pgfpathlineto{\pgfqpoint{10.329327in}{3.345452in}}%
\pgfpathlineto{\pgfqpoint{10.361465in}{3.148386in}}%
\pgfpathlineto{\pgfqpoint{10.393604in}{3.220185in}}%
\pgfpathlineto{\pgfqpoint{10.424706in}{3.369894in}}%
\pgfpathlineto{\pgfqpoint{10.456845in}{2.865262in}}%
\pgfpathlineto{\pgfqpoint{10.487947in}{2.787862in}}%
\pgfpathlineto{\pgfqpoint{10.520086in}{2.906000in}}%
\pgfpathlineto{\pgfqpoint{10.552225in}{3.328138in}}%
\pgfpathlineto{\pgfqpoint{10.582290in}{3.145330in}}%
\pgfpathlineto{\pgfqpoint{10.614429in}{3.028720in}}%
\pgfpathlineto{\pgfqpoint{10.645531in}{2.374176in}}%
\pgfpathlineto{\pgfqpoint{10.677670in}{2.356557in}}%
\pgfpathlineto{\pgfqpoint{10.708772in}{1.994506in}}%
\pgfpathlineto{\pgfqpoint{10.740911in}{2.190553in}}%
\pgfpathlineto{\pgfqpoint{10.773050in}{1.728696in}}%
\pgfpathlineto{\pgfqpoint{10.804152in}{1.352081in}}%
\pgfpathlineto{\pgfqpoint{10.836291in}{1.163570in}}%
\pgfpathlineto{\pgfqpoint{10.867393in}{1.136989in}}%
\pgfpathlineto{\pgfqpoint{10.899532in}{1.284661in}}%
\pgfpathlineto{\pgfqpoint{10.931671in}{1.144627in}}%
\pgfpathlineto{\pgfqpoint{10.960699in}{1.131897in}}%
\pgfpathlineto{\pgfqpoint{10.992838in}{1.318269in}}%
\pgfpathlineto{\pgfqpoint{11.023940in}{1.529084in}}%
\pgfpathlineto{\pgfqpoint{11.056079in}{1.703337in}}%
\pgfpathlineto{\pgfqpoint{11.087181in}{1.951732in}}%
\pgfpathlineto{\pgfqpoint{11.119320in}{2.081072in}}%
\pgfpathlineto{\pgfqpoint{11.151459in}{1.946640in}}%
\pgfpathlineto{\pgfqpoint{11.182561in}{1.987377in}}%
\pgfpathlineto{\pgfqpoint{11.214700in}{1.799477in}}%
\pgfpathlineto{\pgfqpoint{11.245802in}{2.053371in}}%
\pgfpathlineto{\pgfqpoint{11.277941in}{2.028114in}}%
\pgfpathlineto{\pgfqpoint{11.310079in}{2.288424in}}%
\pgfpathlineto{\pgfqpoint{11.339108in}{2.685001in}}%
\pgfpathlineto{\pgfqpoint{11.371247in}{2.827581in}}%
\pgfpathlineto{\pgfqpoint{11.402349in}{2.277629in}}%
\pgfpathlineto{\pgfqpoint{11.434488in}{2.135049in}}%
\pgfpathlineto{\pgfqpoint{11.465590in}{2.277629in}}%
\pgfpathlineto{\pgfqpoint{11.497729in}{2.243512in}}%
\pgfpathlineto{\pgfqpoint{11.529868in}{2.508812in}}%
\pgfpathlineto{\pgfqpoint{11.560970in}{2.477241in}}%
\pgfpathlineto{\pgfqpoint{11.593108in}{2.460437in}}%
\pgfpathlineto{\pgfqpoint{11.624211in}{2.674816in}}%
\pgfpathlineto{\pgfqpoint{11.656349in}{2.914147in}}%
\pgfpathlineto{\pgfqpoint{11.688488in}{3.084734in}}%
\pgfpathlineto{\pgfqpoint{11.717517in}{2.795500in}}%
\pgfpathlineto{\pgfqpoint{11.749656in}{2.868318in}}%
\pgfpathlineto{\pgfqpoint{11.780758in}{2.522052in}}%
\pgfpathlineto{\pgfqpoint{11.812897in}{2.519506in}}%
\pgfpathlineto{\pgfqpoint{11.843999in}{2.655059in}}%
\pgfpathlineto{\pgfqpoint{11.876138in}{2.399331in}}%
\pgfpathlineto{\pgfqpoint{11.908276in}{1.928817in}}%
\pgfpathlineto{\pgfqpoint{11.939378in}{2.127411in}}%
\pgfpathlineto{\pgfqpoint{11.971517in}{1.916087in}}%
\pgfpathlineto{\pgfqpoint{12.002619in}{2.039317in}}%
\pgfpathlineto{\pgfqpoint{12.034758in}{2.257770in}}%
\pgfpathlineto{\pgfqpoint{12.066897in}{2.094821in}}%
\pgfpathlineto{\pgfqpoint{12.096962in}{1.949186in}}%
\pgfpathlineto{\pgfqpoint{12.129101in}{1.956315in}}%
\pgfpathlineto{\pgfqpoint{12.160203in}{1.786746in}}%
\pgfpathlineto{\pgfqpoint{12.192342in}{1.837668in}}%
\pgfpathlineto{\pgfqpoint{12.223444in}{1.749574in}}%
\pgfpathlineto{\pgfqpoint{12.255583in}{1.758230in}}%
\pgfpathlineto{\pgfqpoint{12.287722in}{2.015384in}}%
\pgfpathlineto{\pgfqpoint{12.318824in}{1.783182in}}%
\pgfpathlineto{\pgfqpoint{12.350963in}{1.931363in}}%
\pgfpathlineto{\pgfqpoint{12.382065in}{1.871276in}}%
\pgfpathlineto{\pgfqpoint{12.414204in}{2.000107in}}%
\pgfpathlineto{\pgfqpoint{12.446343in}{1.824428in}}%
\pgfpathlineto{\pgfqpoint{12.475371in}{1.830539in}}%
\pgfpathlineto{\pgfqpoint{12.507510in}{1.701198in}}%
\pgfpathlineto{\pgfqpoint{12.538612in}{1.643657in}}%
\pgfpathlineto{\pgfqpoint{12.570751in}{1.530102in}}%
\pgfpathlineto{\pgfqpoint{12.601853in}{1.546906in}}%
\pgfpathlineto{\pgfqpoint{12.633992in}{1.539268in}}%
\pgfpathlineto{\pgfqpoint{12.666131in}{1.555054in}}%
\pgfpathlineto{\pgfqpoint{12.697233in}{1.622270in}}%
\pgfpathlineto{\pgfqpoint{12.729372in}{1.510243in}}%
\pgfpathlineto{\pgfqpoint{12.760474in}{1.549453in}}%
\pgfpathlineto{\pgfqpoint{12.792613in}{1.557600in}}%
\pgfpathlineto{\pgfqpoint{12.824752in}{1.632964in}}%
\pgfpathlineto{\pgfqpoint{12.853780in}{1.753138in}}%
\pgfpathlineto{\pgfqpoint{12.885919in}{2.000107in}}%
\pgfpathlineto{\pgfqpoint{12.917021in}{2.094312in}}%
\pgfpathlineto{\pgfqpoint{12.949160in}{2.072925in}}%
\pgfpathlineto{\pgfqpoint{12.980262in}{2.018439in}}%
\pgfpathlineto{\pgfqpoint{13.012401in}{2.048483in}}%
\pgfpathlineto{\pgfqpoint{13.044540in}{1.794894in}}%
\pgfpathlineto{\pgfqpoint{13.075642in}{1.740917in}}%
\pgfpathlineto{\pgfqpoint{13.107781in}{1.791329in}}%
\pgfpathlineto{\pgfqpoint{13.138883in}{1.676756in}}%
\pgfpathlineto{\pgfqpoint{13.171022in}{1.678284in}}%
\pgfpathlineto{\pgfqpoint{13.203160in}{1.569312in}}%
\pgfpathlineto{\pgfqpoint{13.232189in}{1.396179in}}%
\pgfpathlineto{\pgfqpoint{13.264328in}{1.554545in}}%
\pgfpathlineto{\pgfqpoint{13.295430in}{1.419094in}}%
\pgfpathlineto{\pgfqpoint{13.327569in}{1.353914in}}%
\pgfpathlineto{\pgfqpoint{13.390810in}{1.158376in}}%
\pgfpathlineto{\pgfqpoint{13.422949in}{1.193002in}}%
\pgfpathlineto{\pgfqpoint{13.454051in}{1.158376in}}%
\pgfpathlineto{\pgfqpoint{13.486189in}{1.040238in}}%
\pgfpathlineto{\pgfqpoint{13.517292in}{1.032091in}}%
\pgfpathlineto{\pgfqpoint{13.549430in}{1.011722in}}%
\pgfpathlineto{\pgfqpoint{13.581569in}{1.001538in}}%
\pgfpathlineto{\pgfqpoint{13.611635in}{0.998483in}}%
\pgfpathlineto{\pgfqpoint{13.643773in}{1.095743in}}%
\pgfpathlineto{\pgfqpoint{13.674876in}{0.992881in}}%
\pgfpathlineto{\pgfqpoint{13.707014in}{1.095743in}}%
\pgfpathlineto{\pgfqpoint{13.738116in}{1.216426in}}%
\pgfpathlineto{\pgfqpoint{13.770255in}{1.128332in}}%
\pgfpathlineto{\pgfqpoint{13.802394in}{1.210825in}}%
\pgfpathlineto{\pgfqpoint{13.833496in}{1.200641in}}%
\pgfpathlineto{\pgfqpoint{13.865635in}{1.279569in}}%
\pgfpathlineto{\pgfqpoint{13.896737in}{1.154302in}}%
\pgfpathlineto{\pgfqpoint{13.928876in}{1.147682in}}%
\pgfpathlineto{\pgfqpoint{13.961015in}{1.252071in}}%
\pgfpathlineto{\pgfqpoint{13.990043in}{1.154811in}}%
\pgfpathlineto{\pgfqpoint{14.022182in}{1.096252in}}%
\pgfpathlineto{\pgfqpoint{14.053284in}{1.047367in}}%
\pgfpathlineto{\pgfqpoint{14.085423in}{1.090141in}}%
\pgfpathlineto{\pgfqpoint{14.116525in}{1.174162in}}%
\pgfpathlineto{\pgfqpoint{14.148664in}{1.335583in}}%
\pgfpathlineto{\pgfqpoint{14.180803in}{1.203187in}}%
\pgfpathlineto{\pgfqpoint{14.211905in}{1.385995in}}%
\pgfpathlineto{\pgfqpoint{14.244044in}{1.265311in}}%
\pgfpathlineto{\pgfqpoint{14.275146in}{1.433352in}}%
\pgfpathlineto{\pgfqpoint{14.307285in}{1.518900in}}%
\pgfpathlineto{\pgfqpoint{14.339424in}{1.538250in}}%
\pgfpathlineto{\pgfqpoint{14.368452in}{1.488347in}}%
\pgfpathlineto{\pgfqpoint{14.400591in}{1.523992in}}%
\pgfpathlineto{\pgfqpoint{14.431693in}{1.683885in}}%
\pgfpathlineto{\pgfqpoint{14.463832in}{1.651295in}}%
\pgfpathlineto{\pgfqpoint{14.494934in}{1.562692in}}%
\pgfpathlineto{\pgfqpoint{14.527073in}{1.437425in}}%
\pgfpathlineto{\pgfqpoint{14.559212in}{1.417057in}}%
\pgfpathlineto{\pgfqpoint{14.590314in}{1.305030in}}%
\pgfpathlineto{\pgfqpoint{14.622453in}{1.274477in}}%
\pgfpathlineto{\pgfqpoint{14.653555in}{1.222537in}}%
\pgfpathlineto{\pgfqpoint{14.685694in}{1.404836in}}%
\pgfpathlineto{\pgfqpoint{14.717833in}{1.462886in}}%
\pgfpathlineto{\pgfqpoint{14.746861in}{1.456164in}}%
\pgfpathlineto{\pgfqpoint{14.779000in}{1.376422in}}%
\pgfpathlineto{\pgfqpoint{14.810102in}{1.357682in}}%
\pgfpathlineto{\pgfqpoint{14.842241in}{1.426223in}}%
\pgfpathlineto{\pgfqpoint{14.873343in}{1.609540in}}%
\pgfpathlineto{\pgfqpoint{14.905482in}{1.956824in}}%
\pgfpathlineto{\pgfqpoint{14.937621in}{1.870258in}}%
\pgfpathlineto{\pgfqpoint{14.968723in}{1.829011in}}%
\pgfpathlineto{\pgfqpoint{15.000862in}{1.526029in}}%
\pgfpathlineto{\pgfqpoint{15.031964in}{1.562183in}}%
\pgfpathlineto{\pgfqpoint{15.064103in}{1.442518in}}%
\pgfpathlineto{\pgfqpoint{15.096241in}{1.381921in}}%
\pgfpathlineto{\pgfqpoint{15.126307in}{1.303400in}}%
\pgfpathlineto{\pgfqpoint{15.158446in}{1.375505in}}%
\pgfpathlineto{\pgfqpoint{15.189548in}{1.388948in}}%
\pgfpathlineto{\pgfqpoint{15.221687in}{1.437935in}}%
\pgfpathlineto{\pgfqpoint{15.252789in}{1.537842in}}%
\pgfpathlineto{\pgfqpoint{15.284927in}{1.698856in}}%
\pgfpathlineto{\pgfqpoint{15.317066in}{1.612290in}}%
\pgfpathlineto{\pgfqpoint{15.348168in}{1.677367in}}%
\pgfpathlineto{\pgfqpoint{15.380307in}{1.766683in}}%
\pgfpathlineto{\pgfqpoint{15.411409in}{1.825752in}}%
\pgfpathlineto{\pgfqpoint{15.443548in}{1.935539in}}%
\pgfpathlineto{\pgfqpoint{15.475687in}{2.025772in}}%
\pgfpathlineto{\pgfqpoint{15.504716in}{1.770248in}}%
\pgfpathlineto{\pgfqpoint{15.536854in}{1.933808in}}%
\pgfpathlineto{\pgfqpoint{15.567957in}{1.978517in}}%
\pgfpathlineto{\pgfqpoint{15.600095in}{1.988803in}}%
\pgfpathlineto{\pgfqpoint{15.631198in}{2.125068in}}%
\pgfpathlineto{\pgfqpoint{15.663336in}{2.124559in}}%
\pgfpathlineto{\pgfqpoint{15.695475in}{1.960490in}}%
\pgfpathlineto{\pgfqpoint{15.726577in}{2.114477in}}%
\pgfpathlineto{\pgfqpoint{15.758716in}{2.160204in}}%
\pgfpathlineto{\pgfqpoint{15.789818in}{2.247789in}}%
\pgfpathlineto{\pgfqpoint{15.821957in}{2.407784in}}%
\pgfpathlineto{\pgfqpoint{15.854096in}{2.606785in}}%
\pgfpathlineto{\pgfqpoint{15.883125in}{3.403706in}}%
\pgfpathlineto{\pgfqpoint{15.915263in}{3.369486in}}%
\pgfpathlineto{\pgfqpoint{15.946365in}{3.025359in}}%
\pgfpathlineto{\pgfqpoint{15.978504in}{2.445466in}}%
\pgfpathlineto{\pgfqpoint{16.009606in}{2.539263in}}%
\pgfpathlineto{\pgfqpoint{16.041745in}{2.314394in}}%
\pgfpathlineto{\pgfqpoint{16.073884in}{2.283434in}}%
\pgfpathlineto{\pgfqpoint{16.104986in}{2.354928in}}%
\pgfpathlineto{\pgfqpoint{16.137125in}{2.882270in}}%
\pgfpathlineto{\pgfqpoint{16.168227in}{3.194011in}}%
\pgfpathlineto{\pgfqpoint{16.168227in}{3.194011in}}%
\pgfusepath{stroke}%
\end{pgfscope}%
\begin{pgfscope}%
\pgfsetrectcap%
\pgfsetmiterjoin%
\pgfsetlinewidth{1.254687pt}%
\definecolor{currentstroke}{rgb}{0.150000,0.150000,0.150000}%
\pgfsetstrokecolor{currentstroke}%
\pgfsetdash{}{0pt}%
\pgfpathmoveto{\pgfqpoint{9.443020in}{0.782881in}}%
\pgfpathlineto{\pgfqpoint{9.443020in}{5.402881in}}%
\pgfusepath{stroke}%
\end{pgfscope}%
\begin{pgfscope}%
\pgfsetrectcap%
\pgfsetmiterjoin%
\pgfsetlinewidth{1.254687pt}%
\definecolor{currentstroke}{rgb}{0.150000,0.150000,0.150000}%
\pgfsetstrokecolor{currentstroke}%
\pgfsetdash{}{0pt}%
\pgfpathmoveto{\pgfqpoint{16.488475in}{0.782881in}}%
\pgfpathlineto{\pgfqpoint{16.488475in}{5.402881in}}%
\pgfusepath{stroke}%
\end{pgfscope}%
\begin{pgfscope}%
\pgfsetrectcap%
\pgfsetmiterjoin%
\pgfsetlinewidth{1.254687pt}%
\definecolor{currentstroke}{rgb}{0.150000,0.150000,0.150000}%
\pgfsetstrokecolor{currentstroke}%
\pgfsetdash{}{0pt}%
\pgfpathmoveto{\pgfqpoint{9.443020in}{0.782881in}}%
\pgfpathlineto{\pgfqpoint{16.488475in}{0.782881in}}%
\pgfusepath{stroke}%
\end{pgfscope}%
\begin{pgfscope}%
\pgfsetrectcap%
\pgfsetmiterjoin%
\pgfsetlinewidth{1.254687pt}%
\definecolor{currentstroke}{rgb}{0.150000,0.150000,0.150000}%
\pgfsetstrokecolor{currentstroke}%
\pgfsetdash{}{0pt}%
\pgfpathmoveto{\pgfqpoint{9.443020in}{5.402881in}}%
\pgfpathlineto{\pgfqpoint{16.488475in}{5.402881in}}%
\pgfusepath{stroke}%
\end{pgfscope}%
\end{pgfpicture}%
\makeatother%
\endgroup%
}
    \caption{Cours historique des contrats a terme sur blé et nickel}
\end{figure}
\subsection{Analyse technique}