\section{Analyse technique et macroéconomique du blé et du nickel}
\subsection{Présentation des deux matières premières}
\subsubsection{Le blé meunier}
L'étude porte en premier lieu sur le cours du contrat à terme de blé de meunerie côté en Euros sur le marché des futures à la bourse
de Paris (Euronext). Le sous-jacent du contrat doit être du blé d'origine Union Européenne, l'unité de cotation du contrat est de 50
tonnes. Ce contrat à terme permet aux producteurs et aux acheteurs de blé de se couvrir contre les fluctuations du prix du blé.\\[11pt]
Le blé meunier est une variété de blé tendre (Triticum aestivum) est la variété de blé la plus couramment cultivée dans les régions 
tempérées du monde. Le blé meunier est particulièrement apprécié pour sa concentration élevée en gluten, ce dernier donne à
la pâte de blé une texture élastique et une capacité à lever. La farine faite à partir du blé meunier est majoritairement 
utilisée afin de fabriquer du pain, des pâtisseries et d'autres denrées a base de blé.\\[11pt]
Au niveau de l'agriculture mondiale, la culture du blé est l'une des cultures les plus largement cultivées.
Les principaux producteurs de blé meunier sont la Chine, l'Inde, l'Union européenne et la Russie, les États-Unis, le Canada
et l'Ukraine.

Les contrats à terme sur le blé meunier sont largement négociés sur les marchés internationaux et constituent une composante
importante du commerce mondial des céréales. Les fluctuations des cours des futures de blé meunier peuvent avoir des implications
importantes pour les producteurs, les négociants, les meuniers et les consommateurs de produits alimentaires à base de blé.



Le blé meunier APT est une variété traditionnelle de blé tendre français. Appelée aussi Touselle blanche de Pertuis, 
identifiée en 1985. Ce blé a été considéré comme une espèce de première valeur alimentaire. Le blé tendre de son nom 
Triticum aestivum, ou blé panifiable. Le blé tendre est alors le plus cultivé, car il possède un taux de gluten et de 
protéines supérieur. Ce blé riche en protéine et pauvre en gluten, plaît aux professionnels qui l'utilisent pour la 
fabrication de brioches et de biscuits.  Et également présent dans la sphère agricole pour l’alimentation animale.
Le lieu d’origine de ce blé meunier est  Buoux, situé  entre Avignon et Aix-en-Provence. 

Célèbre comptine du patrimoine français : meunier tu dors tirée d’une chanson de Léon Raiter et de Fernand Pothier composée au XXème siècle. Une cloche sonnant à chaque tour permettait au meunier d’évaluer la vitesse de son moulin afin de produire correctement la farine de blé.

Situation délicate ces derniers temps avec l’invasion de l’Ukraine par la Russie stoppant ainsi les productions de blé des deux pays. Ajoutant à cela le manque considérable de pluie, on peut s’attendre à une situation générant une inflation agroalimentaire.


\subsection{Analyse macroéconomique}
\subsection{Analyse technique}