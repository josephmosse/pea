

\newcommand{\archtest}[3]{
\begin{equation*}
    \begin{split}
        H_{0} &: \text{Homoscédasticité} \\
        H_{1} &: \text{Hétéroscédasticité}
    \end{split}
\end{equation*}
Statistique de test :
    \begin{equation*}
        LM = n \times R^{2} \sim \rchi^{2}_{0,95} \, (p)
    \end{equation*}
La statistique du multiplicateur de Lagrange est comparée au quantile à 95\% de la distribution du khi-deux ayant pour degrés de liberté #1. Dans le cas suivant :}

\newcommand{\test}[3]{
    \begin{equation*}
        \begin{split}
            H_{0} &: \text{#1} \\
            H_{1} &: \text{#2}
        \end{split}
    \end{equation*}
    Statistique de test pour un niveau $\alpha = 5\%$:
    \begin{equation*}
        #3
    \end{equation*}
    Règle de décision : }

    \newcommand{\stutest}[5]{ % 1 : parametre ; 2 : n ; 3 : k ; 4 : n-k. ; 5 compar 
        \begin{equation*}
            \begin{split}
                H_{0} &: #1 = 0     \quad \text{Non significativité du paramètre} \\
                H_{1} &: #1 \neq 0  \quad \text{Significativité du paramètre}
            \end{split}
        \end{equation*}
        Statistique de test pour un niveau $\alpha = 5\%$:
        \begin{equation*}
            t_{c} = \frac{\hat{#1}}{\hat{\sigma}_{\hat{#1}}}\sim t_{0,975}(n-k)
        \end{equation*}
        Règle de décision : la statistique de student calculée en valeur absolue est comparée au quantile à 97,5\%, de la distribution bilatérale de Student avec comme 
        degrés de liberté $#2 - #3 = #4$. Si elle est inférieure alors la pente du modèle n'est pas significative, elle est en revanche significative si la statistique est 
        supérieure au seuil. Ici : 
        \begin{equation*}
            #5
        \end{equation*}}
    