\section*{Introduction}
\addcontentsline{toc}{section}{Introduction}
Dans un contexte marqué par une pandémie, une guerre militaire en Europe muée en guerre économique et commerciale mondialisée, ainsi que la menace croissante du 
changement climatique, la situation économique et les marchés financiers en sont retrouvés extrêmement perturbés. Une conception répandue
veut que lorsqu'il y a incertitude sur les marchés financiers, les investisseurs considèrent les matières premières comme des valeurs refuge par rapport à d'autres 
investissements moins tangibles. L'analyse économique et la prévision du prix des matières premières sont donc d'une grande importance.\\[5pt] 
C'est pourquoi nous avons choisi pour ce travail d'analyser et prévoir le cours de contrat à terme deux matières premières de natures différentes, une agricole et un métal :
\begin{itemize}
    \item \textbf{Le blé}, céréale essentielle à l'alimentation et dont la production est menacée par le changement climatique, mais aussi dont le commerce s'est retrouvé bouleversé en raison de la guerre russo-ukrainienne.
    \item \textbf{Le nickel}, métal omniprésent dans l'industrie et dont la demande est en augmentation notamment en raison de son utilisation pour la production 
    de batteries.
\end{itemize}
L'objet du travail porte donc sur la modélisation statistique et la prévision des séries temporelles que sont les cours des contrats futures du blé et du nickel. Concernant
la modélisation économétrique des séries temporelles, celle-ci est un piller de la science économique et en particulier de la finance quantitative. Les méthodes dites 
traditionnelles sont les premières méthodes de prévision à avoir été développées, utilisant des techniques de lissage ou d'extrapolation de composantes des séries 
temporelles. Cependant, dans les années 1970, l'arrivée de l'algorithme de Box et Jenkins a marqué un changement de paradigme en introduisant l'utilisation de processus 
aléatoires ARMA pour la modélisation des séries temporelles. Au fil du temps, l'utilisation des processus aléatoires a connu un développement et un approfondissement 
considérable grâce à des économistes tels que Robert F. Engle, qui a reçu le prix Nobel d'économie pour avoir introduit les modèles ARCH dans les années 1980. Aujourd'hui, 
les méthodes de prévision restent d'actualité grâce aux avancées technologiques dans le domaine de l'informatique, notamment avec l'avènement des technologies 
d'intelligence artificielle telles que l'apprentissage automatique. Dans le cadre de ce travail nous n'aborderons que les deux premières citée afin de répondre à la
problématique suivante : \\[5pt]
\textbf{Quelles sont les méthodes parmi celles traditionnelles et celle de Box et Jenkins plus performantes pour prévoir l'évolution du prix en 2023 de deux matières premières de natures différentes, à savoir le blé et le nickel}\\[5pt]
Nous tiendrons également compte de l'impact potentiel de la crise du Covid sur la modélisation de ces matières premières en utilisant deux échantillons de données, un 
ante-Covid19 : de 2016 à 2019 et un post-Covid19 : de 2016 à 2021.\\
Pour répondre à cette problématique, nous adopterons tout d'abord une approche qualitative en réalisant une analyse macroéconomique et technique des cours du blé et du 
nickel. Nous poursuivrons ensuite avec une approche quantitative en analysant les composantes des séries, que nous tenterons de prévoir dans une troisième partie en 
utilisant les méthodes traditionnelles. Finalement dans une quatrième partie, nous appliquerons l'algorithme de Box et Jenkins pour effectuer des prévisions sur ces mêmes 
séries.