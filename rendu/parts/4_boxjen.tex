\section{Prévision selon la méthodologie de Box \& Jenkins}
\subsection{Présentation de la méthode}
Lors de des prévisions faites grace aux méthodes traditionnelles, il a été possible de montrer que les résidus des prévisions n'étaient pas des bruits blancs. En effet, 
les méthodes traditionnelle s'avèrent le plus souvent inefficaces lors de prévisions de chroniques économiques, en particulier financières. Une partie des informations est 
donc perdue, mal modélisée par les lissages ou extrapolations. Une autre classe de méthodes de modélisation de séries temporelles peut donc être utilisée pour modéliser le 
cours du blé et du nickel : les processus aléatoires ARMA.\\[11pt]
La démocratisation de l'utilisation des processus aléatoires dans le domaine de la modélisation économique remonte aux années 1970. A cette époque, deux statisticiens, 
George Box et Gwilym Jenkins mettent au point une méthode itérative de prévision de séries temporelles basée sur les processus aléatoires ARMA. \\[11pt]
La première étape de cet algorithme est de transformer la série de base, c'est à dire procéder a une transformation logarithmique en cas de forte variations, corriger les 
variations saisonnières en cas de saisonnalité, et finalement corriger la tendance si il y en à une. Cette étape de transformation vise donc à rendre stationnaire une 
série temporelle pour que ses caractéristiques se rapprochent le plus à celles d'un processus ARMA. En effet, la méthodologie de Box et Jenkins est bâtie sur la 
modélisation de chroniques stationnaires, or les séries économiques ou financières sont rarement la réalisation de de processus aléatoires stationnaires. Il faut donc 
réaliser un test de racine unitaire afin de déterminer si la série est stationnaire ou non, et si oui alors quel est son type de non-stationnarité.\\[11pt]
Les types de processus non-stationnaires les plus fréquents sont :
\begin{itemize}
    \item \textbf{Les processus DS}  (\textit{Differency Stationary} ) représentent la non-stationnarité aléatoire, forme la plus commune des chroniques financières.
    \item \textbf{Les processus TS}  (\textit{Trend Stationary} ) représentent la non-stationnarité déterministe.
\end{itemize}
Si la série est un DS, il faut appliquer un filtre aux différences pour corriger la stationnarité. Au contraire si c'est un TS, la stationnarité est corrigée par la 
méthode des moindres carrés ordinaires.\\[11pt]
Lorsque la série est stationnaire, vient alors l'étape d'identification. Les caractéristiques des fonctions d'autocorrélation et autocorrélation partielle de la chronique
sont comparées à celles de processus ARMA théoriques, ceci permet d'identifier l'ordre du processus ARMA sous-jacent.\\[11pt]
La troisième étape est l'étape de d'estimation, les paramètres du processus ARMA identifié sont estimés par la méthode des MCO. Pour rappel, un ARMA($p,q$) est une 
combinaison linéaire de processus autorégressif à l'ordre $p$ AR($p$) et de processus moyenne mobile à l'ordre $q$ MA($q$) tel que :
\begin{equation*}
    x_t = \varepsilon_t +  \sum_{i=1}^p \phi_i x_{t-i} + \sum_{i=1}^q \theta_i \varepsilon_{t-i}
\end{equation*}
Une fois que les paramètres du modèle ont été estimés, s'en suit l'étape de tests de validation du modèle. En effet, si le modèle ne répond pas aux critères attendus 
d'un bon modèle ARMA, alors il faut revenir à l'étape d'identification afin d'identifier si possible, un meilleur modèle. Ces critères peuvent être classés de la sorte :
\begin{itemize}
    \item Minimisation des critères d'information construits pour les ARMA.
    \item Stationnarité de la partie AR et inversibilité de la partie MA.
    \item Significativité des paramètres estimés et du coefficient de détermination.
    \item Les résidus suivent un bruit blanc gaussien.
    \item Respect du principe de parcimonie.
\end{itemize}
Si le modèle estimé respecte la majorité des éléments de la liste, il alors est possible de passer à la cinquième et dernière étape : la prévision. Une fois la prévision
faite, il est nécessaire de recolorer la chronique, c'est à dire re-transformer la chronique à son état d'origine.\\[11pt]
Maintenant que les étapes de l'algorithme ont été explicités, ce dernier est utilisé afin de prévoir les valeurs des cours du blé et du nickel. La stratégie est 
sensiblement la même que celle de la partie~\ref{tradi}, c'est à dire : une prévision pour 2022 grace aux échantillons 2016-2019, puis une prévision pour 2022 grace aux 
échantillons 2016-2021, les échantillons utilisés étant ceux ayant été transformés logarithmiquement et corrigés des variations saisonnières si besoin. Les meilleurs 
modèles seront par la suite comparés aux méthodes traditionnelles afin de déterminer la méthode a utiliser pour prévoir 2023.
\subsection{Test de racine unitaire}
La type de non-stationnarité revêt une grande importance lorsqu'il s'agit de traiter des données statistiques d'une série temporelle. Pour cela qu'il est
primordial d'identifier si la chronique est un DS ou un TS, une mauvaise stationnarisation pouvant grandement fausser les résultats. Plusieurs tests permettent de
répondre à cette problématique, ce sont les tests de recherche de racine unitaire, parmi eux, le premier à avoir été mis au point est celui de Dickey-Fuller. Pour ce
travail, le test de Philip-Perron est utilisé, il s'agit d'une extension de celui de Dickey-Fuller qui permet de permet de prendre en compte les erreurs hétéroscédastiques 
et/ou autocorrélées.\\[11pt]
La stratégie de test de Philip-Perron est la même que celle du test de Dickey-Fuller augmenté, elle consiste en une estimation séquentielle de trois modèles : un AR(1) avec tendance et constante, un AR(1) avec constante, et un AR(1) simple. A chacune des étapes deux tests sont fait :
\begin{itemize}
    \item Un test de présence de racine unitaire $H_{0}$ (l'hypothèse alternative $H_{1}$ étant stationnarité de la chronique).
    \item Un test d'hypothèse jointe, permettant de valider la présence de racine unitaire et de différencier TS et DS. 
\end{itemize}
\subsubsection{Echantillon 2016-2019}
Dans un premier temps, la stratégie de test de racine unitaire est fait sur les échantillons 2016-2019 des cours du blé et du nickel.
\subsubsection*{$\bullet$ Test de racine unitaire sur le cours blé}
Le modèle 3 est estimé, c'est un AR(1) avec tendance et constante :
\begin{equation*}
    x_{t} = c + bt + \phi_{1} x_{t-1} + a_{t}
\end{equation*}
L'hypothèse de présence de racine unitaire est testée.
%
\begin{itemize}
\item[-]\textbf{ Hypothèse :} 
\begin{equation*}
    \begin{split}
        H_{0} &: \text{Présence de racine unitaire.}  \quad  (\phi_{1} = 1)\\
        H_{1} &: \text{Stationnarité du processus.}
    \end{split}
\end{equation*}
\item[-]\textbf{Statistique de test :} 
\begin{equation*}
    t_{c} = \frac{\tilde{\phi}_{1}- 1}{\hat{\sigma}_{\tilde{\phi}_{1}}}
\end{equation*}
\item[-]\textbf{Règle de décision :} Pour un niveau de test à 5\%, la statistique de student calculée est ensuite comparée à la statistique de student ajustée de la table 
de Dickey-Fuller (annexe). Si la statistique calculée est supérieure au seuil, alors l'hypothèse nulle de présence de racine unitaire est acceptée.
\item[-]\textbf{Application :} 
\begin{equation*}
    t_{c} = -2,67
\end{equation*}
D'autre part, la statistique de student ajustée lue dans la table de Dickey-Fuller est $t_{ajs} = - 3,5$. La statistique calculée est donc supérieure au seuil critique,
l'hypothèse $H_{0}$ est acceptée au risque de 5\%, il y a présence de racine unitaire.
\end{itemize}
%
Afin de distinguer TS de DS, il faut a présent tester l'hypothèse jointe $H_{0}^{3}$
%
\begin{itemize}
\item[-]\textbf{Hypothèse :} 
\begin{equation*}
    \begin{split}
        H_{0}^{3} &: (c; b; \phi_{1}) = (c;0;1)\\
        H_{1}^{3} &: \text{Au moins un des paramètres est différent.}
    \end{split}
\end{equation*}
\item[-]\textbf{Statistique de test :} 
\begin{equation*}
    F_{3} = \frac{(SCR_{c}^{3} - SCR_{3})/2}{SCR_{3}/(n-3)}
\end{equation*}
Où $SCR_{3}$ est la somme des carrés des résidus du modèle 3 et $SCR_{c}^{3}$ la somme des carrés des résidus du modèle 3 contraint sous l'hypothèse $H_{0}^{3}$ tel que :
$SCR_{c}^{3} = \sum_{t} \left(x_{t} - x_{t-1} - \hat{c}\right)^{2}$.
\item[-]\textbf{Règle de décision :} La statistique de Fisher calculée est par la suite comparée à la statistique de Fisher tabulée de Dickey-Fuller. Si $F_{3}$ est 
inférieur au seuil critique lu dans la table pour un niveau de test à 5\%, alors $H_{0}^{3}$ est acceptée.
\item[-]\textbf{Application :} 
\begin{equation*}
    F_{3} = \frac{(0,100994 -  0,087023)/2}{0,087023/(47-3)} = 3,53
\end{equation*}
La statistique lue dans la table de Dickey-Fuller est $\Phi_{3} = 6,73$; or $F_{3}$ est inférieure à $\Phi_{3}$. L'hypothèse $H_{0}^{3}$ Est donc acceptée au risque 
de 5\%.
\end{itemize}
%
En suivant le diagramme de stratégie de test de racine unitaire, il est désormais nécessaire de tester l'hypothèse jointe $H_{0}^{2}$.
%
\begin{itemize}
\item[-]\textbf{Hypothèse :} 
\begin{equation*}
    \begin{split}
        H_{0}^{2} &: (c; b; \phi_{1}) = (0;0;1)\\
        H_{1}^{2} &: \text{Au moins un des paramètres est différent.}
    \end{split}
\end{equation*}
\item[-]\textbf{Statistique de test :}
\begin{equation*}
    F_{2} = \frac{(SCR_{c} - SCR_{3})/3}{SCR_{3}/(n-3)}
\end{equation*}
Où $SCR_{c}$ est la somme des carrés des résidus du modèle 3 contraint sous l'hypothèse $H_{0}^{2}$, il s'agira donc ici de la somme des carrés de la différence première
du cours en log du blé.
\item[-]\textbf{Règle de décision :}  Si la statistique $F_{2}$ est inférieure à la statistique lue dans la table de Dickey-Fuller pour un niveau de test de 5\%, alors $H_{0}^{2}$ est acceptée.
\item[-]\textbf{Application :} 
\begin{equation*}
    F_{2} = \frac{(0,101423 - 0,087023)/3}{0,087023/(47-3)} = 2,43
\end{equation*}
De plus, la statistique de Fisher lue dans la table de Dickey-Fuller $\Phi_{2} = 5,13$ est supérieure à la statistique calculée. L'hypothèse $H_{0}^{2}$ est donc acceptée
au risque de 5\%. 
\end{itemize}
%
Pour le moment, d'après le diagramme, le cours du blé ne peut pas être un TS. Il est maintenant nécessaire d'estimer le modèle 2, qui est identique au modèle 3 à 
l'exception de l'absence de tendance.
\begin{equation*}
    x_{t} = c + \phi_{1} x_{t-1} + a_{t}
\end{equation*}
De manière séquentielle, comme pour le modèle 3, l'hypothèse de présence de racine unitaire est testée.
%
\begin{itemize}
\item[-]\textbf{ Hypothèse :} 
\begin{align*}
        H_{0} &: \phi_{1} = 1  & &\text{Présence de racine unitaire.}\\
        H_{1} &:|\phi_{1}| < 1   &  &\text{Stationnarité du processus.}
\end{align*}
\item[-]\textbf{Statistique de test :} 
\begin{equation*}
    t_{c} = \frac{\tilde{\phi}_{1}- 1}{\hat{\sigma}_{\tilde{\phi}_{1}}}
\end{equation*}
\item[-]\textbf{Règle de décision :} La statistique de student calculée est comparée à la statistique de student tabulée de Dickey-Fuller pour un niveau de test à 
5\%. Si la statistique calculée est supérieure au seuil critique, alors l'hypothèse nulle de présence de racine unitaire est acceptée.
\item[-]\textbf{Application :} 
\begin{equation*}
    t_{c} = -1,73 > t_{ajs} = -2,93
\end{equation*}
L'hypothèse $H_{0}$ est donc acceptée au risque de 5\%, il y a présence de racine unitaire.
\end{itemize}
%
Un test d'hypothèse jointe $H_{0}^{1}$ est ensuite fait afin de valider la présence de racine unitaire et tester la nullité de la constante $c$.
\begin{itemize}
\item[-]\textbf{Hypothèse :} 
\begin{equation*}
    \begin{split}
        H_{0}^{1} &: (c; \phi_{1}) = (0;1)\\
        H_{1}^{1} &: \text{Au moins un des paramètres est différent.}
    \end{split}
\end{equation*}
\item[-]\textbf{Statistique de test :}
\begin{equation*}
    F_{1} = \frac{(SCR_{c} - SCR_{2})/2}{SCR_{2}/(n-2)}
\end{equation*}
Où $SCR_{2}$ est la somme des carrés des résidus du modèle 2 non contraint.
\item[-]\textbf{Règle de décision :}  Si la statistique $F_{2}$ est inférieure à la statistique lue dans la table de Dickey-Fuller pour un niveau de test de 5\%, alors 
$H_{0}^{2}$ est acceptée.
\item[-]\textbf{Application :} 
\begin{equation*}
    F_{2} = \frac{(0,101423 - 0,094683)/2}{0,094683/(47-2)} = 1,60
\end{equation*}
Cette statistique est inférieure à la statistique de Fisher lue dans la table de Dickey-Fuller $\Phi_{1} = 4,86$. L'hypothèse $H_{0}^{2}$ est donc acceptée au risque 
de 5\%. 
\end{itemize}
%
D'après le diagramme, il faut maintenant tester la nullité de la moyenne du cours du blé. En effet, si il s'avère que la moyenne n'est pas nulle alors le processus 
sous-jacent est un DS.
\begin{itemize}
    \item[-]\textbf{Hypothèse :}
\begin{equation*}
\begin{split}
    H_{0} &: \mu = 0\\
    H_{1} &: \mu \neq 0
\end{split}
\end{equation*}
\item[-]\textbf{Règle de décision :} La statistique calculé est un student et elle est comparée à la distribution bilatérale de la loi de Student. Si la probabilité 
critique d'accepter $H_{0}$ est supérieure à 5\%, alors $H_{0}$ est acceptée au risque de 5\%.
\item[-]\textbf{Application :} Ici la probabilité critique de rejeter $H_{0}$ est de 0 (annexe), l'hypothèse de nullité de la moyenne est donc rejetée au risque de 5\%.
\end{itemize}
La moyenne du cours du blé n'étant pas nulle, il est possible de conclure que le processus générateur de la chronique est un DS. Pour le rendre stationnaire il est donc
nécessaire d'appliquer un filtre aux différences premières.
\begin{equation*}
    \Delta x_{t} = x_{t} - x_{t-1}
\end{equation*}
\subsubsection*{Le nickel}
\subsubsection{Echantillon 2016-2021}
\subsubsection*{Le blé}
\subsubsection*{Le nickel}
\subsection{Identification des processus}
\subsection{Prévision pour 2023}