\section*{Conclusion}
\addcontentsline{toc}{section}{Conclusion}
Le but de ce travail était d'évaluer les performances de différentes méthodes de prévision, notamment comparer les méthodes traditionnelles et la méthodologie de Box et 
Jenkins, dans le but d'identifier la méthode la plus efficace pour effectuer des prévisions sur le cours de deux matières premières de nature différentes. Toutefois, étant 
donné l'ampleur inédite du choc économique survenu en 2020, il était nécessaire de développer une stratégie de prévision pour répondre à cette problématique. Cette 
stratégie nous a permis de rester flexibles dans notre modélisation de ce choc. Pour cela, nous avons utilisé deux échantillons : un échantillon ante-Covid19 (de 2016 à 
2019) et un échantillon post-Covid19 (de 2016 à 2021).\\[5pt]
Lors d'une analyse de prévision basée sur des méthodes traditionnelles, nous avons observé que même en présence de composantes extra-saisonnières (tendance) déterministes, 
les lissages exponentiels étaient les meilleures méthodes pour prévoir les deux cours. En particulier, un lissage exponentiel double a été utilisé pour prévoir le cours du 
blé et un lissage exponentiel de Holt-Winters a été utilisé pour prévoir celui du nickel. De plus, nous avons constaté que la prévision du cours du nickel était plus 
précise lorsqu'on utilisait l'échantillon ante-Covid19, et inversement pour le blé. Donc :
\begin{itemize}
    \item Le lissage exponentiel double s'est avéré être la méthode la plus efficace pour prévoir le cours du blé. En outre, l'inclusion de la période de la pandémie 
    Covid19 a permis une meilleure modélisation de la série.
    \item La méthode appropriée pour prévoir le cours du nickel est le lissage de Holt-Winters. Cependant, contrairement au cours du blé, l'inclusion de la période de la 
    pandémie Covid19 n'a pas amélioré les résultats de la prévision.
\end{itemize}
Par la suite, nous avons comparé les résultats obtenus par ces méthodes aux résultats obtenus en prévoyant à l'aide de la méthodologie de Box-Jenkins. Plus précisément, 
dans la section~\ref{sec:bj}, nous avons tenté de modéliser les prix en utilisant des processus aléatoires de type ARMA. Cependant, il n'a été possible de modéliser que le 
prix du blé, car celui du nickel s'est avéré statistiquement indépendant et \textit{i.i.d.}. En ce qui concerne le prix du blé, après avoir été rendu stationnaire, le 
processus sous-jacent ARMA identifié et estimé fut celui d'un AR(1). \\[5pt]
Après avoir effectué des prévisions à l'aide de l'algorithme de Box et Jenkins, une dernière étape de comparaison a été réalisée, révélant que les méthodes traditionnelles 
demeurent plus efficaces pour prédire les fluctuations de prix des deux matières premières. Il est possible que ce résultat soit cohérent étant donné que les contrats à 
terme sur les matières premières ont un comportement distinct de celui des instruments financiers tels que les actions, qui peuvent être plus volatils et mieux modélisés 
par des méthodes telles que les modèles ARMA ou GARCH.\\[5pt]
En réalité, la pertinence de telles méthodes peut être remise en question, car elles peuvent être peu applicables en finance, étant donné leur difficulté à prédire les mouvements de marché anormaux tels que les chocs et les crises. En effet, à l'heure de la digitalisation croissante de l'industrie financière, cette dernière est de plus en plus axée sur les données \textit{data driven}. L'utilisation de données massives et d'algorithmes pour l'analyse de marché, gérer des portefeuilles et prendre des décision est de plus en plus courante et ont tendance à surpasser les méthodes abordées dans ce travail.