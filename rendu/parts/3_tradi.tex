\section{Prévision par le méthodes traditionnelles}
Une fois que toutes les séries ont été amplement analysées, transformées, et si besoin, corrigées des variations saisonnières, il alors est possible d'appliquer des 
méthodes de prévision traditionnelles. Le choix d'une méthode de prévision depend du caractère déterministe (ou aléatoire) de l'extra-saisonnalité et de la saisonnalité de
la série a prévoir. \\[11pt]
Dans le cadre de l'étude du cours du blé et du nickel, il a précédemment été montré que les deux échantillons du blé, ainsi que l'échantillon 2016-2019 
du nickel ne présentaient pas de saisonnalité, pour ces séries là donc, seule la nature de la composante extra-saisonnière est à prendre en compte. Concernant cette 
dernière, elle est déterministe pour les trois séries citées étant donné qu'elles possèdent toutes une tendance. Dans ce cas là, en théorie la méthode de prévision a 
utiliser serait l'extrapolation par une droite de tendance.\\[11pt]
Concernant l'échantillon 2019-2021 du nickel, malgré une composante saisonnière aléatoire, ce dernière possède comme les autres échantillons, une tendance. La méthode de 
prévision adéquate serait donc aussi l'extrapolation d'une droite de tendance.\\[11pt]
Cependant, afin de ne mettre aucun élément d'analyse de côté, en plus de l'extrapolation, des méthodes de prévision par lissage exponentiel de composantes seront 
utilisées. Parmi ces méthodes, le lissage exponentiel double (LED) et le lissage exponentiel de Holt-Winter sont choisis. \\[11pt]
Avant de prédire le cours pour 2023, pour chacune des matières premières, deux prévisions seront faites. Une première, prévision pour 2020 (grace à l'échantillon 2016-2019)
et une seconde pour 2022 (grace à l'échantillon 2016-2021). L'objectif étant de sélectionner la méthode de prévision minimisant le critère de RMSE (Root-Mean-Squarred 
Errors).
\subsection{Échantillon 2016-2019}
\subsubsection{Prévision pour 2020}
\subsubsection*{Extrapolation d'une droite de tendance} 
La prévision par extrapolation d'une droite est la méthode la plus adéquate pour estimer les tendances déterministes, elle consiste à modéliser la série par une droite.
Cette droite s'écrit telle que :
\begin{equation*}
    x_{t} = \beta t + \alpha + \varepsilon_{t}
\end{equation*}
Où $x_{t}$ est une série temporelle non saisonnière et $t$ le temps. Par la suite, les paramètres $\hat{\alpha}$ et $\hat{\beta}$ sont estimés grace à la méthode des MCO 
(Moindres Carrés Ordinaires). Avant de prévoir, il est nécessaire de procéder aux tests sur les paramètres de la regression afin de valider le modèle.\\[11pt]
Le test est le même pour les deux séries :
\begin{equation*}
    \begin{split}
        H_{0} &: \beta = 0     \quad \text{Non significativité du paramètre} \\
        H_{1} &: \beta \neq 0  \quad \text{Significativité du paramètre}
    \end{split}
\end{equation*}
Statistique de test pour un niveau $\alpha = 5\%$:
\begin{equation*}
    t_{c} = \frac{\hat{\beta}}{\hat{\sigma}_{\hat{\beta}}}\sim t_{0,975}(n-k)
\end{equation*}
Règle de décision : la statistique de student calculée en valeur absolue est comparée au quantile à 97,5\%, de la distribution bilatérale de Student avec comme 
degrés de liberté $48 - 2 = 46$. Si elle est inférieure alors la pente du modèle n'est pas significative, elle est en revanche significative si la statistique est 
supérieure au seuil.\\[11pt]
Ici, les deux statistiques calculées (~\ref{tab:mco_ble19} ~\ref{tab:mco_nickel19})$H_{0}$ sont supérieures au seuil (1,96). $H_{0}$ est donc acceptée au risque de 5\%, 
les pentes des deux modèles sont significatives. Il en est de même pour les constantes du modèle, la probabilité critique d'accepter l'hypothèse nulle étant 0. Les 
paramètres du modèles sont donc significatifs. Les valeurs pour 2020 du blé et du nickel peuvent être calculées en extrapolant les droites.\\[11pt]
\textbf{Le blé} 
\begin{table}[H]
    \centering
    \caption{Prévision par extrapolation linéaire du cours du blé pour l'année 2020}
    \sffamily
    \begin{tabular}{rrr}
\toprule
\multicolumn{1}{c}{Mois} & \multicolumn{1}{l}{Valeurs prévues} & \multicolumn{1}{c}{Valeurs empiriques} \\
\midrule
01-2020 & 192,49 & 191,00\\
02-2020 & 193,34 & 187,50 \\
03-2020 & 194,20 & 196,25 \\
04-2020 & 195,06 & 195,75 \\
05-2020 & 195,92 & 188,25 \\
06-2020 & 196,79 & 180,50 \\
07-2020 & 197,66 & 182,75 \\
08-2020 & 198,54 & 187,75 \\
09-2020 & 199,41 & 197,75 \\
10-2020 & 200,30 & 205,25 \\
11-2020 & 201,18 & 210,25 \\
12-2020 & 202,07 & 213,25 \\
\bottomrule
\end{tabular}%

\end{table}
Les différents tests sur les résidus sont fait.
\begin{itemize}
    \item \underline{Test de recherche d'autocorrelation} : Test de Ljung-Box
    \test{Abscence d'autocorrélation à l'odre h}{Autocorélation à l'odre h}{Q = n(n+2)\sum_{k=1}^h\frac{\hat{\rho}^2_k}{n-k} \sim \rchi_{0,95}^{2}(h)}
    \item \underline{Test d'homoscédasticité} : Test ARCH
\end{itemize}

\subsubsection*{Lissage exponentiel double (LED)}
Les techniques de lissage exponentiel ont été introduites par Holt et Brown. Un lissage exponentiel double consiste à effectuer deux lissage sur une série temporelle 
non saisonnière. Dans un premier temps donc il est nécessaire d'effectuer un lissage exponentiel simple (LES) sur la série. Le LES considère qu'une chronique peut être
décrite comme une combinaison linéaire des valeurs passées pondérées par un poids qui décroît plus les observations sont anciennes. \\[11pt]
Afin de matérialiser ce poids, une  constante de lissage $\lambda$ comprise entre 0 et 1 est utilisée. En fonction de sa valeur, $\lambda$ donnera un poids plus ou moins 
important au passé. Si $ \lambda $ est proche de 0, alors la mémoire du phénomène est dite forte, la prévision dépend beaucoup des observations passées. En revanche, si la 
constante est proche de 1, alors la mémoire du phénomène est faible, le lissage est plus réactif aux observations récentes.\\[11pt]
Dans le cas du blé et du nickel, le programme d'optimisation calcule $ \lambda = 0,43 $ pour le blé et $ \lambda = 0,47 $ pour le nickel. Les deux constantes sont
proches de 0,45, cela veut dire que pour les deux séries, la prévision par lissage apportera très légèrement plus d'importance au passé que au présent.
\begin{table}[H]
    \centering
    \centering
    \caption{Prévision par LED de l'échantillon 2016-2019 du Blé}
    \sffamily
    \begin{tabular}{ccc}
\toprule
Mois & Valeurs prévues & Valeurs réelles \\
\midrule
01-2020 & 189,45 & 191,00 \\
02-2020 & 192,02 & 187,50 \\
03-2020 & 194,64 & 196,25 \\
04-2020 & 197,28 & 195,75 \\
05-2020 & 199,97 & 188,25 \\
06-2020 & 202,69 & 180,50 \\
07-2020 & 205,45 & 182,75 \\
08-2020 & 208,24 & 187,75 \\
09-2020 & 211,07 & 197,75 \\
10-2020 & 213,95 & 205,25 \\
11-2020 & 216,86 & 210,25 \\
12-2020 & 219,81 & 213,25 \\
\bottomrule
\end{tabular}%

\end{table}

\begin{table}[H]
    \centering
    \caption{Prévision par LED de l'échantillon 2016-2019 du nickel}
    \sffamily
    \begin{tabular}{rrr}
\toprule
\multicolumn{1}{c}{Mois} & \multicolumn{1}{c}{Valeurs prévues} & \multicolumn{1}{c}{Valeurs réelles} \\
\midrule
01-2020 & 13817,13 & 12850 \\
02-2020 & 13484,34 & 12255 \\
03-2020 & 13159,58 & 11484 \\
04-2020 & 12842,63 & 12192 \\
05-2020 & 12533,32 & 12324 \\
06-2020 & 12231,45 & 12805 \\
07-2020 & 11936,86 & 13786 \\
08-2020 & 11649,36 & 15367 \\
09-2020 & 11368,79 & 14517 \\
10-2020 & 11094,98 & 15156 \\
11-2020 & 10827,75 & 16033 \\
12-2020 & 10566,97 & 16613 \\
\bottomrule
\end{tabular}

\end{table}


\subsubsection*{Lissage exponentiel de Holt-Winter}
La prévision par lissage exponentiel de Holt-Winter est une méthode de prévision de séries chronologiques saisonnières. La méthode consiste à effectuer un LED de Holt 
sur la partie non saisonnière, c'est à dire la moyenne et la tendance, et un lissage exponentiel saisonnier sur la composante saisonnalité. Ici, les deux échantillons du
blé, ainsi que l'échantillon 2016-2019 du nickel étant non saisonniers, la méthode revient à un LED sur deux paramètres pour ces échantillons.\\[11pt] 
Comme pour le LED, les différentes constantes de lissage ($\alpha, \beta$) sont calculées via une minimisation de la somme des carrés des résidus et sont trouvés dans le 
tableau ci dessous.
\begin{table}[H]
    \centering
    \caption{Constantes de lissage de la méthode HW}
    \sffamily
    \begin{tabular}{lcc}
        \toprule
        & Blé & Nickel\\
        \midrule
        $\alpha$ & 0,78 & 0,89 \\
        $\beta$ & 0,00 & 0,00\\
        \bottomrule
    \end{tabular}
\end{table}
Ainsi la prévision pour 2020 peut être faite :
\begin{table}[H]
    \centering
    \caption{Prévision du cours du blé en 2020 par lissage de Holt-Winter}
    \sffamily
    
\begin{tabular}{ccc}
\toprule
Mois & Valeurs prévues & Valeurs réelles \\
\midrule
01-2020 & 187,26 & 191,00 \\
02-2020 & 186,99 & 187,50 \\
03-2020 & 186,72 & 196,25 \\
04-2020 & 186,46 & 195,75 \\
05-2020 & 186,19 & 188,25 \\
06-2020 & 185,93 & 180,50 \\
07-2020 & 185,66 & 182,75 \\
08-2020 & 185,40 & 187,75 \\
09-2020 & 185,14 & 197,75 \\
10-2020 & 184,87 & 205,25 \\
11-2020 & 184,61 & 210,25 \\
12-2020 & 184,35 & 213,25 \\
\bottomrule
\end{tabular}

\end{table}

\begin{table}[H]
    \centering
    \caption{Prévision du cours du nickel en 2020 par lissage de Holt-Winter}
    \sffamily
    \begin{tabular}{ccc}
\toprule
Mois & Valeurs prévues & Valeurs réelles \\
\midrule
01-2020 & 14272,31 & 12850 \\
02-2020 & 13306,19 & 12255 \\
03-2020 & 11664,18 & 11484 \\
04-2020 & 11825,33 & 12192 \\
05-2020 & 12081,62 & 12324 \\
06-2020 & 12792,85 & 12805 \\
07-2020 & 13731,23 & 13786 \\
08-2020 & 14456,35 & 15367 \\
09-2020 & 14663,90 & 14517 \\
10-2020 & 14916,73 & 15156 \\
11-2020 & 14698,81 & 16033 \\
12-2020 & 16106,49 & 16613 \\
\bottomrule
\end{tabular}

\end{table}
\subsubsection{Choix de la meilleure méthode}
L'objectif du travail étant de prévoir la valeur de 2023, il est désormais nécessaire de sélectionner la meilleure méthode de prévision. Pour ce faire, le critère
de comparaison utilisé est le MSE (\textit{Mean Squared Errors}), ce dernier est calculé comme la moyenne des erreurs quadratiques.
\begin{equation*}
    \text{MSE} = \frac{1}{n} \sum_{i=1}^{n} (X_{i} - \hat{X}_{i})
\end{equation*}

\subsubsection{Prévision pour 2022}

\subsection{Échantillon 2016-2021}
\subsubsection{Prévision pour 2022}
\subsubsection*{Extrapolation d'une droite de tendance}
\subsubsection*{Lissage exponentiel double (LED)}
\subsubsection*{Lissage exponentiel triple (Holt Winter)}
\subsection{Prévision pour 2023}