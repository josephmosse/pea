\section{Prévision par le méthodes traditionnelles}
Une fois que toutes les séries ont été amplement analysées, transformées, et si besoin, corrigées des variations saisonnières, il alors est possible d'appliquer des 
méthodes de prévision traditionnelles. Le choix d'une méthode de prévision depend du caractère déterministe (ou aléatoire) de l'extra-saisonnalité et de la saisonnalité de
la série a prévoir. \\[11pt]
Dans le cadre de l'étude du cours du blé et du nickel, il a précédemment été montré que les deux échantillons du blé, ainsi que l'échantillon 2016-2019 
du nickel ne présentaient pas de saisonnalité, pour ces séries là donc, seule la nature de la composante extra-saisonnière est à prendre en compte. Concernant cette 
dernière, elle est déterministe pour les trois séries citées étant donné qu'elles possèdent toutes une tendance. Dans ce cas là, en théorie la méthode de prévision a 
utiliser serait l'extrapolation par une droite de tendance.\\[11pt]
Concernant l'échantillon 2019-2021 du nickel, malgré une composante saisonnière aléatoire, ce dernière possède comme les autres échantillons, une tendance. La méthode de 
prévision adéquate serait donc aussi l'extrapolation d'une droite de tendance.\\[11pt]
Cependant, afin de ne mettre aucun élément d'analyse de côté, en plus de l'extrapolation, des méthodes de prévision par lissage exponentiel de composantes seront 
utilisées. Parmi ces méthodes? le lissage exponentiel double (LED) et le lissage exponentiel de Holt-Winter sont choisis. \\[11pt]
Avant de prédire le cours pour 2023, pour chacune des matières premières, deux prévisions seront faites. Une première, prévision pour 2020 (grace à l'échantillon 2016-2019)
et une seconde pour 2022 (grace à l'échantillon 2016-2021). L'objectif étant de sélectionner la méthode de prévision minimisant le critère de RMSE (Root, Mean, Squarred 
Errors).
\subsection{Prévision pour 2020}
\subsubsection{Extrapolation d'une droite de tendance}
\subsubsection{Lissage exponentiel double (LED)}
\subsubsection{Lissage exponentiel triple (Holt Winter)}
\subsection{Prévision pour 2022}
\subsubsection{Extrapolation d'une droite de tendance}
\subsubsection{Lissage exponentiel double (LED)}
\subsubsection{Lissage exponentiel triple (Holt Winter)}
\subsection{Classification des méthodes}
\subsubsection{Blé}
\subsubsection{Nickel}
\subsection{Prévision pour 2023}
\subsubsection{Blé}
\subsubsection{Nickel}