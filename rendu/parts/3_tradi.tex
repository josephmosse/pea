\section{Prévision par le méthodes traditionnelles}\label{sec:tradi}
Une fois que toutes les séries ont été amplement analysées, transformées, et si besoin, corrigées des variations saisonnières, il alors est possible d'appliquer des 
méthodes de prévision traditionnelles. Le choix d'une méthode de prévision depend du caractère déterministe (ou aléatoire) de l'extra-saisonnalité et de la saisonnalité de
la série a prévoir. \\[11pt]
Dans le cadre de l'analyse du cours du blé et du nickel, il a précédemment été montré que les deux échantillons du blé, ainsi que l'échantillon 2016-2019 
du nickel ne présentaient pas de saisonnalité, pour ces séries-là donc, seule la nature de la composante extra-saisonnière est à prendre en compte. Concernant cette 
dernière, elle est déterministe pour les trois séries citées étant donné qu'elles possèdent toutes une tendance. Dans ce cas-là, en théorie la méthode de prévision à 
utiliser serait l'extrapolation par une droite de tendance.\\[11pt]
Concernant l'échantillon 2019-2021 du nickel, malgré une composante saisonnière aléatoire, ce dernière possède comme les autres échantillons, une tendance. La méthode de 
prévision adéquate reste l'extrapolation d'une droite de tendance.\\[11pt]
Cependant, afin de ne mettre aucun élément d'analyse de côté, en plus de l'extrapolation, des méthodes de prévision par lissage exponentiel de composantes seront 
utilisées. Parmi ces méthodes, le lissage exponentiel double (LED) et le lissage exponentiel de Holt-Winters sont choisis. \\[11pt]
Afin de prévoir le cours en 2023, il est nécessaire de sélectionner la méthode ayant le meilleur score de prévision sur des données passées. Pour les deux matières 
premières le choix de la meilleure méthode se fera en deux grandes étapes :
\begin{enumerate}
    \item Prévision des cours en 2020 grace aux échantillons 2016-2019, la prévision minimisant le critère MSE sur 2020 sera retenue et prolongée jusqu'à fin 2022.
    \item Prévision des cours en 2022 grace aux échantillons 2016-2021, comparaison des MSE avec les MSE de la méthode retenue pour 2020, la prévision minimisant le 
            critère sera retenue et utilisée pour prévoir les cours de 2023 
\end{enumerate}

\subsection{Échantillon 2016-2019}\label{prev1}
\subsubsection{Prévision pour 2020}
\subsubsection*{Extrapolation d'une droite de tendance} 
La prévision par extrapolation d'une droite est la méthode la plus adéquate pour estimer les tendances déterministes, elle consiste à modéliser la série par une droite.
Cette droite s'écrit telle que :
\begin{equation*}
    x_{t} = \beta t + \alpha + \varepsilon_{t}
\end{equation*}
Où $x_{t}$ est une série temporelle non saisonnière et $t$ le temps. Par la suite, les paramètres $\hat{\alpha}$ et $\hat{\beta}$ sont estimés grace à la méthode des MCO 
(Moindres Carrés Ordinaires). Avant de prévoir, il est nécessaire de procéder aux tests sur les paramètres et résidus de la regression afin de valider le modèle.\\[11pt]
Le test est le même pour les deux séries :
\begin{equation*}
    \begin{split}
        H_{0} &: \beta = 0     \quad \text{Non significativité du paramètre} \\
        H_{1} &: \beta \neq 0  \quad \text{Significativité du paramètre}
    \end{split}
\end{equation*}
\textbf{Statistique de test :} 
\begin{equation*}
    t_{c} = \frac{\hat{\beta}}{\hat{\sigma}_{\hat{\beta}}}\sim t_{0,975}(46)
\end{equation*}
\textbf{Règle de décision :}  la statistique de Student calculée en valeur absolue est comparée au quantile à 97,5\%, de la distribution bilatérale de Student avec comme 
degrés de liberté $46$. Si elle est inférieure alors la pente du modèle n'est pas significative, elle est en revanche significative si la statistique est 
supérieure au seuil.\\[11pt]
Ici, les deux statistiques calculées\footnote{Voir Voir annexe ~\ref{appendix:extra_19} p.~\pageref{appendix:extra_19}} sont 
supérieures au seuil (1,96). $H_{0}$ est donc acceptée au risque de 5\%, 
les pentes des deux modèles sont significatives. Il en est de même pour les constantes du modèle, la probabilité critique d'accepter l'hypothèse nulle étant 0. Les 
paramètres du modèles sont donc significatifs. Les valeurs pour 2020 du blé et du nickel peuvent être calculées en extrapolant les droites.
\begin{table}[H]
        \centering
        \caption{Prévision du cours du blé et du nickel en 2020 par extrapolation linéaire}
        \sffamily
        % Table generated by Excel2LaTeX from sheet 'EXTRA'
\begin{tabular}{ccccc}
\toprule
      & \multicolumn{2}{c}{Blé (\euro)} & \multicolumn{2}{c}{Nickel (\$)} \\
Mois  & Valeurs prévues & Valeurs réelles & Valeurs prévues & Valeurs réelles \\
\cmidrule(r){1-1}\cmidrule(lr){2-3}\cmidrule(lr){4-5}
01-2020 & 192,49 & 191,00 & 15312,42 & 12850,00 \\
02-2020 & 193,34 & 187,50 & 15483,91 & 12255,00 \\
03-2020 & 194,20 & 196,25 & 15657,33 & 11484,00 \\
04-2020 & 195,06 & 195,75 & 15832,68 & 12192,00 \\
05-2020 & 195,92 & 188,25 & 16010,00 & 12324,00 \\
06-2020 & 196,79 & 180,50 & 16189,31 & 12805,00 \\
07-2020 & 197,66 & 182,75 & 16370,62 & 13786,00 \\
08-2020 & 198,54 & 187,75 & 16553,96 & 15367,00 \\
09-2020 & 199,41 & 197,75 & 16739,36 & 14517,00 \\
10-2020 & 200,30 & 205,25 & 16926,83 & 15156,00 \\
11-2020 & 201,18 & 210,25 & 17116,41 & 16033,00 \\
12-2020 & 202,07 & 213,25 & 17308,10 & 16613,00 \\
\bottomrule
\end{tabular}%

\end{table}


\subsubsection*{Lissage exponentiel double (LED)}
Les techniques de lissage exponentiel ont été introduites par Holt et Brown. Un lissage exponentiel double consiste à effectuer deux lissage sur une série temporelle 
non saisonnière. Dans un premier temps donc il est nécessaire d'effectuer un lissage exponentiel simple (LES) sur la série\footnote{Voir tableau~\ref{tab:led} p.~\pageref
{tab:led}}. Le LES considère qu'une chronique peut être décrite comme une combinaison linéaire des valeurs passées pondérées par un poids qui décroît plus les observations 
sont anciennes. \\[11pt]
Afin de matérialiser ce poids, une  constante de lissage $\lambda$ comprise entre 0 et 1 est utilisée. En fonction de sa valeur, $\lambda$ donnera un poids plus ou moins 
important au passé. Si $ \lambda $ est proche de 0, alors la mémoire du phénomène est dite forte, la prévision dépend beaucoup des observations passées. En revanche, si la 
constante est proche de 1, alors la mémoire du phénomène est faible, le lissage est plus réactif aux observations récentes.\\[11pt]
Dans le cas du blé et du nickel, le programme d'optimisation calcule $ \lambda = 0,43 $ pour le blé et $ \lambda = 0,47 $ pour le nickel\footnote{Voir annexe ~\ref{appendix:led_19} p.~\pageref{appendix:led_19} }. Les deux constantes sont
proches de 0,45, cela veut dire que pour les deux séries, la prévision par lissage apportera très légèrement plus d'importance au passé que au présent.
\begin{table}[H]
    \centering
    \caption{Prévision du cours du blé et du nickel en 2020 par lissage exponentiel double}
    \sffamily
    
\begin{tabular}{ccccc}
\toprule
& \multicolumn{2}{c}{Blé (\euro)} & \multicolumn{2}{c}{Nickel (\$)} \\
 Mois  & Valeurs prévues & Valeurs réelles & Valeurs prévues & Valeurs réelles \\
\cmidrule(r){1-1}\cmidrule(lr){2-3}\cmidrule(lr){4-5}
01-2020 & 189,45 & 191,00 & 13817,13 & 12850,00 \\
02-2020 & 192,02 & 187,50 & 13484,34 & 12255,00 \\
03-2020 & 194,64 & 196,25 & 13159,58 & 11484,00 \\
04-2020 & 197,28 & 195,75 & 12842,63 & 12192,00 \\
05-2020 & 199,97 & 188,25 & 12533,32 & 12324,00 \\
06-2020 & 202,69 & 180,50 & 12231,45 & 12805,00 \\
07-2020 & 205,45 & 182,75 & 11936,86 & 13786,00 \\
08-2020 & 208,24 & 187,75 & 11649,36 & 15367,00 \\
09-2020 & 211,07 & 197,75 & 11368,79 & 14517,00 \\
10-2020 & 213,95 & 205,25 & 11094,98 & 15156,00 \\
11-2020 & 216,86 & 210,25 & 10827,75 & 16033,00 \\
12-2020 & 219,81 & 213,25 & 10566,97 & 16613,00 \\
\bottomrule
\end{tabular}%

\end{table}


\subsubsection*{Lissage exponentiel de Holt-Winters}
La prévision par lissage exponentiel de Holt-Winters est une méthode de prévision de séries chronologiques saisonnières. La méthode consiste à effectuer un LED de Holt 
sur la partie non saisonnière, c'est-à-dire la moyenne et la tendance, et un lissage exponentiel saisonnier sur la composante saisonnalité\footnote{Voir tableau ~\ref
{tab:hw} p.~\pageref{tab:hw}}. Ici, les deux échantillons du
blé, ainsi que l'échantillon 2016-2019 du nickel étant non saisonniers, la méthode revient à un LED sur deux paramètres pour ces échantillons.\\[11pt] 
Comme pour le LED, les différentes constantes de lissage ($\alpha, \beta$) sont calculées via une minimisation de la somme des carrés des résidus et sont trouvés dans le 
tableau ci dessous.
\begin{table}[H]
    \centering
    \caption[hw19]{Constantes de lissage de la méthode HW \footnotemark}
    \sffamily
    \begin{tabular}{lcc}
        \toprule
        & Blé & Nickel\\
        \midrule
        $\alpha$ & 0,78 & 0,89 \\
        $\beta$ & 0,00 & 0,00\\
        \bottomrule
    \end{tabular}
\end{table}
\footnotetext{Voir annexe ~\ref{appendix:hw_19} p.~\pageref{appendix:hw_19} }
Ainsi la prévision pour 2020 peut être faite :
\begin{table}[H]
    \centering
    \caption{Prévision du cours du blé et du nickel en 2020 par lissage de Holt-Winters}
    \sffamily
    \begin{tabular}{ccccc}
      \toprule
      & \multicolumn{2}{c}{Blé (\euro)} & \multicolumn{2}{c}{Nickel (\$)} \\
       Mois  & Valeurs prévues & Valeurs réelles & Valeurs prévues & Valeurs réelles \\
      \cmidrule(r){1-1}\cmidrule(lr){2-3}\cmidrule(lr){4-5}
01-2020 & 187,26 & 191,00 & 14272,31 & 12850,00 \\
02-2020 & 186,99 & 187,50 & 13306,19 & 12255,00 \\
03-2020 & 186,72 & 196,25 & 11664,18 & 11484,00 \\
04-2020 & 186,46 & 195,75 & 11825,33 & 12192,00 \\
05-2020 & 186,19 & 188,25 & 12081,62 & 12324,00 \\
06-2020 & 185,93 & 180,50 & 12792,85 & 12805,00 \\
07-2020 & 185,66 & 182,75 & 13731,23 & 13786,00 \\
08-2020 & 185,40 & 187,75 & 14456,35 & 15367,00 \\
09-2020 & 185,14 & 197,75 & 14663,90 & 14517,00 \\
10-2020 & 184,87 & 205,25 & 14916,73 & 15156,00 \\
11-2020 & 184,61 & 210,25 & 14698,81 & 16033,00 \\
12-2020 & 184,35 & 213,25 & 16106,49 & 16613,00 \\
\bottomrule
\end{tabular}%

\end{table}
\subsubsection{Choix de la meilleure méthode}
Il désormais nécessaire de sélectionner la meilleure méthode de prévision pour chacun des cours. En effet, la meilleure méthode sera ensuite utilisée pour prévoir les
cours de l'année 2022. Pour ce faire, le critère de comparaison utilisé est le MSE, ce dernier est calculé comme la moyenne des erreurs quadratiques. la prévision 
minimisant le MSE sera sélectionnée.
\begin{equation*}
    \text{MSE} = \frac{1}{n} \sum_{i=1}^{n} (Y_{i} - \hat{Y}_{i})
\end{equation*}
Où $n$ est le nombre de périodes prévues (ici 12), $Y_{i}$, les valeurs réelles et $\hat{Y}_{i}$, les valeurs prévues. Il est également possible de calculer le critère
RMSE tel que : $\text{RMSE} = \sqrt{\text{MSE}}$, ce dernier permettant de mieux évaluer la distance moyenne entre les valeurs prévues et les données empiriques.
\begin{table}[H]
    \centering
    \caption{Critère MSE et RMSE pour la prévision des cours du blé et du nickel en 2020}
    \sffamily
    \begin{tabular}{lrrrr}
\toprule
& \multicolumn{2}{c}{Blé} & \multicolumn{2}{c}{Nickel} \\
Méthode & \multicolumn{1}{c}{MSE} & \multicolumn{1}{c}{RMSE} & \multicolumn{1}{c}{MSE} & \multicolumn{1}{c}{RMSE} \\
\cmidrule(r){1-1}\cmidrule(lr){2-3}\cmidrule(lr){4-5}
Extrapolation & 78,19 & 8,84 & 7501793 & 2738,94 \\
LED   & 161,02 & 12,69 & 9445216 & 3073,31 \\
Holt-Winter & 192,20 & 13,86 & 525124 & 724,65 \\
\bottomrule
\end{tabular}

\end{table}
Ici, pour le blé, la meilleure méthode de prévision d'après le critère MSE est la prévision par extrapolation d'une droite de tendance. Pour le nickel la meilleure méthode
est celle du lissage de Holt-Winters. Ces deux méthodes sont sélectionnées.
\subsubsection{Prévision pour 2022}
Les méthodes retenues sont donc utilisées sur les échantillons 2019-2016 du blé et du nickel pour prévoir les cours de 2022. L'objectif ici étant de prévoir 2022 avec des
échantillons ne comportant pas la période de crise sanitaire liée à la pandémie de Covid-19.
\begin{table}[H]
    \centering
    \caption{Prévision du cours du blé et du nickel en 2022 avec échantillons ante-Covid-19 }
    \sffamily
    % Table generated by Excel2LaTeX from sheet 'MSE'
\begin{tabular}{ccccc}
      \toprule
      & \multicolumn{2}{c}{Blé (\euro)} & \multicolumn{2}{c}{Nickel (\$)} \\
      Mois & Valeurs prévues & Valeurs réelles & Valeurs prévues & Valeurs réelles \\
      \cmidrule(r){1-1}\cmidrule(lr){2-3}\cmidrule(lr){4-5}
      01-2022 & 214,01 & 266,00 & 22595,97 & 22328,00 \\
      02-2022 & 214,96 & 322,50 & 23029,39 & 24282,00 \\
      03-2022 & 215,91 & 369,50 & 23471,11 & 32107,00 \\
      04-2022 & 216,87 & 400,75 & 23921,31 & 31771,00 \\
      05-2022 & 217,83 & 392,25 & 24380,15 & 28392,00 \\
      06-2022 & 218,79 & 350,25 & 24847,78 & 22698,00 \\
      07-2022 & 219,76 & 343,00 & 25324,39 & 23619,00 \\
      08-2022 & 220,73 & 332,25 & 25810,14 & 21411,00 \\
      09-2022 & 221,71 & 356,75 & 26305,20 & 21107,00 \\
      10-2022 & 222,69 & 352,25 & 26809,76 & 21809,00 \\
      11-2022 & 223,68 & 326,50 & 27324,00 & 26987,00 \\
      12-2022 & 224,67 & 309,25 & 27848,10 & 30048,00 \\
      \cmidrule(r){1-1}\cmidrule(lr){2-3}\cmidrule(lr){4-5}
      MSE & \multicolumn{2}{c}{16645,29} & \multicolumn{2}{c}{19816354,21} \\
\bottomrule
\end{tabular}%

\end{table}
Les MSE ci dessus seront par la suite comparés aux MSE des méthodes utilisées sur les échantillons (2016-2021) afin de vérifier si la période de Covid19 à réellement eu un impact sur la prévision du cours de 2022.
\subsection{Échantillon 2016-2021}
La démarche empruntée ici est la même que celle de la sous-partie précédente. L'objectif est de prévoir les cours du blé et du nickel
pour l'année 2022, les échantillons utilisés seront les échantillons couvrant 2016 jusqu'à 2021. Prévoir grace à ces échantillons, permet d'intégrer à la
modélisation la période de crise liée à la pandémie de Covid-19. La situation conjoncturelle n'ayant pas réellement connue d'accalmie en raison du changement climatique et 
en particulier de la guerre en Ukraine, l'ajout de la période Covid-19 permettra potentiellement aux différentes méthodes de mieux intégrer les variations importantes.
\subsubsection{Prévision pour 2022}
De manière analogue, les trois méthodes traditionnelles utilisées sont l'extrapolation d'une droite de tendance, le lissage exponentiel double, et 
le lissage exponentiel de Holt-Winters. Elles seront par la suite comparées entre elles ainsi qu'à la prévision faite pour 2022 dans la partie précédente par le biais du 
critère MSE.
\subsubsection*{Extrapolation d'une droite de tendance}
Les paramètres de la droite de tendance sont estimés grace à la méthode des MCO, ils sont pour le blé et le nickel tous significativement différents de 0. 
les résidus des droites sont pour les deux matières premières soumis à de l'hétéroscédasticité ainsi que à de l'autocorrélation, malgré cela ils sont normalement 
distribués. Les résidus du blé et du nickel ne suivent donc pas un bruit blanc, cependant le modèle est quand même utilisé pour prévoir 2022, les paramètres étant 
tout de même significatifs\footnote{Voir annexe ~\ref{appendix:extra_21} p.~\pageref{appendix:extra_21}}.
\subsubsection*{Lissage exponentiel double (LED)}
Les constantes de lissage pour les séries de blé et de nickel sont calculés et minimisent la somme des carrés des écarts prévisionnels. Pour le ble $\lambda = 0,37$, cela 
veut dire que la mémoire du phénomène est forte, une plus grand pondération est appliquée aux observations passées. Pour le nickel, $\lambda = 0,5$, cela veut
dire que la mémoire n'est ni forte, ni faible, la prévision accorde autant d'importance au passé que au présent\footnote{Voir annexe ~\ref{appendix:led_21} 
p.~\pageref{appendix:led_21}}.
\subsubsection*{Lissage exponentiel de Holt-Winters}
Ici, le lissage de Holt-Winters pour le blé revient à un LED sur deux paramètres : la moyenne et la tendance. Les constantes de lissages sont calculées de la même façon que
pour un LED\footnote{Voir annexe ~\ref{appendix:hw_21} p.~\pageref{appendix:hw_21}}: 
\begin{align*}
    \alpha &= 0,73 & \beta &= 0
\end{align*}
En revanche, l'échantillon 2016-2021 du nickel présente de la saisonnalité additive, dans ce cas là l'échantillon \guillemotleft brut\guillemotright \,(non corrigé des 
variations saisonnières) est utilisé, et la composante saisonnière est donc lissée par un lissage exponentiel saisonnier de Winters. Les constantes de lissage sont :
\begin{align*}
    \alpha &= 0,9 & \beta &= 0 & \gamma &= 0
\end{align*}
Les valeurs prévues du cours du blé en 2020 par les trois méthodes sont calculées et dans le tableau ci dessous :
\begin{table}[H]
    \centering
    \caption{Prévisions du cours du blé en 2022 par différentes méthodes de prévision}
    \sffamily
    % Table generated by Excel2LaTeX from sheet 'Sheet3'
\begin{tabular}{ccccc}
\toprule
\textit{(en \euro)} & Valeurs prévues & Valeurs prévues & Valeurs prévues & Valeurs \\
Mois  & par Extrapolation & par LED & par HW & Réelles \\
\cmidrule(lr){1-1}\cmidrule(lr){2-4}\cmidrule(lr){5-5}
01-2022 & 234,33 & 292,14 & 280,78 & 266,00 \\
02-2022 & 235,80 & 301,04 & 282,51 & 322,50 \\
03-2022 & 237,28 & 310,20 & 284,25 & 369,50 \\
04-2022 & 238,76 & 319,65 & 286,00 & 400,75 \\
05-2022 & 240,26 & 329,38 & 287,76 & 392,25 \\
06-2022 & 241,76 & 339,41 & 289,53 & 350,25 \\
07-2022 & 243,28 & 349,75 & 291,31 & 343,00 \\
08-2022 & 244,80 & 360,40 & 293,11 & 332,25 \\
09-2022 & 246,33 & 371,37 & 294,91 & 356,75 \\
10-2022 & 247,88 & 382,68 & 296,73 & 352,25 \\
11-2022 & 249,43 & 394,33 & 298,55 & 326,50 \\
12-2022 & 250,99 & 406,34 & 300,39 & 309,25 \\
\bottomrule
\end{tabular}%

\end{table}
Il est dans un premier temps ici facilement remarquable que les modèles traditionnels ont relativement du mal a prévoir l'année 2022, hautement volatile dû à l'invasion 
Russe en Ukraine, le critère MSE permettra donc de discriminer la meilleure méthode. \\[11pt]
Les valeurs du nickel sont pareillement calculées et dans le tableau ci dessous :
\begin{table}[H]
    \centering
    \caption{Prévisions du cours du nickel en 2022 par différentes méthodes de prévision}
    \sffamily
    % Table generated by Excel2LaTeX from sheet 'Sheet3'
\begin{tabular}{ccccc}
\toprule
\textit{(en \$)} & Valeurs prévues & Valeurs prévues & Valeurs prévues & Valeurs \\
Mois  & par Extrapolation & par LED & par HW & Réelles \\
\cmidrule(lr){1-1}\cmidrule(lr){2-4}\cmidrule(lr){5-5}
01-2022 & 18780,42 & 21472,00 & 21039,45 & 22328,00 \\
02-2022 & 19149,75 & 22376,20 & 21558,85 & 24282,00 \\
03-2022 & 17740,34 & 21185,63 & 20351,32 & 32107,00 \\
04-2022 & 18229,36 & 22248,81 & 20923,61 & 31771,00 \\
05-2022 & 18230,73 & 22740,26 & 20797,59 & 28392,00 \\
06-2022 & 19132,73 & 24390,70 & 21631,74 & 22698,00 \\
07-2022 & 20504,99 & 26715,48 & 23198,64 & 23619,00 \\
08-2022 & 21845,07 & 29087,93 & 24336,37 & 21411,00 \\
09-2022 & 20485,09 & 27877,46 & 23365,18 & 21107,00 \\
10-2022 & 21129,49 & 29387,35 & 23984,39 & 21809,00 \\
11-2022 & 20272,23 & 28815,69 & 23297,93 & 26987,00 \\
12-2022 & 20883,14 & 30337,46 & 23608,95 & 30048,00 \\
\bottomrule
\end{tabular}%

\end{table}
L'analyse tacite est similaire à celle de la prévision du blé, le critère MSE permettra de choisir la meilleure méthode. Il reste cependant à remarquer que pour les deux 
matières premières, l'extrapolation d'une droite de tendance semble être la méthode la moins adaptée.
\subsubsection{Choix de la meilleure méthode}\label{sec:mse}
Afin de sélectionner les meilleures de prévisions pour le blé et le nickel en 2022, les critères MSE des différentes méthodes sont comparées. La prévision minimisant le 
critère sera choisie pour prévoir 2023.
\begin{table}[H]
     \centering
     \caption{Critère MSE et RMSE pour la prévision des cours du blé et du nickel en 2022}
     \sffamily
     % Table generated by Excel2LaTeX from sheet 'MSE2'
\begin{tabular}{lrrrr}
\toprule
& \multicolumn{2}{c}{Blé} & \multicolumn{2}{c}{Nickel} \\
Méthode & \multicolumn{1}{c}{MSE} & \multicolumn{1}{c}{RMSE} & \multicolumn{1}{c}{MSE} & \multicolumn{1}{c}{RMSE} \\
\cmidrule(r){1-1}\cmidrule(lr){2-3}\cmidrule(lr){4-5}
Extrapolation   & 11427,19 & 106,90 & 57040109 & 7552,49 \\
LED & 2609,42 & 51,08 & 35361885 & 5946,59 \\
Holt-Winter & 4069,21 & 63,79 & 33115542 & 5754,61 \\
\cmidrule(r){1-1}\cmidrule(lr){2-3}\cmidrule(lr){4-5}
Prévision (2016-2019)  &16645,29 & 129,02 &	19816354 &	4451,56 \\
\bottomrule
\end{tabular}%

\end{table}
Pour le blé, la méthode qui minimise le critère RMSE est celle du lissage exponentiel double. Pour le nickel, c'est la méthode de Holt-Winters sur l'échantillon
2016-2019 qui minimise le MSE. L'ajout de la période de pandémie de Covid-19 aura été utile pour prévoir le cours du blé, mais pas le cours du nickel.\\[11pt]
Les méthodes et échantillons choisis sont donc :
\begin{itemize}
    \item\textbf{Blé}  : Lissage exponentiel double sur l'échantillon 2016-2021.
    \item\textbf{Nickel}  : Lissage exponentiel de Holt-Winter sur l'échantillon 2016-2019.
\end{itemize}
Ces méthodes seront comparées dans la prochaine partie aux prévisions obtenues avec la méthode de Box et Jenkins afin de répondre à la problématique.\\
[11pt]
En théorie, si aucune information n'est perdue lors de la prévision, les résidus des deux prévisions devraient se comporter statistiquement comme un bruit blanc normal. 
Les différents tests sur les résidus sont fait\footnote{Voir annexe~\ref{appendix:traditest} p.~\pageref{appendix:traditest}}.
\subsection*{Test d'absence d'autocorrélation des résidus}
Le test utilisé est le test de Ljung-Box, ce test permet de vérifier si les résidus d'un modèle de série temporelle présentent une corrélation significative jusqu'à un 
certain ordre $k$. Il utilise les autocorrélations des résidus jusqu'à un nombre donné de retards $k$ et les compare à une distribution du chi-deux pour évaluer la 
significativité de la corrélation. \textbf{Spécification du test :} 
    \begin{align*}
        H_{0} &: \text{ Absence d'autocorrélation des résidus à l'ordre $k$.} \\
        H_{1} &: \text{ Autocorrélation à l'ordre $k$.}
    \end{align*}
\textbf{Statistique de test :} 
    \begin{equation*}
        Q = n(n+2)\sum_{k=1}^h\frac{\hat{\rho}^2_k}{n-k} \sim \rchi^{2}_{0,95}(k)
    \end{equation*}
\textbf{Règle de décision :} Si la valeur de la statistique Q est inférieur au seuil lu dans la table du khi-deux alors $H_{0}$ est acceptée au risque de 5\%, il y a autocorrélation à l'ordre $k$.\\
Ici pour le blé et le nickel, les corrélogrammes montrent que pour n'importe quel nombre de retards $k$, la probabilités d'accepter $H_{0}$ est inférieures à 5\%. $H_{0}$ est donc rejetée au risque de 5\%,  il y a autocorrélation des résidus pour les deux modèles\\
\subsection*{Test d'homoscédasticité}
Le test utilisé est le test ARCH, il permet de tester si les résidus d'un modèle sont homoscédastiques. Le test consiste à réaliser une régression auxiliaire des carrés des résidus, en les retardant d'un certain nombre de périodes, puis tester la significativité des paramètres de la regression grace à un test $F$ ou un test $LM$.\\
\textbf{Spécification du test :} 
    \begin{align*}
        H_{0} &: \text{ Homoscédasticité des résidus.} \\
        H_{1} &: \text{ Hétéroscédasticité.}
    \end{align*}
\textbf{Statistique de test :} 
    \begin{align*}
        LM_{ble} &= nR^{2} = 77 \times 0,83 =63,65        & LM_{nickel} &= 77 \times 0,42 =32,00
    \end{align*}
\textbf{Règle de décision :} la statistique $LM$ est comparée au quantile lu dans la table du khi-deux pour un niveau de test à 5\%, ici $\rchi^{2}_{0,95}(7) = 14.07$. Si elle est inférieure, alors $H_{0}$ est acceptée au risque de $5\%$ \\ 
Ici pour les deux modèles $LM$ est supérieure, $H_{0}$ est alors rejetée au risque de 5\% pour les deux, Les résidus des modèles de prévision du blé et du nickel sont hétéroscédastiques.
\subsection*{Test de normalité}
Le test utilisé est le test de Jarque-Bera, le test vérifie si une distribution est normalement distribué. Ce test est basé sur une mesure de l'asymétrie et de l'aplatissement de la distribution, appelée le statistique de Jarque-Bera ($JB$).\\
\textbf{Spécification du test :} 
    \begin{align*}
        H_{0} &: \text{ Les résidus suivent une distribution normale.} \\
        H_{1} &: \text{ Les résidus ne suivent pas une distribution normale. }
    \end{align*}
\textbf{Statistique de test :} 
\begin{align*}
    JB_{ble} &= \frac{n}{6} \left( S^2 + \frac14 (K-3)^2 \right) = 138,46 & JB_{nickel} &= 141,12
\end{align*}
Où $S$ est le coefficient d'asymétrie (\textit{skewness}) et $K$ le coefficient d'aplatissement (\textit{kurtosis}).\\
\textbf{Règle de décision} : La statistique $JB$ est comparée au quantile à 95\% de la distribution de khi-deux ayant pour degrés de libérté 2, ici $\sim \rchi^{2}_{0,95}
(2) = 5,99$. Si la statistique est inférieure au seuil critique $H_{0}$ est acceptée au risque de 5\%.\\
Ici dans les deux cas la statistique de Jarque-Bera est largement supérieur à 5,99, $H_{0}$ est donc rejetée pour les résidus du le blé et le nickel, ils ne suivent pas 
une distribution normale normalité.\\[11pt]
Les résidus des prévisions sont hétéroscédastiques, autocorrélés et ne sont pas normalement distribués. Ils ne suivent donc pas un bruit blanc, une partie de 
l'information est donc perdue lors de la prévision.