\documentclass[11pt,a4paper]{article}
\usepackage[T1]{fontenc}
\usepackage[utf8]{inputenc}
\usepackage[]{geometry}
\usepackage{afterpage}
\usepackage{tikz}
\usepackage{fancyhdr}

\usetikzlibrary{fadings}
\definecolor{light}{HTML}{E0E3C2}
\DeclareFixedFont{\titlefont}{T1}{ppl}{b}{n}{1.2cm}
\DeclareFixedFont{\subtitlefont}{T1}{ppl}{m}{n}{0.6cm}
\DeclareFixedFont{\namefont}{T1}{ppl}{m}{n}{0.6cm}
\begin{document}

\afterpage{\restoregeometry}
\newgeometry{left=1in, right=1in,top=1in, bottom=-0.5in}
\definecolor{mytan}{HTML}{1C3651}
\pagecolor{mytan}\afterpage{\nopagecolor}
\thispagestyle{empty}
\begin{center}
\includegraphics{um3.png} \hspace{5em} \includegraphics[scale=0.23]{eco.png} \par
\vfill
\titlefont \textcolor{light}{Projet d'économétrie appliquée}\par\vspace{0.5cm}
\subtitlefont \textcolor{light}{Prévision des cours du blé et du nickel}
\vfill
\namefont \textcolor{light}{Mosse Joseph - Rubira Pierre}\par\vspace{0.2cm}
\namefont \textcolor{light}{M1 - MBFA - ARB}
\vfill
\namefont \textcolor{light}{Sous la direction de :}\par\vspace{0.2cm}
\namefont \textcolor{light}{Seyte Françoise}
\end{center}

\begin{center}
\begin{tikzpicture}
\node[scope fading=north, inner sep=0pt, outer sep=0pt]{
 \makebox[\textwidth]{\includegraphics[width=\paperwidth]{ble2.jpg}}
};
\end{tikzpicture}
\end{center}

\clearpage
\section*{Résumé}
\setcounter{tocdepth}{6}
\renewcommand\contentsname{Sommaire}
\tableofcontents
\newpage
\pagestyle{fancy}
\fancyhead{}\fancyfoot{}

\section*{Introduction}
\section{Analyse macroéconomique du blé et du nickel}
\subsection{Le blé}
\subsection{Le nickel}
\section{Analyse des séries chronologiques}
Les méthodes traditionelles de prévision, reposent sur la décomposition des différentes composantes d'une série temporelle. Ici il s'agira donc
ici d'analyser ces différentes composantes (c'est à dire la tendance et la saisonnalité).
\subsection{Transformation logarithmique}
Avant tout, il est nécessaire de s'affranchir des fluctuations importantes de la série. Pour cela un test d'homoscédasticité est fait sur la série 
initiale.
\subsection{Analyse graphique}
Correlo ?
\subsection{Analyse de la variance}
\subsubsection{Test de Fisher sur la tendance}
\subsubsection{Test de Fisher sur la saisonnalité}
\section{Prévision par le méthodes traditionnelles}
\subsection{Lissage exponentiel double (LED)}
\subsubsection{Période 2016-2019}
\subsubsection{Période 2016-2021}
\subsection{Lissage exponentiel triple (Holt Winter)}
\subsubsection{Période 2016-2019}
\subsubsection{Période 2016-2021}
\subsection{Extrapolation d'une droite de tendance}
\subsubsection{Période 2016-2019}
\subsubsection{Période 2016-2021}
\subsection{Classification des méthodes}
\subsubsection{Blé}
\subsubsection{Nickel}
\subsection{Prévision pour 2023}
\subsubsection{Blé}
\subsubsection{Nickel}
\section{Prévision selon la méthodologie de Box \& Jenkins}
\subsection{Présentation de la méthode}
\subsection{Test de racine unitaire}
\subsection{Indentification des processus}
\subsection{Tests de validité}
\subsubsection{Significativité des paramètres}
\subsubsection{Tests sur les résidus}
\subsection{Prévision pour 2023}
\section*{Conclusion}


faire par sous périodes



\end{document}
